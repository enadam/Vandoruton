\clearpage
\chapter{TRANSZFORMÁCIÓ \\ Jobbra lépek bal helyett}

\section{Lent}

Amikor már minden jel arra mutat, hogy nagy baj van, mert már szétfeszül
a mellem, minden éjjel felébredek rá, miután az egész nyarat
végigszenvedtem, mert még a télen hiába toporogtam a rendelő ajtajában,
hogy mégis talán most kellene, de rám szóltak, hogy most akarja a
műtétet? -- így eloldalogtam, tehát akkor most, ősszel rettegve visszamegyek
az onkológiára, mert nem szükséges semmilyen vizsgálat, én
már tudom, tudtam már a télen is, hogy valamit nagyon elrontottam.

Másnap telefonon tárgyilagosan közlik velem, hogy holnap legyek
szíves összecsomagolni és befáradni a Kékgolyó utcába. Mennyire
találó utcanév ehhez az intézményhez. Mint a mesebeli királykisasszony,
aki kútba ejti az aranygolyóját, és a kút mélyéről egy gusztustalan
béka hozza fel, akinek kívánságai vannak az aranygolyóért cserébe.
A királykisasszony pedig rákényszerül arra, hogy teljesítse a béka
kívánságait.

Jó mélyen bele kell néznie azoknak a kútba, a Kékgolyó utcában, akik
vissza akarják kapni a golyójukat, vagyis egész-ségüket, akár életüket.

Összevitatkozom az orvossal, hogy semmiféle amputációba nem
megyek bele. Én is tárgyilagosan közlöm, hogy ezt a nyilatkozatot
pedig nem írom alá, minden részem az enyém, és én döntök róla. Ő is
vitázik, hogy ő pedig jobban tudja, hogy mi a megfelelő teendő, ha műtét
közben kiderülne egy s más. Végül pufogva aláírom.

Másnap kábultan ébredek a műtőasztalon, bal felem szorosan bekötözve.
Válaszolnak valamit, mire én zokogva visszaalszom, de ekkor
sírok először és utoljára a ballépésem miatt.

A kórházi ágyon ébredek legközelebb, tétován nézek körül, tehetetlenül,
csövekkel az oldalamon. Emlékszem, mosolyogtam. Egy nagyon
erős belső erő áramlik bennem, és egyértelműen érezhető, hogy ez az
erő nőnemű.

Csodálkozva járkálok házi öltözetben a betegtársak között, és elhűlve
hallgatom sopánkodásukat. Sajnálom és fájdalmat érzek azok
iránt, akikre rá van írva, hogy lemondtak az életről. Számomra fel sem
merült, hogy így is alakulhat.

\bigskip
\begin{itshape}
Mert nem aszerint nő vagy csökken az esély a gyógyulásra, hogy milyen
súlyos a diagnózis, hanem attól, hogy szíved mélyén hiszed-e, hogy neked itt
még feladatod van, amit kér tőled a béka. Vagy inkább elfordulsz tőle, nem
akarsz változni és feladod, mert elfáradtál.

Senkit nem tudsz becsapni, a betegséget vagy a békát főleg nem. Szemedbe
vigyorog, ha nem vagy őszinte, és még közelebb jön arcodhoz, hogy
rémülj csak meg, halálra.

Amikor azonban te fordulsz vele szembe azzal, hogy elég volt -- a kéréseket,
feladatokat megértettem és elvégeztem, a kívánságod már teljesült --, és
a békát a falhoz vágod, mint a királylány a mesében, a béka királyfivá lesz,
és a sorscsapás eséllyé változik.
\end{itshape}

\section{Fent}

A Kék Golyó -- Földünk. Hogyan értelmezzük szabadságunkat vele
szemben? Miféle sejtjei vagyunk testének? Mert ne gondold, hogy
nincs élete és nincsenek szervei ugyanúgy, mint nekünk, embereknek.
Testrészei szorosan összekapcsolódnak, és viszik a híreket földrengéseken,
árvizeken, hurrikánokon át, és tovább. Már észrevette a parazitát,
mely túlontúl elszaporodott, és anyagyilkosságra készül. Még
csecsszopó a gyermek, de fenemód agresszív. Azt hiszi a kisember,
anyja teje kiapadhatatlan. Kilométeres csövekkel szívja életerejét,
idegen anyagokkal szennyezi méhét, és felsír: még, még!

Megeshet, hogy a kozmoszból egyszer orvost hívat, és beutalót kap
ő is a Kékgolyó utcába. Ott megvizsgálják a doktorok, és éles késsel
kimetszik a lázadó területet. Az önző ember nem veszi észre, hogy az
egészhez tartozik millió szállal, nem működhet másképp a rész, mint
az egész. Adni is kell, nemcsak elvenni, buta gyerek. Kicsi vagy még,
anya nélkül semmire se mész, vele halsz te is.

\bigskip
\begin{itshape}
Vedd fel a golyót, melyet elejtettél, még mielőtt messzire gurul, és a béka
nem tér vissza vele, mert nincs alku többé.
\end{itshape}

\section{Utolsó hívószó}

Nyolc év múlva nagyot zuhanok. Egy hívó szóra -- vagy nevezzük megérzésnek
-- mégis egy újabb év haladékot kapok.

A sötét barlang mélyének fénylő szembogarában pislog a halálfélelem
reszketegen. A tenger vize feketén örvénylik, és egyre beljebb tolul
a járatokban. A szikla gyomrában összegömbölyödve nézi a manó a
közelgő áradatot. Mi lelt, kis manó? Cselekedj! Ússz e haláltengeren
a fény felé, hagyd el óvó menhelyed méhét, mert vár rád a bizonytalan
végtelen a biztosnak vélt vég helyett.

A fizikai pusztulás traumájával szembesülve a testi-lelki pokolra
szállás időszaka jön el. Számvetés az élettel, szembesülés az elmúlással.
Mégis egy olyan időszak, mely jelentős továbblépés sorsom
útján.

\bigskip
\begin{Verse}
Újjászületés -- A Tejút vöröslő Antares csillagának ajándéka	\\
\itshape
A padláson állok kígyóbőr testben, felkészült lelkem.		\\
Szétnyílik a nyakamon, s mintegy láthatatlan cipzár,		\\
már a köldökömig nyitva áll.					\\
Nem mozdulok, mint minden állat, ki halni készül,		\\
elrejtőzve, magamban vagyok.					\\
A régi ruha alatt ugyanaz a minta,				\\
de szemem fénye mélyebbről és tüzesebben ég ma,			\\
születésnapom van ma, újra.
\end{Verse}

\bigskip
Minden megfordul a mindennapjaimban. Elhagyom az eddigi hivatásomat,
és végre teljes egészében az írás felé fordulhatok. Kapok egy
nagyszerű társat, és több évtizedes kapcsolatok tisztulnak le,
megerősödnek vagy szétszakadnak.

Köszönet a teremtőnek e rendező időszakért, az összegzésből nyert
tanulságokért.

\section{Csokoládé}

Nem tudok ellenállni a pralinénak és a forró csokoládénak. Mágnesként
vonzanak a francia Riviéra vagy a brüsszeli csokoládéüzletek
kirakatai. A szezon szerinti csokoládészobrok, a színes papírba,
mignonkosárba csomagolt édességek.

Az orrom majdnem az üveghez nyomul, a szemem egészen kikerekedik,
és gyermeki énem feléled. Az ajtón át kiosonó illat orron fog, s mint
egy hipnotizáltat a csokoládés pult elé vezet. Meg akarom ismerni a
titkát, akár egy gyerek, amikor kibelezi a kisautót, mert keresi azt a
valamit, amitől a járgány messzire gurul, ha megnyomja a sofőr fejét.

A Rue du Lombardstraat-i bolt ajtaja csilingelve nyílik. Charlotte
fehér főkötőben és kötényben invitál bennünket a bemutató szalonba.
A pult egy régi üzlet berendezése, rajta nagy tálakban csokireszelék
és drazsék. Egy fémüstben sötét színű, folyékony csokoládé mennyei
illattal. Charlotte belekezd a kakaó történetének elbeszélésébe interaktív
elemekkel tűzdelve. Egy-egy helyes válasz jutalma egy markolás
a tetszés szerinti tálból. Hasznos tudnivaló, hogy legkevésbé a fekete
csoki hizlal, leginkább a fehér.

Pralinékészítés következik. Felveszem a kötényt, és az üstből egy
merőkanálnyit beleöntök a kis tégelyekbe a téglalap alakú lemezen.
Egy nagy késsel végighúzom a tetejét. Újra öntök és újra húzok. A finom
csokoládé úgy tapad, hogy még két nap múlva is sikálom a körmöm alól.

Kóstolás! Epres, barackos, banános, kávés, folyékony csoki a csokiban.
Lassan olvasztom a számban, és forgatom. Minden alkalommal
meglepetés a töltelék. Fehér porcelánkannában a forró csokoládé, fekete
csokiból, tiszta tejszínnel, csipetnyi fahéjjal, melytől enyhén kókuszízt
kap, és megkoronázza a tesztet.

Boldogsághormonok szabadulnak fel, látható gyönyöröket varázsolva
a résztvevők arcára. A csoportterápiának vége. Elégedetten és
elégülten köszönjük meg Charlotte-nak a délutáni örömöket. S mint
B.~úr a piros lámpás házból távozóban, somolyogva, könnyed léptekkel,
sétapálcánkat lengetve biccentünk vissza az ajtóból.

Ez élvezet volt, nem szégyellem.
\clearpage

\newlength{\origparindent}
\setlength{\origparindent}{\parindent}
\savebox{\illustrationImage}{\includegraphics[width=0.35\textwidth]%
				{images/page-131}}

\noindent
\parbox{\wd\illustrationImage}{\usebox{\illustrationImage}}\hfill
\begin{minipage}{\textwidth-\wd\illustrationImage-\columnsep}
\setlength{\parindent}{\origparindent}
\itshape
Most pedig cseréld ki a pálcát a caduceusra.
Ne lengesd. Állj meg és emeld fel. Ehhez
tudom, idő kell.

A kígyók feltekerednek, a fehér és a fekete
csokoládé egységbe olvad.

Most már az egész csokoládébolt a tiéd.
Kheirón, te lettél a főnök, és te rendezed a
kirakatot. A pralinék pedig engedelmesen
várnak, amíg neked más dolgod van.
\end{minipage}

\bigskip
\begin{Verse}
\textbf{Tavasz Eleusziszban}

\bigskip\itshape
Delfinen lovagló gyermekem dús redőjű ruhájában				\\
Eleusziszba vezet, ím itt a misztérium helye és ideje.			\\
Hádész lejárata fölött így zeng a fáklyás gyereksereg:			\\
Visszaveszem tőled éltem, eljött az én időm,				\\
Napfényre vágyom, tavaszra, éj, huss, hess, sötétség.			\\
A termékenység hava, anya, a fordulat pillanata.			\\
Zeusz segít benned rabságodból szabadulni, jönni,			\\
Virágozni, újra termékennyé válni, teremni és teremteni,		\\
Beporozni a világot, gyermekedet derűvel és fényességgel.		\\
Hagyd indulni, tisztítani, mosni, vetni,				\\
Gondoskodni, aratni és rendet tenni.					\\
Fordul a sorskerék, Démétér, jöjj, segíts neki,				\\
Nyisd a kaput, táltos fehér paripák ugranak fel Hádész bejáratán át.	\\
Itt vagyok, megszülettem, madarak füttyében jöttem.			\\
Madarak a hírnökeim, kik szétrepülnek menten,				\\
S trillájuk hét határban, szérűn, mezőn megfogan.
\end{Verse}
