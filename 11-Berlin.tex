\clearpage
\chapter{BERLIN}

1944-ben egy berlini utca sarkán álló bérházból kilép egy fiatal nő,
hogy megszokott élelmiszerszerző útját megtegye. Az utóbbi hónapokban
felforrósodott a levegő a városban, sűrű bombázások alatt áll
Berlin. Nem érti a háború miértjeit, keserűen, de erős belső tartással
viseli a körülötte történteket. Férje már jó ideje nem adott hírt magáról.
Kilép a kapun. Már megint szirénáznak, szinte el sem hallgat ez
a gyomor közepéig ható, fülsüketítő zaj. Mégis nekivág a kereszteződésnek,
mert élelmet kell vinnie az otthon hagyott kisfiúnak. Szalad,
ahogy csak bír, a kiégett, romos területen, talán órák is eltelnek, mire
szerez egy darab húst. Rohan vele haza, többször is megbotlik a
törmelékeken. Ezekben a percekben, a magasban egy pilóta megnyom egy
gombot, és ezzel hosszú időre megpecsételi ennek a nőnek a sorsát.
A nő egyre jobban kapkodja a levegőt a futástól, és attól a rossz érzéstől,
hogy valami éppen most veszik el, talán örökre. Az utcasarokhoz
érve már egészen kifullad, majd mintha már egyáltalán nem kapna levegőt.
A bérház hatalmas lángokban áll, jó része szétszóródva omlott
rá az utcára. Majdnem megőrül a fájdalomtól. Pár nappal, egy héttel,
vagy lehet, hogy egy hónappal később talán az utolsó bombák hullanak
a városra. A nő elhalad egy autó mellett, még tesz pár lépést, majd
valami hátulról őrült erővel megnyomja. Óriási robajt hall, lángokat
lát a lába alatt. Forró a levegő, egyre forróbb, úgy érzi, szétolvad a
környezete, és ő maga is. Szétég a tüdeje.

A berlini eseményekkel párhuzamosan Budapesten, a Széchenyi
rakpart 7. szám alatti háromemeletes bérházban már jó ideje lehurcolkodtak
a lakók az óvóhelyre. A rakpart és az Árpád utca sarkán
álló ház kapuja a Duna felé nyílik, innen nem lehet elhagyni a házat,
mert Budáról folyamatosan lövik a pesti oldalt. Egy tíz és fél éves kisfiú
ki-kijár a hátsó kapun a környékre, és tanúja lesz egy ló halálának,
amelyet eltalál egy Budáról érkező lövés. Az emberek azonnal elkezdik
feldarabolni az állatot. A kisfiú lélekszakadva fut haza a nagy fazékért.
Akkora lóhúst kap a fazekába, hogy visszafelé el is ejti, többször
is megbotlik a törmelékeken. Az óvóhelyre érve kiabál: ,,Anyu, anyu,
nézd, mit hoztam: húst! -- gyorsan főzd meg!''

Nem sokkal később a házat gyújtóbomba-találat éri. Az óvóhelyről
az egyetlen menekülési lehetőség, ha kibontják a szomszédos bérház
pincéjével összekötő alagutat. A laza, csupán téglákkal tömített járatot
gyors kezek megtisztítják, és a lakók egyesével átbújnak a szűk járaton.
A kisfiú nem akar menni. Anyjába kapaszkodva, toporogva sír. De
nincs sok idejük, ég a ház fölöttük. A fiút megfogják és begyömöszölik
az alagútba. Fuldokolva húzzák ki a másik oldalon.

Húsz év múlva a férfinak gyereke születik. Fiút várnak, a kis Lacikát.
Nagy baba lesz, a vékony nő már órák óta vajúdik vele. Minden
erejével tolja kifelé, de a baba megakad a szülőcsatornában. Fogóval
kell kihúzni a kislányt, aki már fuldoklik odabent.

Eltelik újabb húsz, harminc, negyven év is, a kislányból felnőtt nő
lesz. Nem születik gyereke. Soha nem is érezte, hogy neki gyermeket
kellene szülnie. Más feladata van. Sokáig nem tudja, mi az, de vár.
Vár egy jelzésre, amely majd meghozza számára a választ. Egyébként
is mindig valamilyen megmagyarázhatatlan félelem töltötte el egy
gyermekkel kapcsolatban, hogy elveszíti, hogy nem tudna rá vigyázni.
A nő nagynéni lesz. A baba másfél hónappal korábban születik meg.
Rátekeredik a köldökzsinór a magzat nyakára, és csak császármetszéssel
kerülhető el a fulladása. A nagynéni szívét forróság járja át, ahányszor
csak ránéz az angyalarcú, szőke, kék szemű jövevényre. Remeg,
amikor kézbe veszi, nehogy elejtse, nehogy valami baja legyen. Amikor
ő volt kicsi, édesapja egyszer rosszulléte során leejtette. Az óvóhelyi
alagútban kapott sokk hatására szerzett életre szóló betegsége miatt.
A nő ma sem tud meredek hegyre félelem nélkül felmászni, mindig
attól retteg, hogy visszacsúszik.

Teljesen logikátlan tanácsokat ad öccsének, hogy hogyan kell megóvni
egy kisfiút. Mindenekelőtt nem szabad egyedül hagyni, mert az
nagyon veszélyes. Amikor megtudja, hogy a gyerek egyedül jár haza
az iskolából, nem kap levegőt a döbbenettől, izzad, és remegni kezd
a gyomra az aggodalomtól. Mindeközben a nő elindul egy hosszú
úton, egy hosszú utazáson, és egyre közelebb kerül egy nagyon fontos
ponthoz. Amikor a kisfiú nyolcéves lesz, a nagynéni elviszi őt Párizsba.
Ketten utaznak, ez a fiú első repülése. Egyedül a nőé a felelősség,
hogy a gyereket épségben hazahozza Párizsból. Minden idegszálával
arra koncentrál, nehogy elhagyja. Egy már nem használt céges belépőkártya
elejét átragasztja. Rákerül a fiú neve, nemzetisége és a nő
telefonszáma, majd egy rábeszélő mese segítségével a kisfiú nyakába
akasztja. A gyerek látótávolságon kívülre nem kerülhet. Az anyatigris
mellett ugyanakkor felbukkan a cigány lány szerepe is. Meg szeretné
mutatni a fiúnak a szabadságot, ablakot nyitni előtte a világra, hogy
feltámadjon benne a vágy, az igény a szárnyalásra. Két év múlva Londonba
utaznak, rá egy évre Erdélybe. A kis tábla még mindig ott függ
a gyerek nyakában.

Az erdélyi út után egy héttel a nőn életmentő műtétet végeznek.
Szörnyű emésztőszervi diagnózisokat kap, majd három hónappal később
azt is közlik vele, hogy tüdőáttétje van.

\bigskip
\begin{itshape}
Milyen el nem engedett érzelmi traumák terhelik még mindig? Meg
akar fulladni? Na de miért és kiért? Hányadszor próbálkozik valaki más
helyett fuldokolni, akin ezzel már úgysem segíthet? Milyen körtáncot járat
vele a történelem, milyen ördögi kerékben -- szamszárában -- forog, és
mióta? Ekkor felismer egy sorsa szempontjából nagyon fontos dolgot, ami
megfordíthatja sorsának kerekét egy másik irányba. Meddig tart a személyes
felelősség, és hol kezdődik a kollektív, az emberiség vezetőinek felelőssége a
tömegek felett? Akik százezrek, milliók körtáncához indítják el a zenét?
Véletlenül, vagy igencsak tudatosan összecserélik az indítógombokat, és a
káosz indulója szólal meg. Melyik oldalon állsz, ez a kérdés értelmetlen.
Egyszer Berlinben állsz, legközelebb Budapesten vagy Londonban, vagy
bármely más viszonylatban. És ha Berlinben becsomagolnak neked valamit,
azt érezni fogod Budapesten is, de lehet, hogy már nem fogsz rá
emlékezni, kitől kaptad ezt a csúf ajándékot. És szenvedsz, felpakolod a
hátadra, viszed más, mások terhét a válladon, a tüdődben, a combodban,
a beleidben, attól függ, hová helyezed. Vissza kellene küldeni ezt a kéretlen
csomagot, mert te nem lehetsz felelős a rossz indításért. Világ vezetői, jól
figyeljetek, visszaküldöm e terhet, mert téves a címzés. Vezessetek úgy, hogy
amikor postát bontotok, azt nyugodt szívvel tehessétek, mert az elindított
csomagok egyszer visszatérnek a feladóhoz. És az egész emberiség fog
válaszolni nektek. Ha nem talál meg benneteket Berlinben vagy Londonban,
megtalál majd Budapesten, vagy bárhol másutt az univerzumban, akárhol,
ahová úgy hiszitek, hogy elmenekülhettek.
\end{itshape}
\bigskip

A nagynéni gondolatban Berlinbe utazik. Megkeres egy utcasarkot,
hogy ott egy kisfiút, aki már régen szárnyakat kapott, elengedjen. A nő
elhagyja a múltat, hogy végre ne fájdalmakkal élje meg a jelent, és
építhesse a jövőt, hogy szárnyakat adhasson egy másik fiúnak. Berlinben
és Budapesten ma a jövő háza áll. Ezt jelenti számára ez a város.

A nőnek tulajdonképpen már nem kell utaznia. Teljesen más csatornákon
át surran be szobájába a világ. Egy fotelben ülve utazik a legtávolabbi
vidékekre, ahol eddig még nem járt, és legmerészebb álmaiban
sem álmodta e távolságokat és időhorizontokat.

Ha Berlin nevezetességeire vagy kíváncsi, kedves olvasó, kérlek, fedezd
fel őket egymagad. Mi lenne, ha Berlinbe utazva nem olvasnál el
egyetlen útikönyvet sem, csak a megérzéseidre hagyatkoznál? Sétád
közben nem az útikalauzba merülve, hanem előrenézve bontakozna ki
előtted a város. Nem számítanád ki előre, hogy melyik utcasarok után
milyen látvány kell, hogy fogadjon. Teljesen lényegtelenné válna, hogy
mi pontosan hány méter széles és hosszú. Hagynád, hogy a város maga
mutassa meg számodra ezerfényű arcát.

Én nem tudok többet írni Berlinről, mert mindent leírtam a jövő
városáról, ami számomra lényeges.

Mialatt e sorokat rovom, a fiatal fiú felvételi vizsgát tesz egy hatosztályos
gimnáziumba. Hogy milyen pályára fog lépni pár év múlva,
még nem tudja. Humán beállítottságú, kiválóan írja meg a magyar irodalom
felvételit, és a szóbelin a történelmi kérdésekre is tájékozottan
felel. A matematika írásbeli azonban gyatrán sikerül. Őszintén válaszol,
hogy ő bizony nem hagyja abba a vízilabdázást, azért, hogy matematikából
felzárkózhasson. Majd felteszik neki a kissé korai kérdést:
Mi szeretnél lenni, ha nagy leszel? A fiú kis gondolkodás után rávágja:
tűzoltó. A szóbelin részt vevő szülők a háttérben csodálkozva néznek
egymásra.

A család még nem tudja, hogy tollam alatt mindeközben lángokban
áll Berlin és Budapest. Bérházak és autók égnek ki, emberek fulladnak
meg a füstben, és levegő után kapkodnak alagutakban. A fiú segíteni
szeretne.

Így kapcsolódik össze formátlan és formát öltött világunk. Gondolatainkon
keresztül, felszakadt időréseken át a formátlan világ rezgései
elérik az arra fogékony lelkeket, és megfogalmazódnak, megnyilvánulnak
az anyagi létben azáltal, hogy válaszolnak rájuk.

A te gondolataid is szárnyra kapnak és megtalálják párjukat a nagyvilágban.
Figyelj oda, hogy milyen, egyelőre láthatatlan kapcsolatrendszert
építesz.

A fiú lehet, hogy tényleg tűzoltó lesz, de valószínűleg más értelemben,
mint ahogy azt a vizsgabizottság értette.

\chapter{KRAKKÓ}

\chapter{A MAGAS-TÁTRA}

\chapter{KIJEV}

\chapter{SZENTPÉTERVÁR}

\chapter{PRÁGA}

\chapter{SZLOVÉNIA}

\chapter{ISZTRIA}

\chapter{A GARDA-TÓ}

\chapter{ERDÉLY}

\chapter{A MOLDVAI KOLOSTOROK}
