\clearpage
\chapter{NEW YORK -- MANHATTAN}

Éppen taxi után fordulok, amikor az addig csendes, kihalt Wall Street
óraütésre életre kel. Nagy tömegben hullámzanak ki az utcára a fehér
galléros menedzserek. Odafutok egy közeledő taxihoz, nyitnám az ajtót,
de egy kéz rácsap a kilincsre, és határozott mozdulatával odébb
söpör. \textit{It's mine}\footnote{Ez az enyém.} -- közli röviden,
és mire feleszmélek, a taxi utasával együtt elrobog.

Az öltönyös, aktatáskás, jól fésült férfi egyetlen mondatában benne
van New York karaktere. Határozottan férfias, azonban kissé durva
az őserők ököljogán. Emellett lendületes, energikus metropolis, felfelé
törekvő toronyházaiban megmutatkozó libidója révén kivívja csodálatom.
Faragatlan, de egyenes, nem csinál fölösleges köröket, nyíltan lép
eléd. A Wall Streeten, ahol a Kos fejjel megy a falnak, nincs haditerv,
egyezkedés, megkerülés, hanem azonnali roham, és kemény feje valóban
áttör vastag falakat.

Nyers ereje felvillanyoz, és én gyorsan tanulok. Egy félórás eredménytelen
hajsza után elemi erővel ébred bennem a harcos. A Broadway
vár ma este, nincs több vesztegetni való időm. A Trinity térhez
érve feltűnik egy újabb sárga jármű. Egy elnyújtott lépéssel benne
vagyok a mátrixban, nyitom az ajtót. ,,It's mine'' -- mondom nyugodt, de
ellentmondást nem tűrő hangon a szintén odaérkező ballonkabátos,
napszemüveges, pengeélű arcnak, és beszállok a kocsiba. Csak egy
fél pillantást vetek a lemaradt idegenre, és lenyomom a telefonhívást.
Pillantásom másik felét a sofőrnek szánom kihívó felszólítással: húsz
percünk van, hogy a szállóig érjünk, menni fog? \textit{Yes, ma'am}%
\footnote{Igen, asszonyom.} -- kapom a
helyes választ egy fogpasztareklámba csomagolva. A sofőr teljesíti is az
ígérteket, miután a csúcsforgalmi időszakban még csak próbalövéseket
végeztem a vadászaton. Az autóáradat mostanra széjjelebb húzódott,
ütemesen haladunk Manhattan belseje felé.

A concierge biztosít róla, hogy az egyik legjobb helyet szerezte meg
a fennmaradókból, és abszolút sikerdarabra készülhetek, szép lesz az
estém. Sietve, fél óra alatt elkészülök, és New York utcái is megengedik,
hogy késés nélkül elérjek a színházig. A pénztárnál ellenőrzöm a
jegyárakat, és összegzem, hogy harmadáron vehettem volna itt belépőt
-- feltéve ha kapni még --, mint amennyit a kényelmem miatt kiadtam
érte, de ez az árkülönbség természetes. Életem legdrágább színházi élménye
előtt állok, s mint ahogy a szülők lányuk esküvőjekor szorgosan
számolnak, majd mégis minden extrát beleadnak a költségvetésbe, én
is nyögök egyet, majd nagyvonalúan megvonom a vállam, az idő most
értékesebb, mint a harmadárú jegy.

A marsikus várost egy másik harcos, szintén Kos-energiákkal rendelkező
csoport rohanta le, amikor repülőgépekkel hajtott neki a magasra
törő falaknak, nem törődve az áldozatokkal. Hasonlóan sok más
történelmi és jelenkori eseményhez, a Kos csillagkép üzenetében rejlő
áldozat szükségességét félreértve itt is másokkal hozatták azt meg,
a játékot nem fair play alapon játsszák. A World Trade Center, azaz
a Világkereskedelmi Központ nevű épület üvegfalain át vajon milyen
előretörő magatartást tükrözött a közeledő repülőgép képe? A Kos
csillagkép üzenete egónk áldozatáról szól. Rájövünk-e arra, hogy milyen
áldozatot kell hoznunk azért, hogy utunkat, tevékenységünket,
munkánkat áldás kísérje? A tudatosan vállalt áldozat válik áldássá.
E borzalmas eseményt követően példa erre a New York-i tűzoltók
önfeláldozó munkája, mellyel százakat mentettek meg az életnek. De
érdemes-e hasonló traumatikus történésekre várni, hogy egy robbanás
hozza fel a mélyből e jelentéstartalmakat? Látogatásomkor még büszkén
állt ez az épület New York közepén, és én ámulattal repültem fel a
száztizedik emeletére. Él-e még a liftesfiú, aki menet közben lelkesen
és humorosan mesélt az épület méreteiről és adottságairól? Él-e még a
pincérfiú és élnek-e azok az üzletemberek, akikkel együtt költöttem el
ebédemet a W.~T.~C. egyik földszinti éttermében? Előttem van a kép,
mint egy békebeli fénykép a budapesti New York kávéházból. Vasalt
öltönyös, divatos nadrágkosztümös urak és hölgyek szerepük szerinti
testtartással és mozdulatokkal tárgyalnak, csevegnek, újságot olvasnak
vagy csak bámészkodnak a hivatali szünetben.

A kilátó teraszán szédülök egyet, mielőtt stabilan beállok erre a magasságra.
A Szabadság-szobor egy miniatűr alkotás, az autók matchboxok.
Nem látok el a megapolisz határáig, körülöttem felhőkarcolók
sűrűje, távolabb is házak tömegei. Óriásként nézek le innen a dinamikus,
akaraterős, mozgalmas törpebirodalomra. New York alaptermészetem
lenyomata, tükör számomra is fény- és árnyoldalaival.

Pöllner
%\footnote{Dubravszky László Dr. -- Eörssy János Dr.:
%\textit{A tradicionális asztrológia tankönyve}, Bioenergetic~Kft.,
%Budapest, 2001, 238. old.}
szerint New York a Rák jeléhez kapcsolható, vagyis a
Hold uralma alatt áll. Azon túlmenően, hogy New York kikötőváros,
Manhattan pedig egy szigeten terül el, máshogyan nem érzékeltem
a Hold által képviselt tulajdonságok jelenlétét. Annál inkább
marsikus, uranikus és merkúri analógiákat, melyek rímelnek az
Egyesült Államok horoszkópjának 1-es házára, ahol az Ikrek aszcendensen
rajta áll az Uránusz, és első házas Mars-állással rendelkezik.
New York pedig miért ne képviselhetné Amerika első házát,
hiszen szimbolikusan New York az USA kapuja, mintegy bejárat az
Újvilágba. Az Európából kivándorlók a hosszú hajóút után elsőnek
a Szabadság-szobrot pillantották meg, és nem New York, hanem
Amerika földjére léptek. A Szabadság-szobor talpazata tizenegy
ágú csillag. A Vízöntő, melynek eszméje a Szabadság, Függetlenség,
Testvériség, bolygója pedig elsősorban az Uránusz, a 11. zodiákus
jegy. A Vízöntő jelölheti spirituális szinten az emberiség szellemi
vezetőjét, de alacsony szinten képviseli a forradalmárokat és a terroristákat.
S mindjárt itt vagyunk Uránusz bokaficamánál, mármint
,,én azért egy kicsit egyenlőbb vagyok''.

New York galériái tele vannak a modern művészet ,,termékeivel'',
melyek Uránusz szabad és meredek oldalát mutatják. A város nyitott
minden különcségre, új szellemiségre, bizarr figurára. Uránusz a szórakozott
professzor, Woody Allen világa. S ha Mars találkozik Uránusszal,
akkor amit tudni akarsz a szexről, azt itt megtudhatod, sőt meg
is tapasztalhatod a \textit{Szex és New York} nyomdokain lépdelve. Uránusz a
szingliség megtestesítője, aki nem szereti lekötni magát, a szabad vagy
éppen szabados szerelmek híve. Nem pepecsel sokat a konyhában sem.

A Planet Hollywoodban vagy a Hard Rock Caféban a meghökkentő
falfeliratok alatt, ahol elférnek egymás mellett a szanszkrit om jel,
a ,,this is not here'', ,,all is one'' vagy ,,who do you love'' szövegek,
a megrendelt étellel együtt a számla is az asztalra kerül, s ott rezeg a
levegőben a gyors emésztést kívánunk sürgető búcsúmondata is.

Merkúrral találkozunk a világ vezető cégeinek kereskedelmi irodáiban,
s itt állt a Világkereskedelmi Központ. Szárnyas Hermész,
vagyis Mercurius, a hírvivő a világ egyik vezető hírügynökségénél,
az Associated Press székházában, és az országos televíziós társaságok
stúdióiban is megjelenik. Az amerikai újságírók legfőbb kitüntetésének
számító Pulitzer-díj névadója, a magyar származású Joseph
Pulitzer szerkesztő tollát is Merkúr vezette. A híres publicista különleges
szerepét jelzi Uránusszal való kapcsolata, hiszen szintén ő volt
az, aki a New York-i lakosokat adományozásra ösztönözte a Szabadság-szobor
talapzatának elkészítéséhez. Magát a szobrot a franciák
adományozták Amerikának.

Csakúgy, mint Uránusz, New York is csak egy villanásnyi időre
mutatta meg magát, de nagy fényerővel és dörejjel villámlott, hogy e
rövid, de nem mindennapi randevúban összesűrítve láttassa lényegét.

\chapter{KALIFORNIA}

San Diego, San Francisco, Yosemite Nemzeti Park, Las Vegas, Grand
Canyon, Los Angeles

\chapter{LATIN-AMERIKA}

\begin{itshape}
Ha nem nézed az órád, több időd van és nyugodtabb vagy, mert belső órád
pontosabban jár, és mindenre jut időd, ami fontos. Amit nem végeztél el, az
nem is volt lényeges. Belehelyezkedsz a létbe, ahelyett, hogy percenként
továbbugrálnál az óramutatóval. Csak tedd a dolgod, és megnyúlnak a pillanatok.
\end{itshape}
