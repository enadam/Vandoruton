\clearpage
\chapter{KAMAKURA}

\section{Cseresznyevirágzás}

\begin{LVerse}\itshape
Festmény. Közel az origóhoz, az ébredés friss leheletével.		\\
Parfüm. Az üdeség lágy érintése, finom, alig érezhető homeopátiás szer.	\\
Gondtalanság. Kék ég, babaillat, éteri tisztaság.			\\
Tükör a tó. A föléje boruló ágak rózsaszín ruhájú gésák hajlongása.	\\
A gésák fiatal bőrük illatos selymét hullajtják rá.			\\
A szirmok tengere rajta mozdulatlan, elfedi a mélységet.		\\
A felszínen járunk, most indultunk. Minden lehetőség rajta van a vásznon. \\
Méhek jönnek, s velük megmozdul az ág. Erre a lökésre vártak a körhinták. \\
Elindulnak nyugvópontjukból körbe, s fölfelé.				\\
A lányok hátravetett fejjel, harsány nevetéssel kísérik röptét.		\\
Az anya a bölcsőhöz lép, s a Tündék elrepülnek.				\\
Beköszöntött és hódított a tavasz. Előkészült a nászra.
\end{LVerse}

\bigskip
\begin{itshape}\otherfamily
A Fuji-filmre vetített mű címe: \textup{Cseresznyevirágzás Japánban}.
Festette az ég 1996 áprilisában.
\end{itshape}
\vfill

\illustration{page-022}{Cseresznyevirágzás Japánban}

\section{Sinto -- az istenek útja}

A szertartás első mozzanata:

Lelépek a körhintáról, és a forrásnál egy kimonót terítek eléd. Hullámzik
a selyem, mielőtt földet ér, és dús mintája életre kél.

A szivárvány színeiből, a vízből és a formátlan anyagból kipattan a
Menny és a Föld. Az őseredeti rend önmagából létrejött, és kamikkal
népesül be a virágos rét.

Ég atyja és Föld anyja gyermeke, Amateraszu napistennő felrepül a
Mennybe, a többi istenség, jók és pusztítók, sziklahasadékba, faodúba,
barlangok mélyére, a szelet meglovagolva, a fű harmatába, az esőcseppbe,
egy villámlásba, ravasz állatokba, az ősatya szellemébe visszavonul.
A tisztálkodás végére benépesül a panteon.

A szertartás második mozzanata -- ima:

Az örökzöld ligetben átlépek a szent kapun, és a szentélyben imámmal
segítségül hívlak és kiengesztellek benneteket. Áldást kérek tőletek,
segítő kamik, és ajándékaimmal távol tartalak benneteket, ártó
kamik. Kettőtök létezése a haladás és a változás lehetősége. Tisztán
és harmóniában törekszem élni veletek. Fudzsi-jama isten és a Pilis
dobogó kövei erősítsék szívem.

Ünnepség:

Amateraszu szentélyét a férfiak a vállukra veszik, és hullámzó táncot
járva vonulnak a toriin, a szent kapun át belépve végig a főutcán.
A faragott, aranyozott oltár szalagokkal, óriási bojtokkal, piros és
arany-lila kötélfonatokkal, feliratos lampionokkal ékes. Virágokkal és
szalagokkal feldíszített az utca. Szimbolikus jelekkel telerajzolt
öltözetükkel színpompás felvonulás a férfiak rituális tánca. Több oltár is
utazik. A hordozható szentélyek rúdjaira többen kapaszkodnak, mint
ahányan azt egyenesen vinni tudnák, így az oltár imbolyogva, csilingelve
halad előre. Amateraszu aranykakasa ül az oltártetőn. Ébresztő
szavára a napistennő eltolja a követ barlangja elől, és az eléje tartott
tükörbe nézve ismét visszatér az égre.

\section{A zen rövid története}

A zen a mahajána buddhizmusból fejlődött ki. Számos más bódhiszattvát
(Japánban boszacut) is elismer Gautama Buddhán kívül, és
azt vallja, hogy az önmagunkban kifejlesztett erő mellett segítségül
hívhatjuk ezen szellemek segítő erejét is a megvilágosodáshoz.
Alkalmazkodóképessége és liberalizmusa révén terjedt el, magába
olvasztva a hindu kulturális hatásokat és konfuciánus elemeket. Így él
ma is más archaikus vallások mellett, mint például a sintó, egyesítve
isteneiket. A zen Kínából, egy saolin kolostorból indult útjára, ahol
egy Bódhidharma nevű szerzetes egy csupasz falat nézve kilenc évig
meditált. Több bölcs is átvette a csendes szemlélődés eme technikáját,
de a VII. században élő Huj-neng lett minden további zen iskola
alapítója. Hirtelen megvilágosodást elősegítő gyakorlatokat fejlesztett
ki, az irracionalitást szabadjára engedőeket, melyek fölötte állnak
az ésszerűnek. Hakuin Ekako mester fejlesztette ki a zen egyik
leghíresebb gyakorlatát, a találós kérdéseken történő elmélkedést.
Van egy állítás, de amikor áttennénk a gyakorlatba, kiderül, hogy
ellentmondás, ún.~kóan. Erre nem számít az ember, feltámad benne
a kétely, és így jut el egy intuitív felismeréshez, a megvilágosodás
transzcendens birodalmába. Néha heteken, hónapokon keresztül
összpontosít a tanítvány egy fejtörős kérdés megfejtésére.
A lelkigyakorlatok mellett ugyanolyan jelentőségű a lótuszülés meditációs
testtartása. A kolostori hagyományokra épülő zen a XII. században
jelent meg Japánban, és két jelentősebb szektát hívott életre, a rindzai
és a szótó iskolát. Eiszai buddhista pap Lin-csi filozófiáját
tanulmányozva és kínai látogatásait követően először Kiotóban, majd
Kamakurában megalapította a rindzai zen szektát. Kamakurában
ebben az időszakban szerveződött egy új katonai rend, a szamurájok
közössége, melynek megfelelt a zen bölcselet jellemformáló erkölcsi,
etikai tanítása. A szótó a szélesebb tömegeket szólította meg.
Meditációs technikája a hosszú ideig elnyújtott és nehéz testtartásokkal
gyakorolt ún.~zazen.

A zenben az igazság azt jelenti, hogy a valóságot olyannak látjuk,
amilyen, minden belemagyarázás nélkül. Az ég kék, a fű zöld.
Három részből áll az állítása, melyből az első kettő tagadja a valóságot.
Az első, hogy a forma üresség. A második, hogy nincs üresség
és nincs forma, a harmadik pedig, hogy az üresség az üresség.
A cél a Buddha-természet elérése, a teljes megvilágosodás, a szatori.
A mindenséggel való egylényegűség, amikor megszabadulunk a szeretet
és a gyűlölet torzító hatásától. A mindenség azok előtt tárul
fel, akik az univerzum értelmébe bele tudják illeszteni saját
létezésük értelmét.

Részlet Dógen zen mester \textit{Fukanzazengi} (\textit{A zazen
elveinek általános kifejtése}) című munkájából:

,,\dots semmibe se avatkozzatok bele, és hagyjatok fel minden üggyel.
Ne gondoljatok jót, vagy rosszat. Ne helyeseljetek, és ne utasítsatok el.
Szüntessétek meg a tudatos elme műveleteit, ne latolgassatok, ne
formáljatok véleményt. Ne tervezzétek, hogy Buddhák lesztek\dots''

\section{Kóan -- Az aranyhal meséje}

Mögöttem a patakban az aranyhalak az áramlással úsznak. -- Vissza
miért fordulnék -- szól ki az egyik. -- A folyó így is fölfelé visz, s az
Andromédát elhagyva hazatérek vele. Megfordul egy másik az Alrisánál.
-- Még tisztulnom kell a korsó kiömlő vízénél. A megtörténteket
újra átélem, és a Pegazussal együtt szárnyalok. Rálátok szeretet-erőimre,
s ikráim megváltóként leteszem eléd.

Elfordulok a faltól, de a patakban csak egy halat látok. A meditációnak vége.

\bigskip
\begin{Verse}
Hát azt hiszed, hogy mindenre megkapod a választ? \\
Eredj, és találd meg magad. \\
A ráció gúzsba köt, keresd, de ne akard a megoldást. \\
Ha elengeded, a kérdés vitorlát bont, és kiköt a válasz. \\
Íme az én megfejtésem, de te csak eredj, lusta tanuló!
\end{Verse}

\bigskip
Vasakaratú küzdőszellem, azt mondják, erős személyiség. Csendes,
befogadó, ma úgy mondják, gyenge jellem. Messze még a cél, vasakarat
mégis megtörik, a küzdő a felkavart hullámokon elbukik, mert ereje
csupán sérült egója halványuló visszfénye. A befogadó ereje végtelen,
észrevétlen úszik a nyugodt, sima vizű tengeren. Egy a szigetekkel, az
óceán mélyével, az esőcseppel. Nem törekszik célt érni, mert már rég
az Egyben, az Egységben él.

\section{Zen kolostorok}

A Minamoto-ház a szamurájokra támaszkodva fokozatosan félreállította
a császárt, és 1192-től megkezdődött a sógunátusok hét
évszázadig tartó korszaka. Kamakura-korszaknak nevezik a hét
évszázad első 141 évét. Ezt követően más-más családok hosszabb-rövidebb
időre magukhoz ragadva a hatalmat új sógunátusokat alapítottak,
először a császárvárosban, Kiotóban, később az ország más
vidékeire helyezték át a fővárost. A sógunátusok alatt az addig főleg
a művészeteknek hódoló császári kormányzás helyébe katonásabb
erkölcsi felfogás lépett.

A zen buddhizmus épen maradt építészeti alkotásainak egyike az
Engakudzsi-templom. Az 1282-ben alapított templomegyüttes Kamakura
egyik leghíresebb kolostora, művészien elrendezett épületcsoportjainak
köszönhetően. A kínai színezett kapuktól eltérően a zen
kolostorok bejáratát festetlen, de jelképekkel sűrűn átszőtt domborművek
jellemzik. Jellegében a székely kapukhoz hasonlít, de magasabb
és méretesebb is. A kétszintes faragott kapuépítményen belépve az utat
gondosan rendezett kertek övezik. Mint minden művészien megtervezett
zen kert, arra vár, hogy megfejtsük titkait. Sziklák, hullámfodrosan
elgereblyézett kavicsok, kövek, homok, mohák, füvek, cserjék,
vízmedencék, szigetek, hidak, pavilonok, tavak, ösvények, lépőkövek,
vízesések. Céljuk, hogy szimbolikus szavakkal fejezzék ki a természetet,
segítenek, hogy annak mélységére összpontosítsunk.

Nevezetes épülete a Buddha szent fogát őrző csarnok, a Sariden.
A Kamakura-korszak szimbolikus templomharangja az ohgane. A kínai
és japán harangoknak nincsen nyelvük, kongatásuk kívülről történik.
A fából épült ,,magaslesen'', a tetőre függesztett harang mellett
nagy farúd is függ vízszintesen, mellyel oldalról ütik meg a szent
hangszert. A lélek templomának építészeti harmóniája mélységes békét
teremt az elme számára. Ma is szerveznek a templom területén zen
kurzusokat, és gyakorolják a zen művészeteket.

A zen a művészeteket a meditációs technika segítőinek tartja azáltal,
hogy a művészi munkát és cselekvést szigorú összpontosítás és
szertartásos viselkedés hatja át. A teakészítést, a tintafestést, az
ikebanát, a tájkertészetet, a harcművészeteket átitatja az ezotéria
szellemisége. A természet szeretete a japán életvitel minden területén
megmutatkozik, és áthat minden művészeti ágat is.

\section{Zen íjászgyakorlat}

\begin{itshape}
Mikor céloddal eggyé válsz, kezed elengedi a megfeszített íjat, és együtt
repülsz a nyílvesszővel a középpontba.
\end{itshape}

\section{A Nagy Buddha}

A Kamakura-kor a japán szobrászat aranykora. A Végtelen Fény
Buddhája először fából készült, de egy vihar hamarosan tönkretette.
Ekkor bronzba öntötték Daibucut, a Nagy Buddhát, és egy csarnokot
építettek fölé. Azonban a védelméül szolgáló templomot is elsodorta a
szökőár, ezért többé nem építették fel. A Buddha-alak monumentális
méreteivel, kezét ölébe kulcsolva ül a szabadban ötszáz éve. Lábazata
előtt kétoldalt bronz lótuszvirág nyílik. Temploma az erdő és a díszkert,
mely feléje vezet. A templomban éjjel a csillagok a gyertyák. Nappal
a kő- és bronzlámpások észrevétlenül haloványak a természetes
fénnyel ölelkező mester lábai előtt. A kerek dombok között az Örökkévalóság
kapuja. Méretei megengedik, hogy belsejében lépcső vezessen
fel a válláig, a hátán két kis ablakon lehet kitekinteni. Rétegesen
öntötték bronzba, és a rétegeket később egymásra illesztették. A 760
évvel ezelőtt készült mestermunka egyetlen aszimmetriája, hogy bal
füle hat centiméterrel rövidebb a jobb oldalinál.

\fullpageillustration{page-029}{A kamakurai Daibucu, azaz Nagy Buddha}

\section{A szerencse fája}

A szerencse, a szerelem és a művészetek istennőjének barlangszentélyénél
jósszalagot húzok egy fahengerből. Szerencsés papirost választottam,
megtartom a jóslatot. A kevésbé tetsző szöveget kapó vendégek
a jóslatot felkötik egy fára. A fa ágain papírszalagok százait fújja
a szél. Jupiter megengeszteli testvéreit, s csípőjén békésen lengenek a
beteljesületlen kéretlen levelek.

\section{Aikido = a szeretet ereje}

\begin{itshape}
A Naprendszer legnagyobb ereje, a Plútó áll előttem. Az Oroszlán nyakán
ül vadító nyakörve. Felfűzöm rá a múlt összes tanúját, és megrántom
vele a szekér rúdját. Az Oroszlán szívénél az Uránusz villanó fényében
csillagszekérré változik. A múlt szekerét előrehúzom, és a legmagasabb
megújító erőt aratom. Egy jobb, igaz világról álmodom, érte a szeretet
koronaékköveit a Földön szétszórom. Lábam alatt észrevétlen csillannak, de
szívemben váratlan fényt gyújtanak. Lehajolnod érte fölösleges, a szikrázó
drágakő kezedben kővé dermed, szívünkben lángol fel éltető ereje.
Kiáradok és befogadok, bennem az érkezőt megforgatom, s mint bumerángot
visszafordítom. A világ rajtam fogást nem talál, mert átengedem magamon,
s köpenyét maga felé fordítom. Miért hasalsz előttem földhöz teremtve,
mikor meghajolnék szeretet-erőid előtt?
\end{itshape}

\section{Szumi-e köszöntő Tokióból}

Ucsidától ünnepi köszöntő érkezik. A fekete tussal rajzolt tintafestményen
egy daru száll az útján előrehaladó teknősbéka fölött. Egymásra néznek.

Sorsom éberen követik, s szerencsét hintenek elém, ha hírnökként
a betűk szent titkait a Földnek átadom.

\chapter{BANGKOK}

\chapter{SZINGAPÚR}
