\clearpage
\chapter{PEKING}

\section{A pekingi repülőtéren}

Kilépek a gépből. Az áprilisi szellő belekap kendőmbe, és arcom elé
fújja. Kínai selyem, még születésem előttről. A tavaszi napfény gyengéden
végigsimítja arcom, langyos csókkal illet. A szemem előtt csupán a
poros kifutópályák, semmi kézzelfogható romantika nincs a közelben.
Mégis melegség önt el, olyan, mint mikor egy hosszú útról hazatérsz
és tüdőd beszívja az ismerős levegőt. Lelépek a földre, és már tudom,
hogy hazaérkeztem, az a kis porfelhő újra köszönt engem.

Pedig nem készültem erre, meglepetés volt, egyszeri, máig nem ismétlődő.

\section{A sanghaji szobrászfiú}

A Nyári Palota előtt, a pénztárnál kérem a belépőjegyet. A jegyárus
szenvtelen arccal adja ki a kis ablakon át a visszajáróval együtt. Nyúlok
érte, de egy kéz finoman a kezemhez ér és megállítja. Elhangzik pár pattogó
mondat a jóképű fiú és a pénztáros között, a visszajáró pedig megduplázódik.
Ji bemutatkozik, és bemutatja öccsét is. Sanghajból jöttek tanulmányútra
Pekingbe, most végzett a Művészeti Főiskolán. Udvariasan megkérdezi,
hogy lehet-e a kísérőm, mesélne a palotáról, jól ismeri történeteit, és
készítene rólam pár fényképet. Beleegyezem örömmel, Ji pedig mesélni kezd.
Visszarepülünk az időben, és az egyes palotarészeknél el is játssza az
előadott történeteket. Lesi minden rezdülésemet, és minden kis nevetés után
mosolyogva kihúzza magát, olyan, mint egy oroszlán, mikor kölykeivel játszik:
látjátok, ez vagyok én, hagyom, hogy szórakozzatok. Nagyon figyel,
ritka lehetőség, mint mondja, hogy egy európai nővel cseréljen gondolatokat.
Sokat kérdez a szokásainkról, és büszkén meséli élettervét. Öccse alárendelt
szerepet játszik mellette, viheti a fényképezőgépet és hallgathatja
beszélgetésünk. A második tekercset lövi el rólam, én meg már bánom,
hogy nem egy másik sálat vettem, és a sminkelés is elmaradt.

A palota sokszínűségét lehetetlen leírni, csak pár kiragadott kattintással
tudom érzékeltetni a fényűzést és pompát, melyet itt teremtettek.
A Csing-dinasztia nyári rezidenciájaként épült a dombokkal körülvett
tavak köré. Bejáratát, mint általában minden jelentős kínai épületét, két
kőoroszlán őrzi, egy hím és egy nőstény. A hím mancsa alatt a teljesség
jelképe, egy golyóbis pihen, míg a nőstény egy kisoroszlánt védelmez.
A fényűző életmódot jelzik a mesterségesen létrehozott tavak, a kikötőjében
horgonyzó, márványból faragott hajó, a tó fölött kecsesen átívelő,
tizenhét lyukú híd, a színesen dekorált pagodák és templomok hosszú
sora, a császári színházak és a megejtően szép, hétszázhuszonnyolc méter
hosszan elnyúló nyári folyosó. E fából épült, tetővel fedett sétatér a kínai
népmesékből, a régi kínai életből vett jelenetekkel van telefestve. Kicsiny,
részletgazdag rajzait a mennyezeten és az oszlopokon napokig lehetne
tanulmányozni és csodálni. A kert ösvényeire magnóliafák borulnak.

A hosszú séta után felajánlom Jinek, hogy jöjjenek velem taxival,
elviszem őket a szállásukig. Szabadkozik, hogy messze lesz, de én
ragaszkodom. Ahogy haladunk a város sűrűjében, a jelen és a múlt olyan
közelségben és kontraszttal csap át egymáson, hogy szédülök bele.
A biciklisták centiméterekre kerekeznek mellettünk, de egy kanyar után már
az öt-hat sávos út szélére szorulnak. A felhőkarcolók a szemünk láttára
nőnek ki a földből, ha visszanézünk a hátsó ablakon, egy emelettel már
magasabbak. Gomba módra, telepszerűen ,,termesztik'' őket. Peking
félelmetes a méreteivel, egész terjedelmével, tömegével nehezedik rám, és
pulzálása sokkoló. A főutakról letérve a keresztutcákban épphogy elfér
az autó, mindkét oldalon magas falak állnak. Körbeveszik a lakóközösségeket,
amelyekben az udvar körül négy oldalon sorakoznak teljes családok
számára a szobák, az udvar a közös konyha, és előfordul, hogy egy
latrinát használ egy egész utca. A sugárutak mögött ilyen házsorok
terebélyesednek kerületnyi méretekben. Ha a város utcáinak hosszát
összeadjuk, állítólag a Nagy Fal hosszát is eléri. Rohamosan csökken azonban
a labirintus mérete, mert a 2008-as olimpiára modern lakónegyedeket
hoznak létre a helyükön, így folyamatos a ,,kitelepítés.

Megérkezünk Ji szállásához. A sofőr hirtelen éktelen jajongásba kezd
-- a szobrászfiú fordít --, hogy megfájdult a feje, szálljunk ki
mindannyian, azt mondja, így nem tud tovább vezetni. Kiabál rám is,
én legalábbis annak veszem a hangos jajveszékelését. Rémülten és egy
naháttal nézem a pocsék színészi alakítást. Elveszettnek érzem magam
Pekingben. A reakciók szokatlanok, mert nem ismerős a mimika, nincs
meg az a metakommunikációs kapcsolat, amely megvan egy európaival,
még akkor is, ha nem beszélem a nyelvét. Számomra itt semmi nincs az
arcára írva a szembejövőnek, bensőm nem kap támpontokat, így állandó
készenlétben áll egy váratlan mozdulatra, nagyon ősi ösztönök kezdenek
mozogni. Ez az állandó készenlét pedig fárasztó. Angoltudásom
is csak a város szórványos területein ér valamit. Egyébiránt nálam van
a szálloda kártyája, az egyik oldalán angol, a másikon kínai felirattal a
cím, telefonszám. Egyszer szállok buszra egy külvárosnak tetsző helyen
valami hirtelen jött ötletből, de egy megálló múlva le kell hogy lépjek,
mert sokkol a látvány. Utóbb rájövök, hogy egy gyári munkásokat
szállító buszra tévedhettem fel. Rózsaszín tavaszi kabátomban idegen
testként állok az egyforma világosszürke kabátos, sapkás és nadrágos,
fásult tekintetű, mosolytalan emberek között. A két megálló között
majdnem felkiáltok, hogy segítség, álljon meg, mert azonnal le akarok
szállni. Nem felejtem el azt a buszt, Orwell fojtogató falanszterét.

De még mielőtt továbbindulnék, Ji szól, hogy várjak, mutatni szeretne
valamit. A szállás nyitott ajtaján át látom az emeletes ágyakat a
kis helyiségben, ahol nem szobát, hanem ágyat lehet bérelni, és újra
felmérem a város nagyságrendjét, ahol több mint egy egész Magyarországnyi
ember éli mindennapjait.

Ji egy szobrairól szóló fényképalbummal tér vissza. Kinyitja, belenézek,
és úgy érzem, ma már nem tudom tovább kapkodni a fejem
amiatt, hogy már megint változik a kép. Modern szobrok, melyek akár
a barcelonai köztereken is állhatnának.

\bigskip
\begin{itshape}
Elképzeléseink foglyai vagyunk, túl sok az idea és a festett kép, melyet
magunknak rajzolunk, és mindig át kell satíroznunk egy kicsit, ha valamilyen
új részlet előbukkan. Ahelyett, hogy eleve megengednénk, hogy lehet
ilyen is és olyan is, vagyis mindkettő egyszerre. Minél több lehetőséget
tudok belelátni egy dologba, egy emberbe, annál kevesebb különbséget fedezek
fel, a kontrasztok elkezdenek elmosódni, és közelebb leszek a dolog vagy
ember egészéhez. Ne fosszuk meg magunkat a megismerés lehetőségétől azzal,
hogy kijelentjük, valami ilyen vagy olyan. Ezáltal megmerevedik a kép,
és mi is megmerevedhetünk úgy, hogy mozdulni sem bírunk az isiásztól.

Add meg a lehetőséget, hogy másmilyen is lehessen a másik ember, mint
amit te képzelsz róla, egyébként mindkettőtöket fogva tartod, és elveszed az
esélyt nemcsak magadtól, hanem társadtól is, hogy megmutathassa minden
oldalát, és megismerhesd. Duplán véted el a célt.
\end{itshape}

\section{A Tiltott Város környékén}

A XV. században vonalzóval tervezett Tiltott Város huszonnégy császárnak
adott otthont. Központi tengelyére építették rá az Ég templomát,
a mauzóleumokat, és a császári épületeket tökéletes szimmetriában.
A régi Kína a mennyei birodalom földi mása volt, melynek
középpontjában állott a főváros, annak közepén pedig a császári palota.
Egyenes főutak kiindulópontja a négy égtáj felé. A Mennyei Béke
Kapuja, vagyis a Tiananmen tér felőli déli főbejárata, a Vu men, azaz
Meridián kapu volt a Ming- és Csing-korban a palota bejárata. A kapubástyán
tartottak nagy ünnepségeket a régmúltban, és 1949-ben a
Kínai Népköztársaság megalapításának ceremóniájához is helyszínül
szolgált. A falon belül, az Aranyvíz hídról lelépve a csarnokok, épületek,
kapuk nevében visszaköszönnek a taoizmus, konfucianizmus és
buddhizmus kulcsszavai, a harmónia, a tisztaság, az egység és a békesség.
Példaként, a Teljes Harmónia csarnokában áll a császári trón. Tíz
méter magas falai mögött 1991-ig, az utolsó, a mandzsu dinasztia bukásáig
elszigetelten uralkodtak a császárok. A terjedelmes területnek
érzékelt császári negyed valójában igencsak szűk életteret jelentett,
ha belegondolunk, hogy egy teljes élethossz fő tartózkodási helyszíne
volt. Igaz, a területe jóval nagyobb volt a mai, fallal övezett városénál.
Hozzá tartozott a Tiananmen tér egy része, a fölötte elterülő ún.
Szén-domb és a tavak vidéke. Az elszigeteltség sokszor kényszerlakhelyet
is jelentett, valóságos fogságot, de számos esetben előfordult, hogy
úgy elkényelmesedtek az uralkodók, hogy a kormányzást a korrupt
eunuchokra bízták, maguk pedig a háremeikben töltötték napjaikat.
A szűken vett császárvárosban megtekinthetjük a csarnokokat, a kőből
és márványból díszesen faragott összekötő utakat, a lakóépületeket,
parkokat, a császár használati tárgyait. Gazdag kínálatát nyújtja
egy történelmi hosszban mérve csak pillanatokkal ezelőtt letűnt kor
emlékeinek. A Tiananmen tér, a világ legnagyobb tere áll a Tiltott Város
déli főbejáratánál. Ottjártamkor hatalmas digitális óra mutatta,
hogy mennyi idő van még hátra Hongkong Kínához való 1997. július
1-jei visszacsatolásáig. A visszaszámlálás 817 napot, másodpercben 70
milliót jelzett ki. Az Országos Népi Gyűlés, azaz a parlament épülete
foglalja el a tér egyik oldalát. Vacsorameghívást kaptunk az ötezer
személyt befogadó bankett-termébe. Amikor megérkezem a bejáratához,
bal kéz felől előttem a Tiltott Város falához illeszkedő, sárga cseréptetejű
tribün (kapubástya) áll Mao Ce-tung arcképével, jobbra a hatalmas tér,
a kettőt elválasztó műúton biciklisták százai kerekeznek, az
autóforgalom most gyérebb. Kedvesen fogadnak a bejáratnál, de alaposan
ellenőrzik a meghívómat, és felkísérnek az emeleti grandiózus
terembe. A nemzetközi vendégsereg betölti a helyiséget. A köszöntőbeszédek
után igazi kínai vendéglátásban lesz részünk. A középen forgó
nyolcszemélyes asztalokra egymás után kerülnek fel az ínyencségek.
A tizedik fogás után már nem is kísérletezem, hogy egyfolytában arról
kérdezgessek, mi micsoda. A főfogásokat követi a leves, majd az édességeket
szolgálják fel, és egy dalos népitánc-bemutatóval zárul az est.

Időutazást teszek még napközben, taxiban ülve, amikor elhaladok
a Tiltott Város mellett. A rozsdavörösen elnyúló fal mentén borbélyok
gyakorolják mesterségüket. A szabad ég alatt kis zsámolyra ültetik
vendégeiket, és hosszú fehér köpenyükben a járókelők és autósok szeme
láttára szőrtelenítenek hagyományos borotvakéseikkel. Olyan elegánsan
mozog a borbélyok hosszú sora, mintha csak egy drága szalonban szolgálnák
ki az urakat. Érdekes kontraszt ismét az autóforgalommal körülvett
szabadtéri fodrászüzletek kerülete. Azt hinnénk, hogy egy olyan szakma
virágzik a császári falak mellett, mely nem vett róla tudomást, hogy
lebontották a feje fölül az üzletet, elvitték a szerszámait, és csak az az egy
maradt, amelyet éppen a kezében tart. Pedig nincs így, a kínai borbélyok
nem szalonokban dolgoznak. Tamási Áron Ábel az országban című regényéből
a fogorvosra hasonlítanak, aki útitáskájában viszi a rendelőjét,
a vasúti kocsik között páciensek után kutatva. Ezt a világot hozza vissza
számunkra, európaiak számára e pekingi utcarészlet. Kínában azonban
ez nem egy filmforgatás jelenete, hanem egy ősidők óta gyakorolt mesterség
fennmaradása. Emlékek tolulnak fel, nagyanyám falujából a cipész, illetve
gyerekkorom öreg cipész bácsija a fővárosban, az iskolámnál, ahová
rendszeresen vittem sarkaltatni a lábbelimet, mert állítólag úgy jártam,
mint egy katonatiszt, a legvastagabb sarkat is lekoptattam egy idény alatt.
Az öreg kaptafán javított cipőimet legfeljebb a divat sodorta tova, Rémusz
bácsi mindig gondosan újjávarázsolt cipői elnyűhetetlennek bizonyultak.
Ma már nem találkozom Rémusz bácsival sem, és a falusi cipésszel sem.
Üzlethálózatok gyorsszervizében lemondóan nyújtják vissza a javítanivalót,
mondván, már nem érdemes foglalkozni vele.

\section{Az operában, egy sörözőben és egy karaokebárban}

A nemzetközi kiállításon segítségünkre van egy magyar fiú. Péter negyedéves
a pekingi műszaki egyetemen, kiválóan beszél kínaiul, és már könnyen
eligazodik a főváros sűrűjében. Együtt megyünk az opera-előadásra.

A hagyományos kínai színjátszás nehezen megközelíthető a más
kultúrkörben élők számára. Péter folyamatos interpretációja nélkül egy
kukkot sem értettem volna az operából. Egy kis alapismerettel azonban
könnyen követhetővé válik az előadás. Operaáriákból, érdekesen
artikulált dialógusokból, pantomimszerű mimikából és akrobatikus
csatajelenetekből, táncokból áll. Kevés díszletet használ, az is eléggé
szimbolikus, és szintén kevés hangszert, például csattogtatókat, dobot.
A szereplő karakterére vall, hogy áriát énekel, beszél vagy pantomimet
ad elő. Négyféle alapszerep létezik: a férfié, a nőé, a festett arcoké és
a bohócé. Az alapszerepeken belül az arcfestés és a színes sárkányokkal,
főnixmadarakkal gazdagon hímzett jelmezek mintázata is mutatja,
hogy az ártatlan fiatal lány vagy az amazon lépett színre. Az idős hölgynek
festetlen az arca, az idős férfit szakáll jelöli. A pozitív hős arcán sok
a piros, a negatívén pedig a fehér. Kétségtelen, hogy az előadás
leglátványosabb részei a teljes harci díszben előadott akrobatikus epizódok.
Minden meseszerű, hiszen a hagyományos kínai opera nyolcszáz éves
múltra tekint vissza, a pekingi ág százötven éve fejlődött önállóvá.
Évszázados tündérmeséket, mítoszokat visznek színre. Az előadásban
szerepelnek szellemek, majomkirály, buddhista papok, sőt maga Buddha
is megjelenhet nagy, aranyra festett ábrázattal, hatalmas aranyfülekkel.
Mennyországi és pokolbeli lények, sárkányok népesítik be a színpadot.

A színház után bevetjük magunkat a kevésbé kivilágított, távolabb eső
helyekre. Szakmámból adódóan feltérképezzük, hogy mit fogyaszt a lakosság
egy kiskocsmában, a fiatalabb generáció a különböző bárokban.
Közbevetem, hogy személyes ismeretség és ajánlás nélkül nem jutunk
messzire ezen a piacon. A nagy nehezen megszerzett pekingi telefonkönyv
másfél centiméter vastag volt az akkori tizenkétmilliós lélekszámra.
Összehasonlításként, a budapesti telefonkönyv, a nem kevés titkosított
hívószám mellett is, hat-hét centire rúg. Szakmai ún. Arany oldalakról pedig
nem is álmodozhattam, mert egyszerűen nincs nyilvántartás, a hivatalok
száma beláthatatlanul nagy, kiismerhetetlen kapcsolatrendszerrel.

Belépünk a kocsmába, vagyis egy kínai család otthonába, ugyanis
az családi vállalkozásként zárásig sörözőként üzemel. Feltehetően egy
nagyobb helyiségből áll a lakás, és a család holmija az elfüggönyözött
hátsó részbe került. Az utca felőli front asztalokkal teli, kínai szokás
szerint abrosz nélkül, egyszerű székekkel. A pálinkát nem merem
megkóstolni, de a sör egész jó minőségű. A kínaiak többsége ha nem
otthon, akkor hasonló helyen étkezik. Napközben és délidőben ellepik
az utcákat a mozgó konyhák is. A kerekes kocsikon rajta van a
tűzhely, a tárolóedények és az előkészítő pult.

A karaokebár úgy néz ki, mint egy európai szórakozóhely. Villódzó
fények, emeletes táncparkettek, tömeg, hangos zene, időnként nem
teljesen profi dalbetétekkel.

\section{Mit számít egy emberélet?}

Átlátogatok egy másik szállóba a magyar csoport egy részéhez. A recepción
üzenetet hagyok, és táskámban kotorászva megyek kifelé. Felpillantva látom
a szálló előtt álló portást, és teljes erővel nekimegyek
a tisztára mosott, láthatatlan üvegajtónak. Szemüvegem belefúródik
a szemöldökömbe és felhasítja Végigfolyik arcomon a vér, igencsak
meglepődöm. Újra kotorászom a táskámban, kiveszek egy papír zsebkendőt,
és visszafordulok a portához feltűnő piros pöttyöket hagyva
magam mögött. Kérem, hívjon egy taxit -- szólok oda. A recepciós rám
néz, és értetlenül Válaszol: kint állnak a szálloda előtt, miközben szeme
sem rebben arcom tűzpiros vonalvezetésén. Megfordulok, elmegyek a
portás mellett, aki nagyot köszön, és beszállok egy várakozó taxiba,
mutatom a címet. Fütyörészik a részvétlen disznó, én pedig gyűjtöm a
kezemben, mint egy véredényben, az elhasznált zsebkendőket. Végre
eláll a vérzés, letörölgetem a fejem. Úgy lépek be a szállodaajtón, mint
aki éppen most végzett a bankban a teljes személyzettel, kezem, ruhám
a vérfoltos bizonyíték. Szó nélkül, közönyösen nyújtják a kulcsot.

Másnap reggelre teljesen belilul a jobb szemhéjam, fel a szemöldökömig.
Hasonló, lila sminket teszek a bal felemre is, jó hangsúlyosat,
a hetvenes évek stílusában. Ma aztán jól megtévesztem a Keletet az
európai trendről -- gondolom, és így megyek a sajtófogadásra.

\section{A Nagy Fal}

A Nagy Fal az első építmény, amelyet a repülőgépről meglátok Kínából.
A világossárga kígyó nagy kanyarokkal jelöli ki a kopár hegyek gerincét.
Még egy hónap, és virágba borul a hegyoldal. A pilóta megdönti a
repülőgépet jobbra-balra a szebb látványért. Hullámzunk, a levegőben
követve a fal fordulatait. E hullámvasúton érkezünk meg Peking fölé.

Legközelebb busszal indulunk felé a fővárosból. Az út mentén előtűnő
falrészletek alig fél méterre magasodnak csak ki a földből. A többméteres
falat betemette az évszázadok pora. A több mint hatezer kilométer
hosszan elnyúló erődítmény ma sok helyen megadja magát már a növényvilágnak
is, fák kúsznak fel rá, még a fű is meghódítja csúcsait. Nem így
volt ez az i.~e. VII. században, építése kezdetén. Több százezren dolgoztak
a védőrendszer felépítésén a nomád törzsek, úgymint a japánok és a hunok
betörése ellen. A ma látható hosszát a Ming-dinasztia (1368--1644)
alatt érte el. Jól felszerelt lő- és őrtornyok ékelődnek a lőrésekkel
szabdalt falszakaszokba. Fény- és hangjelzésekkel adtak hírt egymásnak a
tornyok, az ősi telefonvonalak kapcsolótáblái vagy átjátszótornyai. Nem
tudom, másodpercenként hány kilobyte-os sebességgel érkeztek és indultak
útnak a hírek, de a korabeli leírások alapján feltételezhető, hogy
lekörözték a kezdeti számítógépek teljesítményét. Badalingba érve már
majd' mindent láttam, amit a Nagy Falból Pekingnél érdemes megtekinteni.
Az idegenforgalmi bejárat teljesen helyreállított része a kanyargó
téglasoroknak. A rengeteg szuvenírárus, és mindaz, ami egy ilyen helyen
a turisták fogadására épül, mindig ellop a varázslatból valamit. Ezért
szeretem jobban az összetört kancsódarabot a tökéletesen rekonstruált
ép edénynél. Minden tárgy magába szív készítőjéből, majd használójából
valamennyit, és ezt hosszú időn keresztül sugározni képes. Beleivódik
a föld ereje is a hosszú kapcsolat alatt. A régmúlt emlékei feltöltődtek a
környezetük által, és ezzel gazdagabbak, mint a futószalagról éppen csak
lepottyant vadonatúj eszközök. Régi és új kiegészítő párok, hiszen én
sem vágnék szívesen kenyeret kicsorbult élű, rozsdás tőrrel. Badalingba
érdemes elutazni az élményért, hogy ráléphessek és sétálhassak azon az
emberi kéz alkotta építményen, mely a Holdról is látható.

Kezem a szemem fölé emelem, és lekacsintok az itt maradt hun lelkeknek
a túloldalra. Talán ük-ük-a-köbön-szüleim helyett most én tehetem ki
a győzelmi zászlót: bevettem a Nagy Falat! Ám nem tudni, hányféle vér
keveredett bennem az évezredek során, mert egy másik felmenőm azonnal
fejbe is kólint: Mit csinálsz, hisz én építettem, védd a falat ellenükben!

\begin{itshape}
Ezennel őseim közt a falat lebontom, a harci nyílvesszőket kettétöröm,
és az őrtoronyból vigyázom a békét.
\end{itshape}

\section{Lámatemplom -- az örök harmónia palotája}

Megtorpanok az egyik udvarában. A járda szélén ül az idős láma, tőlem
úgy öt-hét méterre. Egyenesen vinne tovább az út, de nem tudok
belépni erőterébe. A láma nyugodt tekintettel, középpontjában pihen
az úton. Ránézek az arcára. Nem szomorú és nem vidám. Alázata a
mindenségbe emeli, harmóniában áll önmagával. Erőit az ég felé viszi,
melyek így megsokszorozódnak, és sérthetetlen lesz tisztasága. Óvatosan
jobbra fordulok, és félkörívben teszek egy kitérőt.

Szavak nélkül tanítottak itt.

\section{A kamasz}

A pekingi kamasz felhúzott orral kérdezi:

-- Honnan jött?

-- Magyarországról.

-- Hol van az az ország? -- sose hallott róla.

-- Közép-Európában.

-- Hányan élnek abban az országban, és milyen népek?

-- 10 millió magyar.

-- 10 millió magyar, hmm\dots{} és saját országuk van, még nem hódította
meg magukat senki\dots{} és milyen nyelven beszélnek?

-- Magyarul.

-- Magyarul? Maguk 10 millióan vannak, és még a saját nyelvüket
beszélik?

De a válaszra nem kíváncsi, és másfelé fordul.

Peking robban, és szilánkjai messzire repülnek.
