\clearpage
\chapter{KAPCSOLATOK}

\section{A közeledésről férfi és nő között}

Akarom, és elegem van -- így csak vesztes lehetsz.

Te férfi, ne ugorj a várfalra! Ha megvárod, míg leengedik a hidat,
lehet, hogy hamarabb bejutsz.

Te nő, pedig vedd észre, ha megérkezett a hős, s menj elé.

Csak középen találkozhattok. Ezt gyakorold mindennap.

\section{Apai nagyapám}

Nagyapám és nagyanyám lépdel az utcán, még fiatalon. Mama
bosszús valamiért, nem szól nagyapámhoz. A fiatal nagyapa előveszi
gyufásdobozát, és egymás után gyújtja meg a gyufaszálakat. Lehajol és
menet közben kutat valami után a földön. Jobbra, balra félköröket ír le
a talaj fölött a kis lánggal. Nagyanyám nem bírja tovább, s megkérdezi:
mit keresel ott a földön? Mire nagyapa: ezt az első szót kerestem.

\section{Az irigyekről}

Tartózkodj az alattomosaktól és irigyektől, mert álnokságuk úgy
fertőz, mint a pestis. Szeresd őket mint embereket, és szánd is őket,
ez a legtöbb, amit tehetsz, de kerüld társaságukat, mert menthetetlenek.
Olyan sebeket kaptak korábban, melyeket te nem gyógyíthatsz,
csak feltépheted őket, és a genny elönti környezetedet. Fenevadakként
zsarolják a szeretetet és az elismerést, ha úgy érzik, keveset kaptak. És
mindig úgy érzik. Egészen kellesz nekik, mert különben nincs kiből
táplálkozniuk.

Lépj arrébb, amíg nem késő.

\section{Alteregó}

\begin{Verse}
A szavak hamisak.						\\
A tiszta hang füllel hallhatatlan.				\\
Befelé fordulok, hogy igazán halljalak.				\\
Amikor magamban megtalállak, akkor majd értelek.		\\
Csendben vagyok, mert még kereslek.				\\
Hallgatásom eltéphetetlen kötelék, mely minket összeköt.	\\
Ha majd megszólalok, szabad leszek.				\\
A szavak némán letisztulnak, s kimondatlanul beleolvadnak a csöndbe. \\
Utunk így elágazik, hogy sorban feloldjuk a többi csomót is sorsunk fonalán. \\
Hallgass most te is, és figyelj befelé, hogy lazuljon a kötés.
\end{Verse}
