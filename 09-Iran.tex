\clearpage
\chapter{IRÁN -- AZ ELFELEDETT 1001 CSILLAGÉJ MESÉINEK FÖLDJE}

Két évvel az ezredforduló után, egy augusztusi napon, egy vállalkozó
kedvű, busznyi csapat Perzsia felé vette útját azon kevés, évente csupán
pár ezer iráni látogató egy kis csoportjaként, akik az utóbbi években
ennek az ismeretlen, elszigetelt, és sok esetben tévesen megítélt
országnak a népét a nyugati világ különböző pontjairól felkeresik.

Indulás előtt szinte mindenki döbbenten nézett vissza rám a kérdésre
kapott válasz miatt, hogy Iránba megyek nyaralni. -- Micsoda?
De hát ott lőnek, elrabolnak, megerőszakolnak\dots{} ki tudja,
mi történik veled? Hogy jutott ez eszedbe?

Miért döntöttünk Perzsia mellett?

A kalandvágyam persze, mely már eddig is számos távoli helyre vitt,
valamint egy lezárult szerelem régmúltba nyúlt gyökerei után indultam.
Többen a keleti kultúra szerelmeseként, a nagy történelmi helyszínekért
jöttek. A két fotósnak, Gábornak és Gyurinak izgalmas terepet
ígért az útiterv, az ELTE arab, török, perzsa szakos hallgatóinak
pedig nyári szakmai gyakorlatként is beillett a kirándulás. Hasszán
bácsi a rokonokhoz ment haza látogatóba, így ő távolsági járatként és
a társaság kedvéért választotta ezt az utazási módot. Meglepetésemre
volt alma materem igazgatóhelyettese, Péter is a csapat tagja lett.

\dots és miért is ne, olyan keveset tudunk erről a nemcsak földrajzi
értelemben távoli országról, amit tudunk, az is inkább ókori történelmére
és a jelenkor zűrzavaros politikai híradásaira szorítkozik.

Megadta a helyhez illő karavánutazás hangulatát, hogy útközben
az utaslétszám folyamatosan változott, mert vagy elbúcsúztunk egy
pártól, amely önállóan folytatta tovább az utat Iránon belül, esetleg
Pakisztán felé kanyarodva, vagy éppen csatlakoztak hozzánk utasok
egy-egy városban.

Induljunk el erre a huszonhárom napos, meglepetésekkel teli keleti
barangolásra.

Felajánlok egy varázsszőnyeget, az röpítse az olvasót, amíg az
\textbf{Ararát} hófödte csúcsa ki nem rajzolódik a távolban.
Ekkorra mi már hat
napja úton vagyunk, elhagytuk a Kelet kapuját, Isztambult, a kappadókiai
Tündérlábakat a barlanglakásokkal és a Nemrut-hegyi napfelkeltében
előtűnő istenszobrokat. A Van-tavi hajózás után közeledik
a török-iráni határ, így megállunk átöltözni. Mi nők maradunk a buszon,
és előkerülnek a házi készítésű csadorok, mert Iránban még a
turistáknak is kötelező a helyi viselet, de legalábbis a hosszú, bokát
takaró szoknya vagy nadrág, a hátsó domborulatokat szintén fedő
hosszú ing hosszú ujjal. Leszállunk, a fiúkban erősödik a férfivirtus,
füttyögnek és pontoznak.

A határátkelés elég kaotikus, építkezéseken keresztül történik,
kamionokat látni, és egy-két török vagy iráni buszt. A török oldalon
várakozunk, mellettünk egy iráni busz áll, a nők integetnek a fejükre
mutatva, hogy vegyem fel a kendőt. Az épületben éppen egy cigarettacsempészt
buktatnak le, nagy a felfordulás. Elek és Miki még jóval
a határ előtt nyomatékkal kérdezi, hogy nincs-e valakinél alkohol,
mert az alkoholfogyasztás tilos Iránban, a külföldieknek is börtön jár
érte. Végre leszállhatunk és bemegyünk az épületbe, onnan egy nagy
nyúlketrecre emlékeztető helyiségbe, ahol ötven fok van és tömeg. Egy
órát állunk ebben a sötét, tolongó emberekkel teli teremben, ahol egy
rácson át kell kinyújtani a szabadba a vámosnak az útlevelet. Ha megjelenik
a ketrecajtóban, a tömeg nekiesik a rácsnak. Ömlik rólam a víz,
és ebben a pillanatban elbizonytalanodom, hogy tényleg akarom-e ezt
a kalandot, őrület ez a hely. Majd egyszer csak kint találjuk magunkat
a friss levegőn, és átsétálunk az iráni oldalra, ahol leesik az állam.
Légkondicionált, székekkel berendezett, modern váróterembe lépünk,
ahol mosolyog a tisztviselő, és türelmet kérve az egész csoport útlevelét
elviszi, nincs tolongás. Hogy igazságos legyek: amikor két hét múlva
visszafelé jövünk, a nyúlketrecnek már hűlt helye, a török oldalon is
épül a határátkelő új épülete.

Iránban vagyunk, és jólesik a határ melletti első vacsora, az út során nagyon
megszeretett tuhr, vagyis hideg joghurt, és a finom vajas rizs, melyet
csirkekebabbal eszem, meg valami rózsavízzel leöntött paradicsommal
és uborkasalátával. Majd meg sem állunk \textbf{Tebriz}ig, ahol az első
éjszakát
töltjük. A szállodában rácsodálkozom a telefonkapcsoló szekrényre, mely
láthatóan nem automata, de ez az észrevétel már szakmai ártalom.

Másnap már korán reggeli után nézünk, és pár sarokkal arrébb találunk
egy kenyérsütödét, ahol a kovásztalan iráni kenyeret a szemünk
láttára sütik. A meglengetett tésztát a sütőlapátra teszik és bedobják a
kemencébe. Pár perc, és máris vihető és ehető a forró, kinézetre pászkára
emlékeztető, de annál vastagabb kenyérlepény.

Kora délelőtt indulunk a fél évezredes Kék mecsethez. Gyönyörűek
a kék csempedíszítések. Kúfi írást láthatunk keretben, a perzsa szöveg
pedig végigfut egy szegélyben. A mecsetben fekete csadoros nők,
a férfiak is hosszúnadrágban, hosszú ujjú ingben. Odalép hozzám egy
fiatal lány, Leila. Kíváncsian kérdezi, hogy honnan jöttünk, milyen
Magyarország, hogyan élünk. Egész úton végigkísérnek bennünket az
iráni fiatalok, akik nem tudnak külföldre utazni, szinte semmi, hiteles
forrásból származó információjuk nincsen arról, hogy mi történik a
világban az Irán-barát államokon kívül. Így nagyon örülnek, ha egy-egy
ritkaságszámba menő turistával beszélgetni tudnak, gyakorolják
az angolt, és tájékozódhatnak is. Leila tizennyolc éves, most készül az
egyetemre, és soha nem akar férjhez menni, a viszonylagos szabadság
érdekében inkább az önállóságot gondolja választani. Milyen is a nők
helyzete?

Iránban a nők választhatnak és választhatóak, vagyis női képviselők
is vannak, autót vezethetnek, egyetemre járhatnak, de külön padsorokban
ülnek, például a buszon a jármű eleje a férfiszakasz, hátul
szállnak fel a nők. Tilos a kézfogás és az utcán ismerkedés. Randevú
hivatalosan kizárólag családtag kíséretében képzelhető el. Hoztam
magammal egy képeslapköteget Budapestről és a magyar tájakról, átadok
egyet Leilának, aki csak vonakodva veszi el, mire észreveszem,
hogy a matyó népviseletbe öltözött nők szoknyája bizony csak térdig
ér. Hát igen, a szabadság utáni vágyat erősen kontroll alatt tartja az
elvárásoknak való megfelelés, a tradíciók kényszere, és úgy látom, hogy
a félelem is benne van.

Tebriz az azeri kultúra fellegvára, másfélmillió lakosával Teherán
után a második legnagyobb városa a hetvenmilliós országnak. Tebriz
Hasszán bácsi szülővárosa, kilencen voltak testvérek, az édesapja
százöt éves. Meséli, gyermekkorában volt egy medvéjük, amellyel
néha még birkóztak is, sőt a medve egyedül neki megengedte, hogy
leteperje. A buszon egyszer csak felém fordulva ezt mondja: ,,Tudod,
ott a tónál háttal ültem nektek, de a rózsának nincs eleje és háta, s én
magam vagyok a rózsa.''

Hasszán Bidzsari bácsi számtalan története közül álljon itt egy.
,,Volt egyszer egy ember, aki eljutott Örményországba, s ki tudja,
miért, elvetődött a falu temetőjébe. Nézi a kopjafákat, egymás után
olvassa a sírfeliratokat, és egyre inkább megrémül az olvasottakon.
Visszaballag a faluba, és félve megszólít egy odavalósit. -- Meg ne
haragudjon, de most jövök a temetőből, és ott csupa gyerek van elhantolva,
hogy lehet ez, élt egy vagy három évet, a legidősebb is tíz év alatt hunyt
el? -- Ránéz az emberre a falubeli, s megkérdi tőle: -- Hát maga hány
évet élt eddig, meg tudja-e mondani? A születésétől ez idáig mennyi
volt az Élet? Mert a szerencsésnek évek.''

Elhagyjuk Tebrizt, és színes kőhegyeken keresztül, birkanyájak
mellett érkezünk \textbf{Ardabil}ba, mely az egykori Selyemút egyik állomása
volt, és a síita muszlimok szent helyéhez, a Sejh Száfi ud-Din-mauzóleumhoz.
A szent életű férfi szerzetes volt a XIII. században. A mauzóleumban
volt a világ legnagyobb, a XII. században készült szőnyege,
amelyet azonban eladtak egy gazdag perzsaszőnyeg-kereskedőnek.
Most látjuk, ahogy fényképek alapján újraszövik. A mauzóleumban
található az évszázadokig, 1498 és 1730 között fennálló Szafavid-dinasztia
első uralkodójának a sírja is. A múzeumot többször is kirabolták,
a szarkofágokat azonban nem tudták elvinni, mert nem fértek
ki az ajtón. A kék, sárga, zöld mozaikkockák millióival kirakott
homlokzatokat csodáljuk. Az iszlám a VII. században kialakult arábiai
vallás, mely Iránban 1000 körül jelent meg, öt alaptétele: egy az Isten
és Mohamed az ő prófétája, a mekkai zarándoklat, az alamizsnaosztás,
a böjt és a napi öt ima. A síita iszlám szerint a vallási vezetőnek Ali
kalifa leszármazottjának kell lennie, míg a szunniták azt vallják, hogy
nem kell vér szerinti örököst keresni, a rátermettség számít.

Ardabilt elhagyva a legcsapadékosabb Gilán tartományon utazunk
keresztül a \textbf{Kaszpi-tenger} felé. A Kaszpi-tengerrel kapcsolatban
ellentmondásosak a vélemények. Egyesek szerint a tőle 40 kilométerre
fekvő Aral-tóhoz hasonlóan holt tengerré válik, míg mások véleménye
alapján a tektonikus mozgások révén a Kaszpi-tenger növekszik. Ez
a vidék Irán éléskamrája, érdekességként érdemes megemlíteni, hogy
Iránból terjedt el a görögdinnye, az őszibarack, a rózsa és a tulipán.

Estére érünk \textbf{Rast}ba, egy kis kóválygás után megtaláljuk a
szállodát. Miki pedig elkezd könyveket vásárolni. Miki, azaz Sárközy
Miklós az ELTE perzsa tanszékének tanára, akinek az út során a fantasztikus
idegenvezetést köszönhetjük, s akinek perzsa nyelvismerete nélkül
bizony zökkenőkkel teli utunk lett volna.

Rastban trópusi az időjárás, magas a páratartalom, így a meleget
még izzasztóbbnak érezzük. A csadorom alatt már semmit nem viselek,
de még így is nehezen bírható ki a hőség. Együttérzőn gondolok
azokra az iráni nőkre, akik a csador alatt még hosszú ujjú inget
is hordanak, és pamutharisnyát! -- jó esetben nadrág nélkül. De látok
csador alól kibukkanó farmert is. Két ruhadarabbal készültem, egy kis
fehér virágmintás feketével és egy királykékkel. Még szerencse, hogy
nem vettem teljesen komolyan az utazási iroda eligazítását a színeket
illetően, mármint hogy ó, bármilyen színű lehet, még piros is. Hát az
iráni csador az fekete, néha apró mintás, nem feltűnően szürke vagy
barna. A királykék ruhámmal így meglehet divatot teremtek, mert
amelyik iráni családnál fényképezőgép volt, azoknak mind modellt
kellett hogy álljak. A nők elismerő csettintgetésekkel fogdosták meg
az anyagot, és dicsérték a színét. De volt egy kis szépséghibája a kék
csadoromnak, hogy elszabták az egyik oldalát és kivillant a bokám,
néha még a lábszáramból is mutattam valamennyit, amiért a későbbiekben
még egypárszor megkaptam a magamét. Kriszta nem csinált
különösebb gondot az öltözködésből, hosszú szárú nadrágokat hozott,
a papája ingeit csomagolta be, és gyerekkori kendőit. Ezzel a kis babakendővel,
alatta rövid szőke hajával rendkívüli elismeréseket szerzett a
perzsa férfiak körében, semmiféle incidens nem érte, holott a nosztalgiamintás
gyerekkori kendőcske inkább zsebkendőként hatott a fején,
az eredeti célt, hogy takarjuk a hajunkat, csak jóindulattal töltötte be.

De térjünk vissza a Kaszpi-tengerhez, ahol már nagyon vártuk a
fürdőzést. Miki előre jelezte, hogy külön van a női és a férfistrand. Így
is volt, a strandhoz érve jobbra mennek a nők, balra a férfiak, középen
kizárólag utcai ruhában lehet tartózkodni, itt vannak a büfék. A női
strandot a két oldalán a tengerbe is benyúló vászonnal kerítették el.
A férfioldalnak csak egy vászonfala van, mögötte majd' egy kilométeres
partszakaszon fürödhetnek. A nők női felügyelők mellett egy
ötvenméteres partszakaszon botorkálhatnak be a hínáros vízbe, igaz,
akár bikiniben. Na nem, köszönjük, inkább felmegyünk a semleges területi
teázóba, és talpig felöltözve hűsölünk az árnyékban. A cölöpökre
épített étteremből belátni a férfistrandra, de elég messze van, így csak
kis alakokat látunk, nem ismerjük fel igazán a mieinket, pedig egyéb
szórakozás híján azon igyekszünk. Majd eldöntjük, hogy elsétálunk az
allé végén található palotához, de szörnyű meleg van, és a Miki által
említett tizenöt perces séta valószínűleg futólépésben értendő. Péntek
lévén, ami az iráni vasárnap, a családok piknikeznek az allé mellett,
a füvön. A piknik nagyon népszerű közösségi időtöltés, a családok,
baráti társaságok magukkal hozzák a bográcsot, tüzet raknak, levest
főznek, kebabot sütnek, esznek, beszélgetnek, énekelnek. Számos család
női tagja odaint, hogy csatlakozzunk. A strand dél és tizenöt óra
között zárva van, a fiúkat is kiterelik a férfiszakasz őrei, mert nekik is
jár a szieszta. A programot az ebéd menti meg, a roston sült Kaszpi-tengeri
hal, a sáfrányos vajas rizs, a párolt és savanyított zöldségek,
a diós olívabogyó és az elmaradhatatlan joghurt, majd dinnye.

Továbbutazva még hosszan kéklő, homokos partokat látunk, de
végül belenyugszunk, hogy nem mi fogjuk megnyitni az első szabadstrandot
Iránban. Majd délnek fordulunk, és estére megérkezünk Teheránba.

\textbf{Teherán} jelentése forró hely. Irán fővárosa egykor karavánszeráj
volt a Selyemút mentén. Ma már tízmillió lakosa van, a peremvárosokkal
együtt harmincmillió, és ötmillió autó pöfékel útjain. Nem csoda,
hisz a benzin ára literenként akkor 4 forint 20 fillérbe került. Irán a
kőolajtartalékokat tekintve a negyedik, a földgázkészletek tekintetében
a második helyen áll a világ országainak sorában. Az autóáradatnak
megfelelően a város fölött vastag szmogfelhő látható, mindenütt
dugók. Oázis a kőrengetegben a Golesztán-palota, magyarul Rózsa-palota,
és kertje. A sah európai uralkodóktól kapott dísztárgyait nézzük
meg, perzsa festők műveit, és egy 1800-as éveket bemutató fotókiállítást
az alagsori részben, gyönyörű csobogóval. Salamon király
fehér márvány- és alabástromtrónja, sahokat ábrázoló festmények,
a tükrös íván egyik szobájában valamelyik sah, valamint hatvan fiú-
és harminc lánygyermeke közül néhánynak a portréja látható a séta
közben. A palota egy-egy termének pompája a versailles-i kastélyéval
vetekszik, az arany bősége, a tükrök, a faragások, a gazdag díszítés
ámulatba ejtő. Salamon trónján ülve látom, hogy az előttem lévő szökőkút
kávájára kilép egy fehér hattyú. A Tejút Hattyú csillagképéhez
hasonlóan a nyakát magasra nyújtja, szárnyait kiterjeszti és gyönyörű
dalt hallat. Az útról csadorban két fekete hattyú is figyeli énekét.

Később végigjárjuk a Nemzeti Múzeum termeit, és a múzeumlátogatásokat
méltó módon a Makarzi Bank tulajdonát képező ékszermúzeummal
zárjuk. Mindent leadunk a ruhatárban, majd az alagsorban
egy vaskapun és egy páncélozott ajtón át jutunk be az egykori sahok
kincseskamrájába. A huszonhétezer drágakővel ékesített nadír pávatrón
mesterien tervezett, kecses darab. Külön-külön többpolcos vitrineket
állítottak ki csak a briliáns, vagy csak a rubin, a zafír vagy a
smaragd drágaköveknek, az ezekből készült ékszereknek, kardoknak,
koronáknak, diadémoknak, fülbevalóknak, brossoknak, gyűrűknek,
evőeszközöknek és tálaknak. Függönybojtok sora csüng igazgyöngyből,
megcsodálok egy földgömböt, melyen az óceánok és földrészek
ötvenkétezer drágakőből vannak kirakva, és ékkövek ezrei ragyognak
csak úgy a polcokra szórva. A 182 karátos ,,Nagy Asztal'' gyémántra
pillantva befejezem a nézelődést, már nem tudok többet befogadni.

Visszataxizunk a szállóba, és irány a bazár, mely sajnos nagy csalódás
az isztambulihoz képest, egyedül a préselt sárgadinnyelével
vagyunk elégedettek. Suhancok lábtaposással figyelmeztetnek, hogy
kilátszik a bokám, a bejáratnál még ennél durvábbat is kapok. Zoli
pizsamanadrágot vásárol fillérekért, melyet magán is tart, itthonra alternatív
szerelésnek szánja. Itt viszont az egész bazár rajta röhög, így egy
idő után lecseréli az út közepén, hárman körülálljuk. Ő bezzeg nem
kap ezért büntetést, férfi! Kriszta mellé újabb hódoló lép, egy fiatal
értelmiségi, aki szintén ,,nyugati'' útról álmodik, Párizsba készül utazni,
még dátum nélkül. Kedves, udvarias és tartózkodó, mint minden
fiatal perzsa, akivel megismerkedtünk. Nagyon előzékenyek, érdeklődőek,
és szívesen osztják meg érzéseiket az életükről, a rendszerről.
A hozzánk csapódott hétköznapi emberekkel történt találkozás igazi
élmény volt. Ettől a fiútól is, a félórás ismeretség után két évvel, még
mindig érkeznek apró ajándékok postán.

Az irániak a politikai szövetségeseiket kivéve (Szíria, Libanon,
Közép-Ázsia országai vagy Törökország) nem utazhatnak külföldre.
A politikai helyzet miatt 1979-ben négymillióan emigráltak az
USA, Németország, Ausztria, Anglia irányába. Magyarországon körülbelül
négyezren élnek. A sah fia Washington külvárosában telepedett
le, a látottak alapján nem sok esélye van a visszatérésre.

Este egy szűk kis csapat úgy dönt, hogy taxival felmegyünk Észak-Teheránba.
A taxissal nincsenek nyelvi korlátok, Zoli élénken magyarázza
magyarul, erősen gesztikulálva (mintha úgy jobban érthető
lenne): Nekem otthon egy kisfiú, és neked? Majd a könnyebb végén
fogjuk meg az ismerkedést, és a zene nemzetközi nyelvén rázendítünk
a \textit{Kék a szeme} nótára, amit taxisunk egy perzsa dallal viszonoz.
Kisebb fulladásos rohamok és dugókban töltött kényszerű pihenők, üdítőkért
való ki-kiugrálások után, közel másfél óra múlva megérkezünk az elit
Darban negyedbe, ahol érezhetően jobbak az oxigénviszonyok. Libegőzni
szeretnénk, de az csak péntekenként működik. Nekiindulunk
a hegynek, úgy érezzük, a Szalajka-völgybe érkeztünk, lezúgó patak
mellett kaptatunk felfelé, jólesik a hűs levegő. A patak mentén nyitott
éttermek hosszú sora, ahol az asztalok a sziklába vájt teraszokon állnak
lépcsőzetesen, a kis utak közöttük lampionokkal vannak megvilágítva,
gyertyák égnek több helyütt. Egy idő után betérünk az egyik
vendéglőbe és lefekszünk a legfelső terasz szőnyegágyára, vacsorát rendelünk,
teázunk és az élet nagy kérdéseiről vitázunk. Éjfél után indulunk
vissza, az út most huszonöt perc a szállóig.

Másnap újabb érdekes részleteket tudunk meg az iráni életről. Teheránban
a kukásokkal Beethoven- és Mozart-zenét hallgattatnak,
hogy gyorsabban dolgozzanak. Tudományosan bizonyított, hogy e két
muzsikuszseni művei serkentőleg hatnak a munkavégzésre.

Délelőtt a Szabadság-torony modern épületét járjuk körbe, melyet
a 2500 éves iráni államiság emlékére emeltetett a sah 1971-ben.
A sahok Nagy Kürosz halálától számították az iráni államiság kezdetét.
A 40 méter magas épület tetején kilátó és étterem működik.
Alakja az ívánt, vagyis fogadócsarnokot próbálja stilizáltan utánozni.
A Szabadság-torony elnevezést már Khomeini ajatollah alatt
kapta 1979-ben.

Khomeini az Iszfahánhoz közeli Khomeini városban született, és
Komban, Irán vallási központjában tanult teológiát. 1940-ben tiltakozott
az amerikai és a szovjet megszállás ellen, ekkor figyeltek fel
rá, majd emiatt száműzték 1960-ban. A forradalom győzelme után,
1979--89 között ő irányította Iránt, a teljes elzárkózás politikáját
folytatta. Mohamed államát akarta megteremteni a monarchiával szemben.
Katonai államot hozott létre, ahol ő lett a forradalom vezetője.
Irán 1979 óta köztársaság, kormánya és parlamentje van négyévenkénti
választásokkal, de mindezt felülírja, hogy mindenek felett áll a
forradalom vezetője, aki egyben a legfőbb vallási vezető és a hadsereg
főparancsnoka is. Az Őrök Tanácsa pedig felülbírálhat minden törvényt.
A mollahok tanácsának tagjai közül kerül ki a következő vezető,
akit a hatalmon lévő választ ki utódjául.

Khomeini maga jelölte ki sírját Teherántól nem messze. Amikor
temették, a tömeghisztéria miatt kiesett a koporsóból, így végül
helikopterrel szállították a rézkupolás mecsetbe. Ottjártunkkor
Mohamed lánya, Fatima halálának évfordulóját ünnepelték, az egykori
május elsejei felvonulásra emlékeztetve bennünket. Fekete csadoros
iskolás lányok menetelnek csoportokban a várost és az iskolát megnevező
táblákkal. A mecsetbe férfiak és nők külön bejáraton mehetünk
be a cipők levétele után, a detektoros kapun át. Koszorúzás folyik
éppen, tévéfelvétellel.

\textbf{Kom} városa a legrégebbi muszlim település, a legendák szerint itt
hunyt el Fatima. Főleg a meddőségben szenvedő nők járnak ide imádkozni.
Kom a síita teológiaoktatás központja zárt főiskolával, ahol magas
szintű lexikális tudást követelnek és magolva tanulnak. 1978-ban
itt kezdődtek azok a tüntetések, melyek végül a sahot megbuktatták.
Nem síita muszlimnak tilos a bemenet az itteni mecsetbe. Vámbéri
Ármin volt az egyetlen magyar, aki muszlimnak adva ki magát, bejutott
a templomba.

A komit kivéve azonban szinte minden iráni mecset szabadon látogatható,
ami nem mondható el a marokkói vagy egyiptomi mecsetek
legtöbbjéről.

Híres a komi süti, veszek is ajándékba jó pár dobozt a pisztáciás,
mézes, diós és karamellás édességből. Ezzel nem vagyok egyedül, és
később kavarodás is kerekedik abból, hogy a busz csomagterébe pakolt
dobozok és nejlonzacskók kihez is tartoznak.

Még aznap délután \textbf{Kásán}ba érünk, és egy angol étteremben
ebédelünk. Az étterem attól angol, hogy a perzsa tulajdonos felesége angol
nő, ami az étterem berendezésében, hangulatában, a terítésben is visszaköszön,
de az étlap perzsa választékot kínál. Újabb perzsa édesség
kerül fel a toplistámra, a diós gránátalmajoghurt.

Gazdag XVIII. századi szőnyegkereskedők házait nézzük meg,
a színes üvegablakokkal, a porcelánmintákra emlékeztetően gazdag
díszítéssel festett falakkal és mennyezetekkel, átriumszerű kertekkel
ezek a házak kis palotáknak illenek be.

Közben a buszunkat szervizbe viszik, mert kilyukadt a gáztartálya,
így sétálunk még a jelenkori lakóházak között is, a nők arcukat
takarva szaladnak be a házakba, a küszöbről még kíváncsian hátratekintenek,
aztán becsapják előttünk az ajtót. Csak késő este tudunk
továbbindulni Iszfahánba. Útközben felforr a hűtővíz, szíjcsere kétszer,
éjfélre érkezünk meg.

\textbf{Iszfahán} a legvarázslatosabb megálló az út során, meseszerű
helyszínekkel és jelenetekkel. Az 500x160 méteres Imám tér a világ egyik
legnagyobb tervezett főtere 1612-ből, középen parkosított kisebb terekkel,
szökőkutakkal. Az egész hétszer akkora, mint a velencei Szent
Márk tér. A teret mecsetek és bazársor veszik körbe. Abbász sah egy
régi lovaspólópályából építtette ünnepélyek felvonulási terének. Az
Imám-mecset is Abbász sah idejében épült, híres a kék-sárga-fehér-rózsaszín,
néhol 30 méter magas geometrikus és virágmotívumokkal,
kalligráfiával díszített csempe- és mozaikmintáiról. Az egyik íván
tökéletes visszhangzású, az emberi fül tizenkét echót képes érzékelni,
itt negyvenkilencet mértek. Sorban kipróbáljuk a visszhangot, nagyon
furcsa a saját hangomat többszörös ismétlésben a falakról visszahallani.
Egy középkorú tanítónő lép mellém, és kérdezi, honnan és miért
jöttünk. Válaszom nem győzi meg, és kissé bizalmatlanul tovább kérdezget,
hogy érezzük magunkat, mit gondolunk az itteni emberekről.
Amikor azt válaszolom, hogy az itteniek is ugyanolyan emberek, mint
bárhol Európában, hirtelen kissé meghökkenve rám néz, majd gyorsan
elköszön. Ki tudja, ki volt és mit gondolt magában? Fogy a képeslapkészletem,
már dedikálnom is kell magyar kísérőszöveggel, Mikitől
ráadásként kérik a perzsa fordítást is mellé. Viszonzásként apró
papírpénzt adnak szerencsehozónak, melyet illik elfogadni. A kislányoknak
úgy tízéves korig nem kell csadort hordaniuk, színes európai
ruhákban járnak, de sokszor kis kendőt vagy kalapot viselnek.

\illustration{page-095}{Az Imám-tér a nagy szökőkúttal Iszfahánban}

Az út számomra első számú helyszínén, a Lotfollah sejk-mecsetben
már ültömben osztom szét az utolsó képeslapokat egy ilyen gyerekseregnek.
Ülök, mert ekkor már nem tudok tovább állni ebben a misztikus,
energikus és harmóniát árasztó kupolahelyiségben. Yves Gandon
\textit{Az éden nyomában} című művében elragadtatással ír e helyről:
,,A pici
mozaikokkal borított falak, az arab írás fehér vonalai és cikornyái vég
nélkül ölelkeznek a csempék kék alapján, és a színek, a minta bonyolult
összhangjában szinte megmozdul a sztalaktit boltozat, és forogni
kezd a bejárati árkád egy tömbben kiégetett, csavart kerámiaoszlopa.''
A kupola krém- és rózsaszín között váltakozik, és az ablakokon beszűrődő
fény játéka folyamatosan más és más megvilágításba helyezi a
mozaikok ábrázolásait. Szédülve lépek ki újra a térre.

Az iszlám legrégebbi mecsetében az időközben szakállasodó Miki
már teljesen beleolvad az imára érkező férfiak sokaságába.

Ezután szétválnak útjaink, és ki-ki önállóan fedezheti fel az egykori
perzsa fővárost. Páran visszatérünk az Imám térre, és Zoli a rekkenő
hőségben megirigyli a suhancok fürdőzését a tér hatalmas medencéjében.
Gatyára vetkőzik és berohan a szökőkútba. Én csak bokáig tervezem,
leveszem a szandált, már ez is merészség. Zoli nem szól, hogy
a medence alja algás és befelé lejt, így elég gyorsan elérek a közepéig,
és ott belezuhanok. A nevetéstől alig bírok feltápászkodni, a csadorom
szétterül a vízfelszínen. Orsit a medence szélén rázza a nevetés, de
azonnal a gépért nyúl és megörökíti a balesetet. Zoli kisegít a szökőkútból,
közös erővel kicsavarjuk rajtam a csadort, a negyven fokban
gyorsan megszárad. Az irániak nem nevetnek, inkább feltűnés nélkül
elsétálnak a közelből. Egyedül Ali lép oda hozzánk, és fejcsóválva int,
jó, hogy nincs a közelben a vallásrendőrség, Zoli is jól teszi, ha felveszi
a nadrágját. Felvisz bennünket a tér kereskedői által látogatott egyik
étterembe, az árkádsor emeletére. Együtt ebédelünk, kiválóan beszél
angolul. Ebéd után meghív a boltjába egy teára, de ez nem vevőcsalogató
meghívás.

Elköszönünk, majd nemsokára kontrasztként meglátom a második
,,Down with USA'' graffitit az egyik házfalon. Le akarom kapni,
de szinte a föld alól előpattan egy gárdista, és mutogat, hogy nem
fotózhatok. Mérges leszek, megvárom, amíg a társával elmegy, és akkor
mégiscsak fényképezek.

Visszatérünk a hatalmas bazársorra, ahol az ősi kövek megadják a
bazár alaphangulatát. A legtöbb bolthoz műhely is tartozik, ami vagy
azt jelenti, hogy a bolt előtt munkálják meg a rezet, de lehet, hogy
közvetlenül a pult mögött dolgozik a tulajdonos. Egymás után döntöm
le a banánturmixokat és eszem a fagyit, szerencsére késleltetett hatás
nélkül. Iránban iható a csapvíz, ellentétben a törökországi területekkel,
és feltűnő a tisztaság. A csoportban senkinek nem lett problémája,
bármit fogyasztottunk is, pedig még alkohollal sem tudtunk
fertőtleníteni. Miniatúrákat szeretnék vásárolni, így ezeket a boltokat
keresem. Tevecsontra festik a különböző jeleneteket színesben vagy
elegáns fekete, fehér és arany színben, bársonyra ragasztják és apró
mozaikos, geometriai mintás keretbe teszik. Szinte lehetetlen választani,
annyira szépek. Minden boltnak megvan a maga művésze, így
nagyon sokféle stílus közül válogathatunk. Kiválasztok négyet, de este
elcsábulok újabb kettőre. Alkudni lehet, de én sajnos nem tudok, ha
valami megtetszik, az nagyon meglátszik rajtam, így hátrébb állok, és
a fiúk eredményesen besegítenek.

Hazatérünk pihenni egy kicsit, bekapcsoljuk a tévét. Négy állami
csatorna működik és egy tartományi, mindegyik szigorú ellenőrzés
alatt áll. Az egyik csatorna kizárólag sporteseményeket sugároz, focit
vagy birkózást, ez a két sportág a legnépszerűbb. Amikor beszélgetés
közben említettük, hogy honnan jöttünk, az első reakció legtöbbször
az volt, aha, Puskás! A másik csatornán folyamatosan forradalmi klipek
mennek, az énekes mellé néha beúszik egy Khomeini-kép. A harmadik
csatorna a forradalmi emlékműsorok és vetélkedők színtere, pl.
olyan feladatokkal, hogy ki tud egy bizonyos idő alatt a legtöbb forradalmi
jelszót spray-vel felírni egy falra. Majd a negyedik csatornán főleg
iráni vagy japán szappanoperákat vetítenek, melyek terjedelmüket
tekintve hasonlítanak a dél-amerikai változatokhoz, de az érzelmeket
az egykori némafilmek visszafogottságával ábrázolják. A felszínen
mutatott prűdség mögött azonban nyilván ugyanúgy parázslanak az
ösztönök, szomorú példa erre az újságban olvasott hír: Teheránban elfogták
az iráni futballválogatott hét tagját, akik egy titkos bordélyházban
múlatták az időt. Büntetésük fejenként ötven korbácsütés, a nőket
azonban kivégzik! Ezzel szemben furcsa módon létezik a két-három
hónapos próbaházasság intézménye, illetve megengedett a válás, melyet
a nő is kezdeményezhet. A szigorú illemszabályokra válasz a korai
házasság, melyet általában a húszas éveik elején kötnek. Külföldi férfi
azonban nem vehet el helybeli muszlim nőt, ha a választottja mégis
hozzámegy, elveszíti iráni állampolgárságát.

Naplemente előtt indulunk Iszfahán legrégebbi hídjához, a Hadzsu
hídhoz. A Fatima-ünnep miatt a hídi teázó sajnos zárva, de egyre többen
gyűlnek össze a hídon, beszélgetnek és piknikeznek. Mi is leülünk
a lépcsőkre, mire több fiatal hozzánk csapódik beszélgetni, de itt a
vallásrendőrség is, odajönnek szólni az irániaknak, hogy ne zavarják a
külföldieket. Nem törődünk velük, ők pedig tisztes távolságból figyelnek.
Visszataxizunk az Imám térre, mely kivilágítva talán még lenyűgözőbb.
A füvön családok százai pihennek és piknikeznek. A sok-sok
fekete ruhás nő, a mecsetek kupolái, a kivilágított bazársor árkádjai,
a szökőkutak moraja és a konflisok kerekeinek zaja egy megelevenedett
mese az ezeregyből. Bérelünk egy konflist és mi is körbegurulunk,
majd az árkádsor alatt vacsorázunk. Desszertként iráni fagyit eszünk,
mely rizs, cukor és vanília összegyúrva, cérnametéltre vágva, majd
fagyasztás után citromlével leöntve.

A következő nap a Lengő minareteknél indul a délelőtt. Ha az egyik
minaretben állunk és meglengetjük a testünkkel, akkor a másik is elkezd
mozogni. A mutatványra várva Miklóst fiatal iráni lányok ostromolják,
s ő védekezésül a Krisztával kötendő házassági esélyeit ecseteli.

Iszfahán örmény negyedét keressük fel ezután. Ma 50-60 ezer,
főleg kézműves örmény él itt. Érdekesség az aranyozott hajszálra írt
örmény szöveg, melyet mikroszkóppal olvashatunk. Itt láthatjuk a világ
legkisebb könyvét, 0,7 gramm, és 14 oldalon hét nyelven olvasható
egy ima, kb. ekkora: \Square. Lehetséges, hogy napjaink nanotechnológiája
révén már elveszítette elsőségét. Mindenesetre e miniatűr könyvbe
emberi kéz jegyezte be a fohászt.

\fullpageillustration{page-099}{Iszfaháni utcarészlet}

Továbbindulunk az 1000 éves halott város, \textbf{Náin} felé. Kimagaslik
az agyagerőd és a sok helyütt látott szellőzőtornyok. Ezek a résekkel
ellátott tornyok tökéletes légkondicionálást biztosítottak már sok száz
évvel ezelőtt is a lakóházakban. Sikátorokban kanyargunk, valóban
kihalt a város, néha látunk egy-egy ajtórésbe besuhanó alakot. Nem
ajánlatos egyedül nézelődni, mert szemtelen motoros fiúk tűnnek fel.
Még közösen megnézzük a X--XVII. században épült egyik legősibb és
legnagyobb iráni templomot, a Dzsáme-mecsetet. Csipkés falfaragásai
a granadai Alhambráéihoz hasonlítanak. A harmincezer négyzetméteren
a szeldzsuk, mongol, szafavid és barokk idők nyomai is fellelhetők.

Elromlik a légkondicionáló a buszban, majd a másodszori javítás
után már tetőszellőzéssel érjük el \textbf{Jazd}ot.

Minden szállóban van mosoda, így problémamentesen vészelem át
két csadorommal a kéthetes sivatagi időjárást. Este pedig egy egészen
olaszos pizzaéttermet találunk vacsorázóhelyül. A visszás helyzetet
jelzi, hogy nyugati áru nemigen kapható, de van Iránburger és a
pepsi-colás üveg hasonmásába palackozott, helyi gyártású, kólaízű üdítő.

Jazdban csatlakozik hozzánk az ELTE jelenleg perzsa szakos tanulója,
Ede, aki később, a háború alatt több hetet tölt Irakban egy ismert
tudósító mellett arab tolmácsként.

Jazd látképe nem sokat változott az évszázadok során, ma is úgy néz
ki, mint ahogy Marco Polo láthatta egykoron. A távolban kopár hegyek,
a város fölött a mecsetek nyúlánk tornyai és a szellőzőtornyok magasodnak
ki, egy-egy pálmafa ékelődik az egységesen homokszínű házak közé.

Reggel a zoroasztriánus temetkezési helyeket, az ún. Hallgatás tornyait
másszuk meg. Negyven éve már nem használják ezeket a felhordott
domb tetején álló, erődszerű építményeket temetési célból, a sah
tiltotta meg a turisták miatt. Meredeken mászunk felfelé, a végén a
kétméteres védőfalat is le kell küzdeni, hogy feljussunk az erőd tetejére.
A halottakat ide helyezték ülő helyzetben, koncentrikus körökben.
Legbelül a gyerekek, középen a nők és a fal mellett a férfiak. A keselyűk
és a kutyák tisztították le a csontokat. A felügyelő pap figyelte,
aki egyedüliként léphetett be ide, hogy mikor alkalmas a csontokat
összeszedni és urnába helyezni. A zoroasztriánusok úgy vélik, hogy
a négy elemet (tűz, víz, levegő, föld) nem szabad beszennyezni, ezért
temetkeztek ilyenformán.

\illustration{page-100}{A Hallgatás egyik tornya Jazdban}

Gyuri felmászik a falra, hogy a távoli hegyeket fotózza, majd befelé
fordul és úgy guggol, mint egy keselyű, még szerencse, hogy hullák
már nincsenek.

Az időszámítás előtti időkben gyökerező egykori iráni államvallásnak
ma már csak hatvanezer híve él országszerte. Nem térítő vallás,
vagyis nem lehet belépni, és a vegyes házasság sem megengedett. Valószínűleg
ez lett a veszte, és ez a nagyszámú fogyatkozás oka. Indiában
él még jelentős zoroasztriánus közösség, ők voltak az angol gyarmatbirodalom
idején India nagy vagyonra szert tett, kereskedő, az angolok
és a hinduk között közvetítő népcsoportja. A Queen együttes legendás
énekese, az iráni származású Freddie Mercury is zoroasztriánus
volt. Tűztemplomokat építettek, a tüzet mint jelképet, Isten utolsó
legtökéletesebb teremtményét, a tisztaság és bölcsesség szimbólumát
tisztelik. Ezekben a tűztemplomokban barackfával táplált tüzet őriznek,
abban, melyet meglátogattunk, ezerötszáz éve lobog. A templomok
homlokzatán az egyiptomi feliratokból is jól ismert, kiterjesztett
sólyomszárnyakon lebegő napkorongból kiemelkedő emberalak képében
itt Ahura Mazdá látható a három erkölcsiség: a jó szavak, jó gondolatok
és jó tettek feliratával.

Attila előző este remek éttermet talált, ezt próbáljuk most megkeresni,
de a taxis pofátlanul körbevisz a városon, hogy majdhogynem
a kiindulópontra térjen vissza. Jót kiabálnak vele a fiúk, a töredékét
fizetjük, mint amennyit kér, és beviharzunk ebédelni a tényleg elegáns
étterembe. A joghurtot minden formában készítik Iránban, itt megkóstolhatjuk
levesként, szenzációs a sajnos nem tudom, milyen fűszerekkel
ízesített, hideg leves. Attila végre itt talál az étlapon faggyúlevest,
mely iráni specialitás. Tűzforrón kell enni, egyébként rárakódik
a faggyú a gyomorfalra. Régen eredményes kivégzési módszer volt ez,
amikor a faggyúra hideg vizet itattak a szerencsétlennel. Szagolni se
merek a csészéje felé. Ebéd után átvonulunk egy helyiséggel arrébb
teázni és vízipipázni. Pár órát elücsörgünk itt a szőnyegeken, a legtöbbször
almaízesítésű dohány mellett. Az egyébként nem dohányzók,
mint én is, füstölünk most egy kicsit.

Este összeverődünk a mecset előtti téren, csak páran jöttünk el
lányok, mert az iráni birkózást, az úgynevezett zurhanét csak férfiak
nézhetik. Miki örömmel jelenti, hogy sikerült rábeszélnie a vezetőt,
hogy egypár külföldi nő is bemehessen. Belépünk a kétszáz éves ciszternába,
és leülünk körben a szőnyegekre. Középen, egy kör alakú arénában
állnak a férfiak már beöltözve, egyforma kockás takaró van a
derekukra csavarva. A zurhane az iszlám előtti időkre nyúlik vissza,
közös időtöltést, kötelékeket jelent, célja a férfiközösség kialakítása.
Inkább a testgyakorlatok közbeni közös imádkozásra helyeződik
a hangsúly, mint az edzésre, de ahogy láttuk, azért jól megizzadtak
a fiúk a torna közben. Spirituális vezetőjük a mozséd, aki ütemesen
dobol egy emelvényen a torna közben, döntően vallási, síita énekeket
énekel, illetve klasszikus perzsa szövegeket mond. Az edzés gyakorlatsorokból
és erőfitogtatásból áll, amelyhez 4-5 vagy 8-10 kilogrammos
hosszúkás súlyzókat használnak mind a két kezükben. Rituális testgyakorlás
ez, benne extatikus körözés a súlyokkal, melyeket a válluk
felett lengetnek. A zurhane, hasonlóan a jógához, testi-lelki edzés,
melyet, péntek kivételével, mindennap űzhetnek.

Másnap, újabb sok kilométer után \textbf{Pászárgád}ba (Paszargadai),
a kopár területen fekvő ókori romokhoz érkezünk. A hely az ókorban
erdős, dús vegetációjú erőd lehetett, ahonnan szemmel tartották a völgyet,
de ma pusztaság. A tűző napsütésben nyitott esernyővel indulok
Miki után, aki már sokadszorra invitál bennünket történelmi sétára.
A felállványozott Nagy Kürosz-sír, szégyellem, de nem hoz lázba, inkább
hullámzó habokat képzelek a gyepű szélére a kövekről visszaverődő
hőség ellentételezésére. Majd visszazökkenek, és hallom, hogy az
aranykoporsóban nyugvó, bebalzsamozott Kürosz sírját kifosztották.
Nagy Sándor elfogatta a sírrablókat és felakasztatta őket. Pedig Kürosz
megérdemelné töretlen figyelmem, a méd és a perzsa népet egyesítő,
hatalmas birodalmat építő uralkodó volt. Uralma alatt ugyanis az
egyesített népek szabadon gyakorolhatták vallásukat. Kürosz utódai
azonban felhagytak az államalapító vallási politikájával.

\textbf{Naks-e Rosztam} az óperzsa királysírok és szikladomborművek területe.
Az építésükkor valószínűleg a talajszinten álló sziklasírok bejárata
ma már több tíz méter magasról közelíthető csak meg. Palotahomlokzat-utánzat
veszi körül a sírbejáratot, melyen életképek, vallási jelenetek
láthatók, itt nyugszik a Perszepoliszt építő Dareiosz és Xerxész is. Miki
perzsa kultúra iránti rajongását jelzi az I.~Sándor (i.~sz. 240--272)
megkoronázását ábrázoló, sziklába vésett domborműről mondott megjegyzése,
hogy a jelenet a szászánida lángoló barokk szép példája.

\illustration{page-103}{%
	I.~Sahpur királynak a Valerianus római császár csapatai feletti győzelmét
	megörökítő dombormű Naks-e Rosztamban. A sziklába vájt alkotáson a lovon
	ülő perzsa uralkodó jobb kezét a legyőzötteknek nyújtja, a balját a
	kardmarkolaton tartja.}

Délutánra Perszepoliszba, az i.~e. VI. században épült és mindössze
180 évig fennálló császárvárosba érkezünk. I.~Dareiosz és I.~Xerxész
építtette a palotát egy teraszra. Bejáratát négy, bikafejben végződő,
asszír hatást mutató alak őrzi, az oszlopfőkön egyiptomi lótuszvirág
látható. Főként ceremoniális funkciója volt a palotának, az ünnepségek
során a különböző népek képviselői itt nyújtották át ajándékaikat,
amit a lépcsősorokon domborművek is megörökítenek. Nagy Sándor
hódító útján 335-ben foglalta el a várost, bosszúvágytól hajtva egy
hetéra biztatta fel a palota felgyújtására. Nagy Sándor később megbánta
a palota elpusztítását, és az ő leírásaiból tudjuk, hogy a falak agyagból
épültek, melyeket kívül-belül aranylemezek borítottak. Az oszlopfők
maradtak fenn napjainkra is, mert csak ezek épültek kőből. A tetőzetet
cédrusfa borította, a padlózatot bíbor bársonyszőnyegek fedték. És
most hódító Nagy Sándor névrokona, Nagy Sándor útitársunk visszatér
a tett színhelyére a Xerxész-kapun át. Előtte olyan elődök jártak itt
hazánkból, mint gróf Széchenyi Andor, akinek bevésése olvasható az
egyik oszlopon 1892. december~3-i dátummal. De számos más külföldi
is itt hagyta keze nyomát a köveken, a századelőről a Herald Tribune,
a New York Times újságírói is megörökítették látogatásukat az egykori
hatalmas birodalom székhelyén. Ma már természetesen tilos bármiféle
maradandó üdvözletet magunk után hagyni Perszepoliszban. Lassan
lenyugszik a nap, és a romterületen azzal játszadozom, hogy az oszlopok
előtt különböző hajlásszögekben állok be, így van, amikor a teljes
lenyugvó napkorong ragyog két dombormű között, sejtelmes fénybe
varázsolva az átjárókapukon az egykori perzsa harcosok alakját, de ahol
sűrűbb az oszloperdő, a napfogyatkozáshoz hasonlóan a fénykorong
glóriája jelenik csak meg. Az oszlopok egyre sötétebben merednek a
hegyek mögé lebukó tűzgolyótól narancsszínűre festett égboltra.

\fullpageillustration{page-105}{Óperzsa domborművek Perszepoliszban}

\fullpageillustration{page-106}{Óperzsa domborművek Perszepoliszban}

Buszra szállunk, és \textbf{Siráz}ban, a perzsa költészet fellegvárában
térünk nyugovóra mi is.

A másnapi egész napos, szinte étlen-szomjan megtett szarvesztáni
kirándulást már megbocsátottam Mikinek. Estére érünk vissza Sirázba,
és az elsőként utunkba akadó Bagaméri, ki fagylaltját maga méri
típusú babakocsi-kerekes, szendvicseket árusító vendéglátós aznapi
bevételéhez szépen hozzáteszünk. Az éhhaláltól megmenekültünk, de
muszáj beülni rendesen is megvacsorázni valahová. Kihozzák a levest,
amikor feltűnik, hogy két iráni nő a pincérrel beszélgetve Krisztára
és rám mutogat. Kiderül, hogy a két hölgy megkéri a pincért, hogy
kérdezze meg Eleket, feltételezett férjünket, hogy készíthetne-e
kettőnkről egy fényképet. Micsoda kommunikációs lánc! -- a nők nem
szólíthatnak meg idegen férfit --, Elek pedig vigyorogva int, és minket
már a kutya se kérdez, hogy zavar-e. Kanállal a kezünkben mosolygunk
a szimpátiájukról biztosító ,,nagyon szép, nagyon szép, nagyon
jól áll a kék kendő''-t mutogató két kedves nőre.

\fullpageillustration{page-108}{Három kislány Szarvesztánban}

Következő nap délelőtt Háfiz, a XIV. századbeli nagy perzsa költő
síremlékét látogatjuk meg egy díszkertben, mely zsong az iráni fiataloktól.
Háfiz sírja a titkos találkák helyszíne, a díszkert növényei, virágai
között és a mögötte álló teázóban az iráni fiatalok magas fokra
fejlesztett testbeszéddel és arcjátékkal közvetítenek üzeneteket egymásnak.
A legtöbbször kettesével járó lányoknak csak az arca és a keze
látszik ki a csadorból, de az hangsúlyosan ki van festve, nem tilos a
szemfesték, a rúzs és a feltűnő körömlakk. A változatosságot a fekete
egyenruha mellett a különböző színű kendők adják. Egy-egy verseskötet
pedig jó alkalom a beszélgetés megkezdésére. Háfiz mérhetetlen
életszeretetét tükrözik a nőkről, mulatozásról szóló versei.

\bigskip\noindent
{\otherfamily Samszu'd-Dín Muhammad Háfiz: \textit{Ha rózsák nyílnak\dots}}
\bigskip

\begin{LVerse}
Ha rózsák nyílnak, ennél, mondd, van-é szebb?	\\
De százszor szebb, ha bort kortyolva nézed.	\\
A szépség int, tagadd meg hát, s ne gondold:	\\
Örökké vár a kagyló gyöngye téged.		\\
Ne késs bort inni rózsakert ölében:		\\
A rózsák még talán egy hétig élnek.		\\
Arany borral színültig töltsd a kancsót,	\\
S kínáld kincsként a megfáradt szegénynek.	\\
Siess, ó, sejk, az én kocsmám borát idd:	\\
Ilyen bort még a mennyekben se mérnek.		\\
Tanulhatsz itt, de nem kell könyv, füzet sem:	\\
Szerelmet holt betűkből, mondd, ki ért meg?	\\
(\dots)
\end{LVerse}

\hfill (Képes Géza fordítása)
\bigskip

Másnap, a Perzsa-öböl felé tartva megállunk \textbf{Bisápúr}ban, újabb
szászánida történelmi pillanatokat ábrázoló szikladomborművekért.
I.~Sápúr győzelmét láthatjuk a rómaiak felett, több száz perzsa lovas
sziklába vésett felvonulását.

Elérjük utunk legdélebbi pontját, \textbf{Busehr}t, melynek kikötőjéből a
portugálok Ázsia más részeibe exportálták a kitűnő sirázi bort a XVI.
században. A Perzsa-öböl és utazásunk következő állomása, Ahvaz
már nem perzsák, hanem főként arabok lakta terület. A negyvenfokos
hőségben a Perzsa-öböl vize is harminc fok körüli, a fiúk nekiindulnak
a haboknak, Orsi sortra és pólóra vetkőzik, és szintén úszik egyet
ruhástul. Térdig én is belegyalogolok, majd inkább felsétálok a
légkondicionált étterembe. Indulunk tovább, az út mentén egymást érik az
égő tetejű olajkutak, mellettünk haladnak a csővezetékek, előrenézve
látjuk, ahogy a hegyeken, völgyeken át, az országút mentén hosszan
kígyóznak. Olykor pálmaligetek szakítják meg a kopár tájat.

A \textbf{csoga-zanbil}i zikkuratot, vagyis szentélyt, a legrégebbi ókori
perzsa emléket is egy olajpróbafúrás során, véletlenül fedezték fel. Az
i.~e. 1300-ból származó temetkezési hely áldozati toronnyal eredetileg
60 méter magas volt, ma csak 25, és a hét emeletéből öt látszik ki a
földből. Az agyagból tapasztott téglát csempe borította, a tetején álló
istenségszoborhoz a planétaszférákat jelképező lépcsőfokok vezettek
a világegyetemet szimbolizálva. Az egyik kőlapon egy árva lábnyomot
látunk, állítólag 3300 éves ez is. A falak hatalmas tégláira vésett
legrégebbi ékírás sok száz sorában azonban nem kételkedünk.

Majd az óperzsa királyok téli rezidenciájaként szolgáló \textbf{Szúzá}ba
érünk. E korból származó emléket nem láthatunk, a mongolok elefántokkal
tapostatták szét a palotákat. Ellenben itt áll Dániel próféta sírja,
egyike az Iránban található három bibliai sírnak Eszteré és Habakukké
mellett. Nagy Sándor Indiából visszatérve tartatta meg a nevezetes szúzai
menyegzőt, amikor több ezer makedón katonai vezetőt és harcost
perzsa arisztokraták és közrendűek lányaival való házasságra kényszerített,
ő maga is feleségül vette III.~Dareiosz lányát, Sztateirát.

Már \textbf{Hamadán}ban megveszem a Bangkokban vásárolt aranyláncom
karkötőjét. Nem gondoltam volna, hogy a két szép olasz ékszert
a világ e két, egymástól távoli pontján párosítom össze. Az ószövetségi
Eszter királyné, I.~Xerxész felesége a nyarakat az egykori méd fővárosban,
Ekbatanában -- melynek neve gyülekezőhelyet jelent -- töltötte,
e területet kedvelte leginkább, így itt lett eltemetve. E városban hunyt
el Avicenna, a világhírű mozlim tudós, orvos, filozófus.

A szállodaszobánk ablaka egy mecset udvarára néz, garantált a hajnali
ébredés az ötórai hangos imaszóra. A \textbf{tak-e-bosztan}i domborműveket
elhagyva érünk a fele-fele arányban perzsák és kurdok lakta, az iraki
határhoz közeli \textbf{Kermansah}ba. Megrendítő az iraki-iráni háború
múzeuma. A háború kirobbanása az 1930-as évek határfolyó-problémájára
nyúlt vissza. Azért kerülhetett Khomeini is száműzetésekor Dél-Irakba,
hogy ezzel sakkban tartsa Iránt. Szaddám Huszein iraki elnök és a sah
határegyezményét követően menekült Franciaországba. A Szovjetunió
mellett az USA-nak is megfelelő volt a sah rendszere, amikor azonban
az iráni felkelők elfoglalták a teheráni nagykövetséget, Amerikának nem
volt érdeke tovább támogatni Iránt. Szaddám Huszein bátorítva érezte
magát, a határviták kiújultak, és 1980-ban Irak megtámadta Iránt. Az
irániak pedig lassan felzárkóztak -- a történelem fintora -- az USA által
itt hagyott fegyvereken kiképzett emberekkel. A háború semmit nem
változtatott a határokon, viszont 1988-ra hárommillió ember vesztette
életét. Az ENSZ 1989-ben agresszornak nyilvánította Irakot. A háború
lezárulta után eltelt három év, és 1991-ben, az öbölháborúban Irán már
Irakot támogatta, az 1979-es rendszerváltásból fakadó Amerika-ellenessége
miatt. Huzisztán tartomány, ahol most járunk, szenvedte a legtöbbet
ebben az értelmetlen politikai, hatalmi harcban, amely nem kímélte a
hétköznapi emberek, a nép életét, a környezetet, az élővilágot. Borzalmas
képeket látunk a vegyi fegyverek okozta szenvedésről, halálról.

Próbáljuk elhessegetni a letaglózó múzeumlátogatás emlékét, este a
városban perzsa popzenét tartalmazó CD-t keresünk, de az utcai zenebódéban
egy sincs. Hangszóróból szól a zene, Kriszta igazán moderáltan
ringatja magát, mégis rászólnak, hogy hagyjon fel az illetlen viselkedéssel.
Továbbsétálva megismerkedünk két perzsa fiúval, őket is megkérdezzük.
Elkísérnek bennünket egy boltba, ahol óvatosságból kattan mögöttünk
a zár. A kérdezett popzene ugyanis hivatalosan nem hallgatható, a Kaliforniába
szakadt perzsa közösség készít klipeket, és ezekből válogathatunk
most, melyeket számítógépről CD-re másolnak nekünk.

Szeretném megköszönni minden perzsa ismerősünknek, akikkel
az út során találkoztunk, a kedvességüket, a nyíltságukat, a vendégszeretetüket,
a történeteiket, amiért bepillanthattunk a fekete fátyol
mögé, hogy lássuk, mennyire sokszínű és reménnyel teli a kényszerűen
eltakart életszeretet és szabadságvágy.

Még két állomásunk van hátra, \textbf{Takt-e Szolejman}, egy kihunyt
vulkán tetejére épült párthus kori szentély, ahová a szászánida királyok
zarándokoltak el megkoronázásukat követően. Az út mentén furcsa
táblára leszek figyelmes, egy telefonkagylót ábrázoló hirdetőtábla jelzi
e terület gyér telekommunikációs ellátottságát. \textbf{Karakerisza} a Szent
Tádéhoz kapcsolt legnagyobb örmény zarándokhely. Júniusban a világ
minden részéről érkeznek ide örmények ünnepelni, ilyenkor tele van
sátrakkal a völgy.

A határon átérve, a török oldalon átöltözhetnénk, de ekkor furcsa
érzésem támad, és kiderül, nemcsak nekem. Nem akaródzik levenni a
fejkendőt, ez hihetetlen! Eltelik pár perc, mire lekapom, de mintha
meztelen lennék hirtelen. A török utolsó kapunál megállítanak, mert egy
pecsét korábban lemaradt a busz okmányairól, de az ablak, ahol ezt
rányomják, bizonytalan időre bezárt. Leszállunk, én kiskutyákat etetek és
itatok, nagyon sajnálom a sovány állatokat. A morózus vámos fél szemmel
lesi, várunk vagy egy órát, majd megkönyörül rajtunk és kinyittatja
a hivatalt. Felajánlja a kutyákat ajándékként, de nem vállalkozom, tudva
a még hátralévő három napot és sok száz előttünk álló kilométert.

Ankarában már a bőrönd mélyén lapul a kendőm, felszabadultan
járjuk a várost. Isztambulban irány a bazár, ajándékokat vásárolok,
többek között a megígért iráni kaviárt is itt veszem meg. Az utolsó estén
kiülünk a Boszporusz-partra, a Kék mecsetnél élőzenét hallgatunk hajnalig.
Három óra alvás után nekiveselkedünk az utolsó útszakasznak.
Már Pest határában járunk, amikor a Juventus rádiót hallgatva nem
hiszünk a fülünknek. A rádió hírül adja, hogy az iráni Zam-Zam kóla
jó minőségének köszönhetően betört a szaúd-arábiai piacra. 12650~km
megtétele után, reggel 7~órára megérkezünk a Hősök terére.

{\centering ***\par}

(Varró Zoli emlékére, akit az út után három hónappal halálos motorbaleset
ért.)
