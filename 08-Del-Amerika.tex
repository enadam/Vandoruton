\clearpage
\chapter{ÚJRA DÉL-AMERIKÁBAN -- BRAZÍLIA, PARAGUAY, BOLÍVIA, PERU}

\begin{Verse}
Kedves ég, tüzet szeretnék!		\\
Lángoló fekete szempárt,		\\
felszabadító szerelmes kegyet.		\\
Izzó lávatömeget,			\\
sűrű, lezúgó vizet.			\\
Értelmem kapujába tisztító tüzet,	\\
szívem bánatára szivárványhidat.

Kedves ég, tüzet szeretnék!		\\
Égető tajtékkal rohanót,		\\
fehér fátyollal gyógyítót.		\\
Buja zöld nászágyon			\\
cirógató szeretőt.

Kedves ég, tüzet szeretnék!		\\
Könnyű szívvel élőt,			\\
szenvedéllyel féltőt.			\\
Alászálló angyalt			\\
felbukkanó szárnnyal.			\\
Intenzív perceket az öröklét mellett.

Szentséges ég, tüzet szeretnék!		\\
Lobogjon szívem, így LEGYEN.
\end{Verse}

\clearpage
A kaiganok főnökének szép lányát választotta feleségül a kígyóisten.
Naipi azonban a bátor harcost, Tarobát szerette. A lány és a kígyóisten
esküvőjének napján a násznép és maga az istenség is bódult álomba
merült, az istenek kedvelt italától, a cauimottól. A két szerelmes csónakba
szállt, és evezni kezdett lefelé a folyón. Csakhogy a kígyóisten
felébredt, nagy haragra gerjedt, és egy óriáskígyó alakját öltve farkával
felkavarta a csendes Iguacu vizét, majd magát a gránit folyómedret is
megrázta. Amikor a porfelhő feloszlott, előtűnt egy hatalmas kráter,
amelybe a folyó óriási robajjal kezdett belezuhanni. Így szól a helyi
indián legenda.

Százhúszmillió évvel ezelőtt egy vulkánkitörés hozta létre a három
ország: Brazília, Paraguay és Argentína határcsücskén elterülő, több
mint kétszáz zuhatagot számláló, legnagyobb víztömegű vízesésrendszert
a világon, a \textbf{Foz do Iguacu}t.
