\clearpage
\chapter{TÁJAK, VÁROSOK, PILLANATHANGULATOK}

\section{Apollón karjában \hfill Nap -- Vénusz}

Abban az évben nem ragyogott a nap Ciprus felett. Leakasztottam
egéről, és kölcsönvettem.

Apollón karja az olajfák ölelése, mosolya a hevülő sziklaszirté,
melyre Aphrodité kikél a habokból, és megpihen rajta. Haja színe a
perzsa harcosoké, szeme melegsége barnítja bőröm. Sugárzó napfény,
hát itt a virágzás ideje! Szirmok bomlanak, vörösek. Táncolva élvezi
illatát, és íjának húrját pengetve hangja zenéjével bolondít. Hagyja élni a
virágot, nem tépi ki a földből. Egy szirmot örökül nyújtok, és hajlékony
derekammal intek felé, menj tovább utadon, Nagy Élet, szerettelek,

\section{Néva-parti harmonikaszó \hfill Hold -- Nep\-tu\-nusz}

Szentpétervár az egyik kedvenc városom, mert lelke van. Versailles-nak
felel visszaverődő fénnyel.

Télen a legmegkapóbb, amikor hiába takarítják a havat, kis idő
múlva megint minden olyan szikrázóan fehér, csendes és andalító,
mint egy Csehov-darab. Ilyenkor várakozik a város, mint a három nővér,
hogy valakit végre magához ölelhessen. Nem siet, megvárja, míg
intesz, és nagyon tud várni, mint Olga, Mása és Irina. Egy város, mely
nappal is álmodozó. Szürkületkor, mely hamar jön, a Néván hangtalanul
siklanak a hajók, és vöröseskék az ég alja. A parton harmonikaszó
csendül, és a férfiak búsan, ábrándos szemekkel énekelnek szerelemről,
örömről, bánatról és a messzeségről.

Te felszállsz nagy lelkük ladikjára, és védő karjukban ringatózol az
óceánon. Már álmodsz.

Átülsz a díszesen faragott szánra, meghúzzák a gyeplőt, és csengettyűszóval
repülsz a sűrű fenyvesek felé. Pavlovszkba érsz, a cári vadászkastélyhoz.
A kert útjait beborítja a hó, csakúgy mint a mesterek
szobrainak és kútjainak hajlatait. Szentpétervár fehér hattyúnyaka
köré pazar nyakék került, Pavlovszk, Petrodvorec és Puskin palotáival
mint ékkövekkel, foglalatként az őket körülölelő terebélyes díszkertekkel.

Puskinban a költő egy bronzpadon pihen maga elé nézve, elrévedő
tekintettel.

Nyár van. A petrodvoreci kastély előtt, két aranyozott Poszeidón-szobor
társaságában matrózok fújják a rezet. A tenger istene a Finn-öböl
felé fordulva vigyázza birodalmát. Görög, római hősök és kancsók
mellett visz le a lépcsősor a kertbe. Séta közben előfordul, hogy
megindul mellettünk egy vízsugár, és széles ívben hátba talál. De
legyünk óvatosak a pihenőpadokkal is, és ezt a kis, örökül hagyott
ős-vidámparkot bocsássuk meg Nagy Péter cárnak.

A szökőkutaktól induló csatornán Poszeidónnak szabad az útja az
öböl felé. Itt szárnyashajóra szállunk, és visszakanyarodunk Észak
Velencéjébe. A lemenő nap fényén behajózva, a Néva aranyhídján
kikötünk az Ermitázs mellett.

A szerenád lassan véget ér. A harmonika szélesre tágul, majd összezáródnak
lapjai, és a Puskin-képeskönyv becsukódik.

\section{Párizsi újév \hfill Vénusz -- Merkúr}

A clamari kiskastély manzárdszobájában a rózsaszín selyemtapétáról
halványlila madarak szállnak az ablakból idelátszó Eiffel-torony felé.
A fürdő közepén lábaskád megkopott fényezéssel, illatos szappannal.
A nagypolgári otthont ősfák birtokolják, körbeveszik a házat és
beköszönnek lombjukkal az emeletekre, melyekre nyikorgó falépcsők
tekergőznek fel. A konyhában, öregedő fadeszkán sajtok pihennek,
a tűzhely alatti fonott kosárból friss fűszernövények hajolnak ki.
A macska nyugodt, lassú léptekkel szeli át a konyhát, sármosan rám
pislant, majd elegáns faroklengetéssel távozik a hátsó ajtón át.
A kisrádióból ferde hanyagsággal, peckesen áll ki egy villa, a szerkezet
enélkül már nem működőképes. A nagymama néha teker egyet a villán, és
a készülék recsegve felébred.

A fiúk zsirardikalapot öltenek a mulatságra, és Újév napján kuglófot
eszünk belesütött arany papírkoronával. Cinkos összenézések tarkítják
a menüt, a családfő nyílt, szenvedélyes pillantást küld a felesége
felé. Könnyedén zárjuk az étkezést, s úgy lépünk ki az ebédlőből, mint
egy tárlatról, ahol élveztük a kiállítás darabjait, s ma is tettünk valamit
a művészetekért.

\section{Dakar-rali \hfill Mars -- Vénusz}

A rajtvonalnál acélszerkezetek duzzadó izmaikat ed\-zik a start előtt.
Bőgő, zajos motorok, fényes abroncsok feszülnek a nyílhegyre, s a
startpisztoly dörrenésekor egyszerre, elsőnek ugranak előre gyorsuló
fordulatszámmal. Sivatagi hajsza Afrika dűnéin, küzdelem a túlélésért
minden egyes ütközés a kemény homokba. Céltudatos harc a győzelemért.
Hőstett a Szahara tűzviharában száguldó forró szekéren a
legrövidebb úton célba érni a dakari Rózsa-tónál. A versenyszellemmel
beoltott ego minden ellenállást elsöprő, féktelen rohama a szemben
álló ponthoz, Vénusz hullámzó sziromfodrai felé. Az egyetlen
bátor győző megérdemelt jutalma beágyazódva örökre megpihenni e
rózsaágyon.

\section{Greenwich \hfill Jupiter}

\begin{LVerse}
Hétmérföldes csizmámban keletet és nyugatot is átfogom,			\\
A nulladik hosszúsági fokon széttárom karom.				\\
Távlatok nyílnak az égi törvény felé,					\\
Erre biztat a teleszkópban a kép,					\\
Bőségszaru az égi csillagtermés.

Az obszervatóriumban segítőkészen hajol fölém éppen,			\\
Békülékenyen nézi csámpás beilleszkedésem.				\\
Hitem szerint, és nem tévedek, az ott a Nagy Göncöl fénylő szekere.	\\
Azonnal van is egy ötletem, majd a többi közé ezt is felveszem,		\\
Terveim száma egyre csak növekszik bőségesen.

Szerencse, hogy belátom, tán túllőttem a célon,				\\
Mert lent a széles dombon egy nemes kis lovag sürgető reménye,		\\
Hogy nagynénje igazságérzete felébred,					\\
Lehullott csillaga már vár rá a réten.					\\
Az élet nagy kérdése az ő személye, igyekezzen tanulni tőle,		\\
Különben ő leül a fűre, s még hat ökörrel sem mozdul el tőle.
\end{LVerse}

\section{A grönlandi jégmezők \hfill Szaturnusz}

\begin{LVerse}
Szikrázó fehérség, ég felé nyúló kúpok,					\\
áttetsző jégruhák a tiszta kék vízben.					\\
Azúr kanyarok a nagy fehérségben,					\\
jéghegyek mozdulnak a hatalmas csöndben.				\\
Tintába mártott hópelyhes paplan,					\\
a dermedt jégvilág aluszik abban.
\end{LVerse}
