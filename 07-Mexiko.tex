\clearpage
\chapter{MEXIKÓ, GUATEMALA, HONDURAS}

A Bellas Arrestől nem messze, a kék-fehér csempével borított Andalúz-ház
előtt, a hátán elterülve fekszik egy caballero, mintha csak egy
hegyoldalon aludna. A mexikói kutyák is hasonló pózban, teljesen
kinyúlva pihennek egy-egy ház oldalánál vagy az úton, de nem a meleg
miatt, mert nincs kánikula. Valóban kilehelték lelküket, vagy csak
léteznek ,,a lét elviselhetetlen könnyűségében''%
\footnote{Utalás Milan Kundera azonos című regényére.}? Ennyi döglött kutya
mégsem lehet a mexikói utakon. Háziállataink egyszerűen átveszik
a gazda életstílusát, és egy idő után még külsejük és modoruk is
hasonlatossá válik hozzá. Fekszenek a létezés mozdulatlanságában, mint
egy Salvador Dalí-kép, melyen lefolyik az óra a falról, és eltűnik az idő.

A caballero és a kutya nem csinál semmit az Andalúz-ház előtt,
csak van, és megmozdíthatatlan. Így találkozott tudatom az abszolút
szellemmel az aszfaltozott járdán. S az ilyen találkozások mindig
formálnak rajtunk, egyszerűen attól, hogy meglátjuk, hogy van, hogy
létezik. Mi pedig mozdulunk tovább az időben, de már emlékezünk,
hogy létezik egy örök jelen, s lépteink lelassulnak.

\bigskip
Az aztékok alapította \textbf{Mexikóváros}ban vagyunk, vagy, ahogy az
aztékok hívták, Tenochtitlánban, a főtéren. A legenda szerint Tenoch
főpap addig vezette népét, míg jelet nem kapott az égiektől a
városalapításra. Amikor megpillantott egy kaktuszon pihenő, a szájában
kígyót tartó sast, úgy gondolta, ez az a hely, amely alkalmas a
letelepedésre. A szájában kígyót tartó sas a fény és a sötétség kettősében
a fény győzelmét hirdeti. Így alapították meg egy mocsaras tórendszerre
építkezve a mai Mexikó fővárosát. A tér nagy kockakövei néhol
hosszabb vonalban is kidudorodnak és behorpadnak a vizes, homokos
talaj miatt. Ugyanígy a templomok oszlopai és falai is süllyednek,
mutatja a kupolából leengedett inga.

A katalán Gaudinak volt honnan merítenie, az 1700-as években
épült katedrális külső oldalfalai kísértetiesen hasonlóan burjánzóan
díszítettek, mint a barcelonai Sagrada Família. A tenochtitláni rommező
előtt népviseletbe öltözött indián férfi fennhangon magyarázza
a hitük szerint a Tejútról származó nép naptárát, még Attila és a
magyarok is szóba kerülnek.

A téren a spanyol gyarmati stílusban épült Nemzeti Palota őrei
tisztelegve engednek be bennünket a parlament udvarára. A lépcsőház
és a galéria falai Diego Rivera freskóitól gazdagon színezettek.
Hatásuk hasonlatos Rivera Frida Kahlóhoz fűződő viszonyához.
Szenvedélyesek, szabadelvűek és színesek, mint a mexikói népviselet.

A Latin-Amerika-torony 42. emeletéről letekintve a lábunk alatt
fekszik a közel 20 milliós város, nyüzsgő hangyabolyként. Háromszor
akkora, mint fővárosunk, de lakosainak száma tízszerese Budapestének.
Ahogy lebukik a nap a hegyek mögé, a fényt a sugárutak kivilágítása
adja, a lámpák egybeolvadó fénye aranyfonalú pókhálóként szövi át a
metropolist, és a hömpölygő autóáradat fényszórói folyékony aranykötelekké
válnak. Lezuhanunk a lifttel, vissza a város tengerszint feletti 2240
méteres magasságára. A szitáló esőben a kockaköveken nagyokat lépve
beugrom a zeneboltba. Luís Miguelt, Roberto Carlost és Gloria Estefant
hallgatva végképp magam mögött hagyom az öreg kontinenst. Elkap a
latin temperamentum, mely egyszerre lüktető és ernyedt. Belehelyezkedem
Latin-Amerika ritmusába, és a zöld svábbogár iránytaxiból hangos
muchas graciasszal kiszállva teljes szívvel megérkezem az Újvilágba.

Másnap a Chapultepec Parkon át közelítjük meg az Antropológiai
Múzeumot. Pálcikára tűzött, hámozott és cikcakkosan bevagdosott
mangót kóstolgat a csoport. Hősiesen ellenállok, félve a beharangozott
gyomorpanaszoktól. De eltelik pár nap, és már minden után odanyúlok,
amire gusztusom támad. Egy hét után fenntartások nélkül, jóízűen
megvacsorázom a piacon is, az árusoknak üzemelő talponállóban.
A négyhetes körutazást szó nélkül tűri a gyomrom.

Délután utazunk \textbf{Teotihuacán}ba. Istenek lakhelye, ahol az istenek
születnek, a hely, ahol az emberek istenekké válnak -- ezek a jelentései
náhuatl nyelven az ősi város nevének.

A város ismeretlen okból elnéptelenedett, az aztékok ideérkezte
előtt hatszáz évvel már lakatlan volt. Amikor az aztékok meglátták
a hatalmas, lépcsős piramisokat, nem tudták elképzelni, hogy azokat
emberkéz hordta fel. Tőlük származnak a Nap-, illetve Hold-piramis
és a Halottak útja elnevezések.

Nincsenek történeti források és feliratok sem, így nem lehet megállapítani,
hogy kik voltak az alapítói és mi volt az eredeti neve. A város
életét a feltételezések szerint azok a nemesi csoportok irányították,
amelyek érvényt szereztek a kozmikus erőket megtestesítő isteni hatalommal
kialakított védnökségüknek. Általában Közép-Amerikában a
fővárosok a világmindenség központjai voltak. A legtöbb kormányzati
pozíció dualitásában a világegyetem alapvető kettőssége nyilvánult
meg. Ilyen istenek voltak az égi isten Tollaskígyó és az alvilágot
képviselő Jaguár egymással szemben álló, ugyanakkor egymást kiegészítő
alkotóelemként.

\illustration{page-043}{%
	A Halottak útja Teotithuacánban --
	a Termékenység és Újjászületés útja?}

Zihál a tüdőm és szédülök, mire a magas, meredek lépcsőkön a
Nap-piramis tetejére felérek. Végigtekintve az ősi vallási központ
területén, rálátást nyerek egy olyan városra, amely mintha egy nagyobb
,,csillagváros'' szabályai szerint épült volna meg pontosan így.
Egy mexikói csoport égre emelt tenyerekkel, terpeszállásban fordul
arccal az ég felé. Testtartásuk pentagramma alakzatot formál, mint
azon papoké, akik akár itt, Teotihuacánban a piramis tetején állva,
vagy az ősi Egyiptomban kapcsolatot akartak felvenni az égiekkel.
A szimbolikában a pentagramma jelenti a mágikus, beleérző képességeket,
melyek az anyagi világon túlmutatnak a szellemi felé,
és megnyitják az öt titkos kaput, vagyis az öt érzékszervet: a látást,
a hallást, a szaglást, a tapintást és az ízlelést. A pentagramma
testtartás egy nyitott állapot, ahol ki van tárva az ember szíve.
A hagyományok szerint az aztékok beszélgettek a szívükkel, pontosabban a
benne lakozó istenséggel.

Úgy áll itt a mexikói csoport, mint egykor az istenek, akik Teotihuacánban
találkoztak, hogy megteremtsék a Napot és a Holdat.
A legenda szerint ők hozták létre a két piramist, hogy annak tetejéről
a mélybe vessék magukat, majd Napként és Holdként újjászülessenek.
Itt állt valaha a szentély az öt emelet magas piramison, ahol a papok
még dobogó emberi szíveket áldoztak az isteneknek tűzbe vetve, majd
lehajítva azokat a mélybe azért, hogy a Nap újra feltűnjön a horizonton
és termékennyé tegye a földet.

A piramis lábánál indiánok maguk készítette faragott botokat,
szobrokat árulnak. Az egyik indián puha ruhából kicsomagol egy
obszidiánszobrot. A fekete anyagba mintha aranyport szórtak volna. Ha
forgatom, hol hollófekete színű, hol pedig aranyosan csillogóvá válik a
Napisten alakja. Az indián lehajol, és előveszi táskájából a Napisten
párját, a Holdistennő szobrát is.

A Hold-piramisra már nem tudunk felmászni, mert zár a romterület.
Még elgyalogolok a lábáig, ahol a Nap mellém áll a fényképezkedéshez,
én belekarolok, ő pedig átölelve tart az áldozati kőnél.

Lassan homályba borul a piramisváros, a fényhegy kőből rakott
mása, de kiviláglik egy másik, egy boltíves, fölötte. Sólyomasszony
életre kel, szerelmese azonban farkassá változik. Az obszidiánszobor
aranyos csillogása éjfeketébe vált, s átlép egy másik világba. Ha majd
véget ér az átok, a két világ összeér, az asszony újra kedvese arcába
nézhet, és minden eggyé válik. Ezt üzeni Teotihuacán, Gíza, a kínai bölcs,
a Tádzs Mahal láthatatlan fekete párjával, Borobudur, Ahura Mazda
és Animán dualizmusa az ősiráni hitvilágban, valamint a pálosok
rendje. Addig azonban szétszakítva áll a két világ egymást elkerülve s
meg nem értve, boldogtalanul, vágyakozva egymást keresve. Amikor
Coatlicue istennő két egymással szembeforduló kígyófeje egymásba
olvad, a béke időszaka jön el, jin és jang kiegészíti egymást.

\bigskip
\begin{itshape}
Ha megteremted az egyensúlyt magadban, akkor már nem imbolyogsz
a piramis csúcsán, mert egy teljes világot látsz.
\end{itshape}
\bigskip

A harcos aztékok által elnevezett utat és a Tollaskígyó piramisát
másképp is lehetne értelmezni. Halottak útjának nevezték el, mert
úgy vélték, hogy a főút melletti halmok sírokat takarnak, de ezt az
időközbeni ásatások megcáfolták. A város Hold-piramishoz vezető
főútját nem Halottak útjának nevezném, mert az véleményem szerint
a Termékenység és Újjászületés útja. A város jelképrendszere pedig a
Tejútra röpít bennünket. A kifejtéshez azonban előbb ismerkedjünk
meg röviden az egyes világkultúrák egyezőségeivel, pár asztrológiai
fogalommal és a közép-amerikai népcsoportok időbeliségével.

\bigskip
A piramis négyzetalapja a négyes szám, az anyag száma, a piramis
pedig a Bika jelének feleltethető meg a mezopotámiai asztrológiában.
A Bika megfelel Tláloc esőisten alakjának, aki, ha megjelenik,
termékennyé teszi a földet. Teotihuacán harmadik piramisa a Nap- és a
Hold-piramisok mellett a Quetzalcoatl, melynek oldalán kukoricacsövek,
az Esőisten arca és a Tollaskígyó (Quetzalcoatl) keveredik, ez pedig
a Skorpió-jegynek felelhet meg. A termékenység áldozatot követel,
így kerül a Hold-piramis elé az áldozati kő. A tavaszi napéjegyenlőségkor
ember- vagy állatáldozattal biztosították az aztékok a termést.

Az olmék nevet az aztékok adták a teotihuacáni ősi kultúrának,
melynek eredete mind a mai napig rejtély. Az olmékok az i.~e. II-I.
évezredben éltek ezen a helyen. Az időszámítás előtt 1200 táján jelentek
meg az első piramisok és vallási központok Közép-Amerikában.
Az időszámítás előtti 400 táján az olmék civilizáció eltűnt, azonban
kultúráját valamennyi későbbi mezoamerikai kultúra átvette. Teotihuacán
i.~sz. 250-től virágzott, ez idő tájt emelték a Nap-, a Hold- és a
Quetzalcoatl-piramisokat. A kutatók máig nem tudják, hogy kik voltak
a teotihuacáni civilizáció hordozói. Közép-Amerikában itt bukkan
fel először a Tollaskígyó ábrázolása.

Erős hatást gyakorolt az ősi kultúra az i.~e. 600--i.~sz. 800 között élő
zapotékokra, akiknek központja Oaxaca környékén, Monte Albánban
volt. A legnagyobb prekolumbián civilizáció, a maja kultúra magába
olvasztotta az olmékokat, a zapotékokat és a teotihuacáni civilizációt
is. A maják hanyatlásakor harcos népcsoportok érkeztek Észak-Mexikóból,
az aztékok, akik katonai társadalmat hoztak létre, majd őket a
spanyol korona hajtotta uralma alá.

A Nap-piramist az i.~sz. I. században építették, és a Napnak a
napéjegyenlőség idején megtett mozgása szerint tájolták, amerre az égen a
Fiastyúk kel. A Fiastyúk pedig a Bika csillagkép hátán kotlik, és benne
van a termékenység. A Bika szarvai között lép a Nap a Tejútra (a Bika
és a Kos csillagképek határán), alatta van az alvilág folyója, előtte és
felette a Tejút. A Fiastyúk, vagy más néven a Plejádok csillagkép a Bika
csillagkép fölött az a mitikus hely, ahová az istenek szerelmeskedni
járnak, teremtenek. Csillagászatilag az égen egyedül a Plejádok
csillagképben metszik egymást a hét ősbolygó, a Nap, a Hold, a Merkúr,
a Vénusz, a Mars, a Jupiter és a Szaturnusz pályái, mitologikus
fogalmazásban itt van az istenek szerelmi fészke.

A Plejádok csillagkép felelhet meg az aztékok legendás eredethelyének,
Chicomotzocnak, az azték hét barlangnak, mely a hét ősbolygót
képviselheti. Erre utalhat az aztékok legendája, mely szerint az első
Napisten, Tezcatlipoca kezében fekete obszidiántükröt tart, amelyről
hét csillaggal ékes öv vagy nyaklánc függ. A barlang pedig termékenységi
szimbólum, illetve Tejúthasadék-jelkép.

Korai eleink még nem építettek templomokat, barlangokat használtak
naptemplomként. A napkapu a fényhegyek közötti barlangnyílás,
amelyet a csillagos égen a Tejút hasadékainak feleltettek meg. A Nap
és a Hold a dualista teremtésmítosz két fő alakja. A Nap-piramisban
pedig benne van mind a hét bolygó, mert egylényegűek, napistenek
voltak egykoron.

A Nap-piramis alatt hét mesterséges termet találtak, amelyek a
több közép-amerikai nép körében élő ,,Héttermes barlang'' emlékét
idézik, a világ teremtését.

Bonampak egyik falfestményén csatajelenetben hadifoglyokat és
égitestek jelképes ábrázolásait láthatjuk, többek között a Fiastyúk
és az Orion (vagy Nimród) csillagképet. A Nimród a beavatás fázisa
a Tejútra lépés előtt. E csillagképben a teremtésen túl vagyunk, itt
már csillagok születnek. A pálosok a Pilisben és a Káli-medencében a
Nimród erővonalaira építettek kolostorokat. Az itt érzékelhető energiáknak
a mai napig megvan a gyógyító hatásuk.

A teotihuácani ősi vallási központ üzenete egy és ugyanaz a bibliai
üzenettel: porból lettünk és porrá leszünk. Fontos pontosítani, hogy
csillagporból lettünk és csillagporrá válunk ismét, a csillagokból
jöttünk és oda térünk vissza.

Mi most egyelőre térjünk vissza a Tollaskígyó-piramishoz és ábrázolásaihoz.
A Tollaskígyót a legfőbb teremtő és civilizációt létrehozó
isten szerepével ruházták fel, aki az isteni tudást és teremtő energiákat
jelképezi, vagyis Tollaskígyó szimbolizálhatja a Tejutat.

A kígyótest mellett rajzolt kagylók és csigák nemcsak a kutatók
által megállapított termékenységet vagy vizeket jelenthetik, hanem
utalhatnak arra is, hogy ez a kígyó nem itt a földön tekeredik, hanem
a galaxisban. Kagylóval jelölték a nullát, mindennek az eredőjét,
a fentiek alapján pedig ez utalhat tejúti származásra. A Tejút pedig
összeköttetés az istenek és az emberek világa között.

A Tollaskígyó kígyó-madár eredete, mint említettem, az olmék civilizációig
nyúlik vissza. A La Venta-i első olmék központ kőemlékén
csőrös, madárbóbitás kígyó látható. Quetzalcoatl a szél, az orvoslás
és a művészetek védőistene. A La Venta-i piramis kúp alakú, vagyis
a négyzetalapú anyagvilágot jelképező piramis felett a felső világ
szimbóluma. A szertartások célja az volt, hogy transzállapotban kapcsolatba
kerüljenek az istenekkel és természetfeletti tudás birtokába
jussanak.

Mi történik a babiloni Skorpióval, ha el tud szakadni az anyagi léttől?
Sassá válik a kígyó, ez az asztrológiai szimbóluma az Ég
istenének, ahol a sas az ember világa fölötti uralkodó erőt jelképezi.
Mindent megért és éles szemével mindent lát, aki ösztönei felett
uralkodni képes.

A zsidó-keresztény szimbolika is a Sas csillagképet öltözteti tollruhába.
Feltámadásjelkép, mert a sas tollazata megújul és évente egyszer
visszanyeri ifjúságát, amikor a Nap közelébe emelkedik. A Napba
néző sashoz hasonlították az isteni világosságot felfogó embert. János
evangélista jele a kígyóval együtt.

A piramisfalakon tekeredő kígyó és a spirál formájú kagylók (galaxis)
mellett megjelenik a nullát formázó fésűskagyló is, két spirálkagyló
között. Az eddigi termékenységértelmezés mellett a fésűskagyló
jelentheti ismét a születést vagy újjászületést, hiszen Tollaskígyó az
újjászületés és körforgás jelképe is. Hasonló értelmű Venus születése
Botticelli festményén, ahol az istennő egy óriási fésűskagylóból lép ki.
Továbbá a nulla itt jelentheti mindennek az eredőjét, a tejúti származást
és a felszabadult szellemet, amely teste lehullása után boldogan
felrepül a halhatatlan dimenziókba. Ezzel az értelmezéssel újra csak a
Tejútról lépünk le és lépünk oda vissza.

A tollak már a félig sassá, illetve mezoamerikai megfelelőjeként az
isteni quetzallá, a színpompás madárrá váló kígyót jelképezik. Az indián
törzsfőnökök díszes madár-toll-koronájának jelentése is ez, azt a
vezetőt jelöli, aki kapcsolatban áll a szellemvilággal.

Emberi testben (anyagvilág -- kígyó szimbolika) az ember feletti
szellemet is képviseli (sas vagy quetzalmadár, főnix vagy turulmadár).

A teotihuacáni Quetzalcoatl-templom díszítései szintén ezt jelképezik.
A papi lakhely az isteni quetzalmadár palotája, vagyis a Tollaskígyó
teremtő energiáit jelképezi, az ösztönerő felemelését és a földtől
elrugaszkodott lélek gyógyító energiává tételét, a szakrális, teremtő
tudás hírnökét, közvetítőjét, melyet a sok-sok körbe foglalt szem is
jelez. A szem az égi csillag közép-amerikai jele, így válik érthetővé az ősi
város azték elnevezése, Teotihuacán, ahol az istenek teremtenek és az
emberek istenekké válnak.

\fullpageillustration{page-049}{%
	A nagy fogú Tollaskígyó és Tláloc esőisten szögletes fejű, dülledt
	szemű képe a teotihuacáni Qu\-et\-zal\-co\-atl-piramistemplom falán.
	Az istenábrázolások között kígyótestek hullámzanak.}

A Tollaskígyó vagy a babiloni Skorpió, a megújulás szimbólumai
vezetik a lelket a Tejútra. A Hold a lélek jelölője. Teotihuacán
Hold-piramisa áll a Halottak útja (a Teremtés útja) végén, lezárva a város
közepén végigfutó főútvonalat. Azon túl, szimbolikusan a Hold-piramis
alatti nedves barlangon túl már egy az emberek számára láthatatlan
világ áll. A Hold-piramist a valóságban is egy barlang fölé építették.
Így a főút nem is futhat láthatóan tovább, a Hold-piramis lezárja azt,
de ,,méhében'' rejti a barlangot, vagyis a Tejúthasadékot, melyen át a
lelkek átléphetnek a felső világba.

Napútként is felfogható a város Halottak útjának nevezett főútja,
falloszként, mely a Hold-piramisba ,,ütközik''. A fogantatáshoz a Nap
sugaraira van szükség, a sugarak helyezik a lelket az anyaméhbe.

Tegyünk egy kitérőt Tulába, a Yucatán-félszigetre. A toltékok (i.~sz.
800--1200) legendás királyának, Ce Acatl Tolpiltzin Quetzalcoatlnak
nevében a tollaskígyó szó utal pap-király voltára. Gonosz hadisten
fivére, Tezcatlipoca elűzte, így a ,,jó'' király új várost alapított, Chichén
Itzát. A yucatáni dokumentumok leírják a nagytudású személy érkezését,
aki elhozza a tudást, főként az orvoslást és a művészeteket,
Chichén Itzába.

Úgy gondolom, hogy a Harcosok templomának tetején álló kígyófejű
emberalak őt takarja, és kozmológiai jelentése az anyagi világ
tetején, a piramis csúcsán álló lény közvetlen kapcsolata a felső világgal.
A pap-király a kígyó nyitott száján át, vagyis a Tejúthasadékon
keresztül hozza le a gyógyító energiákat, ő az, aki el is mondja a
magas üzeneteket. A Tollaskígyó kitátott szája nem fenyegető, mint
a kutatók gondolják, hanem szakrális üzenetekkel telítődik. A piramis
kilenc szintje jelképezi, hogy az embernek kilenc rétegben kell
pszichéjében és szellemében fénybe borulnia ahhoz, hogy sorsa kötéseit
feloldja.

Nézzünk egy további példát a Tollaskígyóról. Xochialcóban, a naptárkészítés
központjának Akropoliszán a Tollaskígyó kitátott szájával
és nyelvével néz az ég felé. Az épület lépcsőfokai mellett végigfut a
pikkelyes díszítés az anyagi, lenti világot szimbolizálva. Az épület tetején
körbeérő frízen kagylómotívumok díszlenek, vagyis felértünk a felső
szférákba. Párhuzam az indonéziai Borobudur buddhista sztúpája,
mely csigavonalban vezeti a látogatót lentről, a Vágyak világából fel a
Nirvánáig.

És egy újabb példa, Palenquében, a Pakal király szarkofágját fedő
kőlapon a Tejutat szimbolizáló kozmikus fa közepén egy kétfejű kígyó,
Tollaskígyó fekszik keresztben, az ő segítségével születik újjá a
király, akit a fa tetején már madár-kígyó alakban jeleznek.

Elgondolkodtató egy tikali piramis elnevezése, melynek neve Elveszett
Világ. Lehetséges, hogy az elveszett fényhegyre (csillagokra, égboltra)
való emlékezést, a felső világokkal összefűző kapcsolattól való
eltávolodást jelenti.

\bigskip
\begin{itshape}
Mert ha kapcsolatban is állunk az égiekkel, kérdés, hogy emberként a
tudást gyógyításra, a természettel való együttes megújulásra és fejlődésre
használjuk, vagy hódítunk vele, és a hatalom próbatételét nem álljuk ki.
Felsőbbrendűnek tekintjük magunkat, és alázat gyakorlása helyett emberéleteket
követelünk. A fény útjára vezetés helyett félelemben tartjuk a lelkeket,
megalázzuk a népet a hatalom és anyagi előnyök élvezetéért.
\end{itshape}
\bigskip

A Tejút a kiút lehetősége, a minőségi felemelkedésé. Az úgy, mint
fent, úgy lent elvén életutunk is a lélek, lelkünk útja, ahol lehetőségünk
van magunkban rendet teremteni, eloszlatni a sötétséget, vagyis
a káoszt.

A káoszt jelképezi a maja dzonot, amelynek spanyol megfelelője a
cenote. Mély, kerek kút vagy vízgyűjtő. E fekete vizet nevezték az
alvilágba vezető útnak, mert rend helyett a káosz oda vezet. A maják
megkülönböztették az áttetsző folyadékot és a fekete vizet vagy lyukat.
Nagy jelentőséget tulajdonítottak a naptárkészítésnek, hiszen segítségével
úrrá tudtak lenni a káoszon, le tudták írni az állandóságot.
Az időt isteni eredetű szubsztanciának tartották, így évszázadokon át
az egyik legfontosabb kiváltságos tevékenység, a naptárkészítés szent
tevékenységnek számított.

\section{Oaxaca}

Az oaxacai sajtból hosszú, vékonyabb, vastagabb, ízletes szálakat
téphetünk le a falatozáshoz. Nem nehéz kitalálni, a város a sajtjáról
nevezetes. A közeli völgyek kaktuszerdejével ellentétben buja
növényzetű, virágokkal teli, a gyarmati idők hangulatos városa. Ma
is zapoték-mixték bennszülöttek lakják. Épületei gyarmati villák
loggiás teraszokkal, árkádokkal. A Zócalo felé haladva az egyik
sarkon irodát találunk egy nagy papírlapra írt \textit{Escritorio publica}
felirattal. Az utcára nyitott pulton a helyi vállalkozó vagy köztisztviselő
az írástudatlanoknak nyújt segítséget az ügyfelek hivatalos iratainak
kitöltésével, vagy éppen távoli rokonoknak szánt és tollba mondott
levelek megírásával.

A főtérre kanyarodva egy tíz-egynéhány év körüli indián lány
tangóharmonikával kísért dallal fogad, a tér körül a fejükön széles
kosarakból virágcsokrokat áruló bennszülött señoriták lengetik bő
szoknyájukat. Mire esteledik, a Zócalo élete majálissá alakul át.
A tér egyik sarkán színes léggömböket kínálnak, és érkezőben van
a rezesbanda klarinétosokkal. Úgy látjuk, virágfesztivál készülődik,
mert a zenészek mögött a fonott kosarakban tartott, lelógó feliratos
szalagokkal díszesen összeállított csokrokkal vonulnak körbe-körbe
férfiak és nők.

\illustration{page-052}{A Santo Domingo-templom Oaxacában}

A kavalkádot kikerülve sétálunk a XVI. században épült Santo
Domingo-templomhoz. Mexikó legszebbik aranytemplomának tartják.
Belső aranydíszítése a burjánzó barokk mintapéldája. Körben az
oldalfalak és a mennyezet is teljes aranyborítást kapott, a kazettás
felületeken pápák, egyházi személyek, szentek képmása látható.

A templomból, mint egy kiadós étkezés után, elnehezülve, jóllakottan
jövünk ki. Súlyos a feladat, hogy ilyen miliőben emeljük lelkünket
az ég felé. Profán megjegyzésem meglehet illetlen és igazságtalan,
mert a templom lenyűgöző. Díszkáposztára hasonlító óriási kövirózsák
rózsaszínlenek a templom homlokzatának kivilágításában. Jól
kivehetőek a szintén rózsaszínnek látszó talajon, nektárjukat kolibrik
szívják. A madarak légies megjelenése, propellergyorsaságú szárnycsapkodásuk
jól ellenpontozza a hátteret.

A színes házak az esti fényben olyan tarkaságot mutatnak, mint az
élénk színű szőttesek a piacon. Az előttük sorakozó pálmafák a sötétedésben
ráfestik fekete lenyomatukat a házfalakra. Egy kisfiú színes
karszalagokat árul. A babona szerint aki felköti ránk a karszalagot,
azzal örök barátságot kötünk, de ehhez leszakadásáig a csuklónkon
kell hagyni. A fiút gondolatban örökbe fogadom. Még egy gyereket
összeszedek útközben, Chamulában, a piaci banánárus fiút. Kettejük
szemvillanását ma is magamban hordozom.

A Zócalóra visszatérve egy zenekar szórakoztatja a közönséget.
A nagy marimbán, mely olyan, mint egy óriási xilofon, hárman is
játszanak egyszerre, a kisebbiken ketten pörgetik az ütőket. További
hangszerek a gitár és a dobok, valamint két kis hengeres fa, melyek
ütemes összeütése tisztán kihallatszik a dallamok mögül. Szívmelengetően
merengő zene, melyre lassú léptekkel, párunk szemébe nézve
lejthetünk a kockaköveken. Pár utcás kitérő után ismét visszatérünk
a főtérre, ahol két mariachizenész vette át a stafétabotot a marimbásoktól.
A fehérre festett fémpadokon, a terebélyes indiai babérfák alatt
hallgatjuk a megrendelésre felhangzó szerenádokat.

A hotel medencéjénél ülve még meghámozunk pár szem narancsot,
majd andalogva kívánunk jó éjt a langyos levegőben.

\section{Monte Albán}

Az olmék befolyást tükröző zapoték piramisváros számomra a legszebb.
A legszebb, mert fényhatásai finomak, selymesek. A területre
mintha aranyport szórna az égbolt, amitől a föld, a fű mézesen csillogó.
A Monte Albánt távolról körülvevő hegyek párásan kékbe hajlanak.

Értelmezésem szerint a mézben rejlő kettősségről szól ez a terület,
a termékenységről és a halálról, a darazsak kettős természetével
szimbolizálva. A mezoamerikai indiánok Nap-Vénusz párosa, a Darázscsillag
bírt e kettős természettel. A Nap-Vénusz párost jelképező
Darázscsillag maja neve Sus~ek.

A nyílt terepen eltávolodunk mindennemű fedezéktől a leghátsó,
déli Hold-piramisig. S mintha a zapotékok szelleme térne vissza,
megjelenik Sus~ek fülsértő zümmögéssel, egy több száz vadméhből
álló raj a fejünk fölött. Vezetőnk halkan int, hogy ne mozduljunk, és
csend legyen. Végtelenül hosszú percekig állunk dermedten a ,,Darázscsillag''
alatt marsikus, háborús indulójukat hallgatva. Megnyilvánul
a méhek kettős természete, mely mézet termelni, gyógyítani és
ölni is képes.

A Monte Albán-i Vénuszhoz háborús jelek kapcsolódtak, melyek
a termékenységet szolgálták halálos áldozatokkal. Az előttünk magasodó
nyílhegy vagy J betű formájú csillagvizsgáló alakja előtt értetlenül
állnak a régészek. Véleményem szerint az épület ,,furcsa'' alakjába
szimbólumot rejtettek a zapotékok, mégpedig a méh mint marsikus
állat asztrológiai szimbólumát, a Mars jelét, az \male-t ábrázolja, illetve
a méhek szúró fullánkját takarja a formája. Emellett jelképezheti a
tekintetet is, és a megnyílást, hogy merrefelé kell fordulnunk, merrefelé
kell megnyílnunk, amit a Rák csillagkép egyik csillaga, az Altarf
szimbolizál. Az épület nyílhegye pontosan a Szekeres (Auriga) csillagkép
legfényesebb csillaga, a Kapella felé mutat. Egy csavarral azt is
feltételezhetjük az ,,iránymutatás'' miatt, hogy a csillagvizsgáló a Kos
csillagképet szimbolizálja, hiszen annak feje az égen pontosan a Kapellára
néz. A Kos jel bolygója pedig a Mars. A Kapella csillag sárga színével
kiemelkedik más csillagok milliárdjai közül, s a ,,megszólításra''
válaszul a fényét sugározza Monte Albánra, amitől az aranysárgává válik.

A Kos szemével a jövőbe tudunk tekinteni, a Szekeres Tejútkapun
keresztül pedig onnan, ahol vagyunk (csillagvizsgáló), ki tudunk emelkedni
az Istenek útjára. A zapotékok szándéka tehát a fölemelkedés volt
isteneik közé. Ezt jelzi vissza a megvételre felkínált zöld indiánmaszk is.
A maszk homlokán a harmadik szem, a spirituális látás helyén egy fekvő
kutya éberen pihen. A Kapella mellett a fehér fényű Szíriusz csillag is
Tejútoszlop a négy tartóoszlop csillagai közül. A Szíriusz a Nagy Kutya
csillagképben az alvilág kapuját őrzi, ahol fehér színével figyelmezteti a
belépőt, hogy csak a tiszta lélek üdvözül a kapun túl.

A vadméhek végül megkönyörülnek rajtunk, és egyszer csak elnyújtott
S alakban továbbrajzanak. Antropológus vezetőnk, Eduardo
komolyan adja tudtunkra, hogy az imént pontosan fele-fele esélye volt
annak, hogy a méhek vagy továbbrepülnek, vagy halálos csapást mérnek
ránk csípéseikkel.

\illustration{page-055}{A Labdajáték tér Monte Albánban}

\bigskip
\begin{itshape}
A látottakból azt valósítom meg, amit számomra jelent, azzal, hogy
átszűröm magamon. Értelmezésemből teremtem a magam valóságát.
\end{itshape}
\bigskip

Ez nem új keletű felismerés, erre épít a terapeuta. Ha a terapeuta
vagy a pap azonos a királlyal, vagy szövetségesek, megszületik az
értelmezésből a mítosz.

A zapotékok idejében a Rák csillagképben volt a nyári napforduló.
A Rák a Hold változásának, a teremtés szent hármasságának a szimbóluma,
az újholdé, a növekvő holdé és a teliholdé. Más megfogalmazásban
a befogadó, a tápláló és az elengedő szeretete. A zapotékok
Darázscsillaga, mai nevén a Méhkas csillagkép a táplálás szimbóluma,
így tehát a Rák csillagképben látható. Sus~ek harcias szerelemistennő,
a Vénusz rosszabbik énje, az esthajnalcsillag alvilági istent is jelent.
A Rák csillagképben van az alvilág egyik bejárata. A zapotékok emberi
szíveket áldoztak, hogy táplálják istenüket, azért, hogy istenük is
gondoskodjék népük táplálásáról. Virágok háborújának hívták az áldozati
foglyokért kiszámított időpontokban indított csatákat, hogy amikor
eljön az áldozat naptári ideje, a termés a rituálék segítségével biztosítva
legyen. A zapotékok által első hallásra visszásan, de mégis zseniálisan
megfogalmazott Virágok háborúja elnevezésben is benne foglaltatik a
termékenység, hiszen a virág Vénusz-szimbólum (\female), a háború pedig
Mars (\male), s kettejük nászából születik az új élet.

Roman Polanski \textit{Keserű méz} című filmje láttatja művészien a
szerelem magasba röpítő és mélybe taszító végleteit. A film két
szerelmespárja Sus~ek modern kori leképzése. Hamvas Béla írja:
%\footnote{%
%Hamvas Béla: \textit{Patmosz}, Medio Kiadói Kft., Budapest, 2004}:
,,\dots a sötét pont
és a méz egy. Az élet centrumában a halál, és a halál centrumában az
élet, és a kettő egy.''

Az i.~e. 500-ból való, Táncosok Templomának elnevezett épület kőküszöbe
egy hadifoglyot ábrázol, akinek a szívét kitépték (feltehetőleg, hogy
ne tudjon kapcsolatba lépni a szellemvilággal, illetve feláldozták az
istenségnek). Kitépett szívű és nemi szervüktől megfosztott foglyok csaknem
másfél száz ábráját találjuk Monte Albán legősibb épületén. Amennyiben
helyes a feltételezés, és az ábrák valóban táncoló alakokat ábrázolnak,
akkor ugyancsak egy, a Rák csillagkép szimbolikájának szóló rituális
táncról van szó. A foglyok ilyetén való megalázása azonban ellentétes a
Rák üzenetével. Ezen csillagkép alatt látható, annak alapja a Vízikígyó
lehajtott feje, mely a belső alázat megszületését jelképezi a magas
szellemiséggel szemben, és az nem szólhat mások megalázásáról. Az önfeláldozó
szeretet csillagképe nem egyenlő az öncsonkítással, megsebzéssel, mely
áldozathozatalt oly sűrűn követték itt el. Egy korábbi felfogás szerint a
templomfalakat díszítő képek különböző pozitúrákban szülő nőket ábrázolnak.
Ez a feltevés is a Rák csillagképhez kötődik, hiszen ezen csillaghalmaz
az anyaság jelképe. A Rák csillagképben látható a Jászol (születés
helye), az Olló (elengedés -- köldökzsinór elvágása), valamint az Északi és
Déli Szamár, a táplálás, és az Altarf, a tekintet és megnyílás csillagképei.

Itt lesz jelentősége annak, hogy hová néz a Monte Albán-i csillagvizsgáló
hegye. Csak nézek, vagy látok is a szememen túl a szívemmel?
Merre fordul a tekintetünk, mi születik meg bensőnk szülőszobájában,
s azt hogyan tápláljuk? Kitépett emberi szívek kellenek istenünk
,,táplálásához'', vagy be tudjuk engedni magunkba, a saját szívünkbe a
fényt, és ez által megtermékenyülni? Milyen ,,véletlen'' egybeesés, hogy
a csillagvizsgáló tájolása szerint arra a csillagra néz, a Kapellára, mely
ezt a fent leírt szellemi táplálást és termékenységet közvetíti felénk.

\bigskip
\begin{itshape}
Figyelünk-e még arra egyáltalán, hogy miféle szellemi vagy testi táplálékot
veszünk magunkhoz nap mint nap? Csak nézünk, vagy gondolkodunk
is a látottakon, olvasottakon, amelyeket felénk közvetítenek nagy fényerővel?
Észre tudjuk-e még venni a vakító, elvakító neonfények mögött az élő
fények ,,tájékoztatását'' is?

Mennyire él még az az étel, amelyet megeszünk, vagy folyamatosan az
élettelen üzenetét kódoljuk testünk felé, amitől egyszer csak megjelennek a
lázadók, s építő sejtjeink megadják magukat a nagy étvágyú, de
tiszavirág-királyságú hatalmaknak?

Gombnyomásra és műanyag táplálékkal a robotok működnek jól, és tökéletesebben,
mint ugyanezzel a vezérléssel az ember. Lehet, hogy leköröznek
nemsokára, hisz ők születésüknek megfelelő ellátást kapnak. Míg az ember
csak egyre tökéletlenebb lesz ezen a gyorskonyhán, elkorcsosul, fényét veszti,
mint koncentrációs táborokban nevelt állatai és megbolondított növényei.
\end{itshape}

%\clearpage
\textit{Mit választasz?}
\bigskip

\newcolumntype{L}{>{\otherfamily\setlength{\parskip}{0.5\baselineskip}}X}
\noindent
\begin{tabularx}{\textwidth}{@{}LL@{}}
\textbf{Az érett őszibarack (emlékül az utókornak)} &
\textbf{Paradicsom-pró\-ba\-vá\-sár\-lás} \\
\addlinespace

Az érett őszibarackot a fa felém
hajolva nyújtja,

Megszólít a gyümölcs rám kacsintva,
s kezemet húzza.

A nap sugarától meleg bőre puhán
huppan tenyerembe,

S a csap alatt szakadozva meztelen
testét mutogatja.

Illata kibomlik, nedvei kezemben
megindulnak, ujjaim közt szétomlik
húsa.

Édes virágok harmatfelhője fölé
hajolok,

Mely ígéretből nektárzamatú
rostok ölelése válik,

S eltelít, mint szeretőm belém
feledkezése.
&

A paradicsom lám óriás, de kemény,
mint a beton,

Még pirosas is, hogy teljes lehessen
illúzióm.

Idősebb társai a láda végén
beteges repedten, könny nélkül
múlnak ki.

Serceg a kés, miközben felvágom,

Mintha előre szégyellné sápadt,
lázas belsejét előttem.

Olajjal, fűszerekkel javítok
állapotán, s mögé képzelem Bács-Kiskun
egész földi paradicsomát.

Még létezik a földnek e klimatikus
szeglete, ahol a természet tán
még természetes. Természetes?
\end{tabularx}

\clearpage
Visszatérve utazásunk helyszínére, látszólag ismét ugrunk egy
nagyot. Ha Monte Albánt az ókori Görögországba helyezzük át,
Artemisz méhistennő szentháromságának szentélyében találjuk magunkat.
A két, egymástól földrajzilag távoli hely ugyanazzal a szimbolikával
hat, csak a megnevezésekben van különbség. A szavak sok
részre szakadtak, de megmaradt a világ népeinek egy közös nyelve,
melyet szimbólumaiban őriz, elrejtve népmeséiben és népművészetében,
költészetében, építészetében és egyéb ábrázolásaiban.

Sus~ek téged is szólít, Nimród, s népedet, a magyarokat:

\bigskip
\begin{itshape}
A szavakkal gyógyíts, s teremts, és ne ölj velük, vadász, nyilaid magasabb
szférák felé irányítsd. Ezt zizegték fölöttünk a vadméhek -- zim-zi-zim.
%\footnote{Weöres Sándor egyik gyermekverskötetének a címe.}.
\end{itshape}
\bigskip

Elnyújtott S alakban továbbtartottak Artemisz jelét írva az égre,
mely az S betű. Így néztek hátra még egyszer fejükkel a múltat
jelképező Rák csillagkép piramisára.

\bigskip
\begin{itshape}
Térj vissza a kezdetekhez, ősapáid hagyatékát vizsgáld, s arra alapozz.
Nézd végig, merre jártál, s mit tettél a félköríven mostanáig, mert
fordulóponthoz értél. Lelked erkölcsi tisztasága uralkodásra teremt,
a múltból táplálkozó bölcsességed világfordítóvá tesz, ha a fényt
a kezdeteknél keresed.
\end{itshape}
\bigskip

Ezzel a jelentéstartalommal már továbblépünk az Oroszlán csillagképbe.
Az égi Oroszlán hátraforduló feje szimbolizálja a fentieket.

\bigskip
\begin{itshape}
Így menekülhetsz meg és élhetsz tovább, vén Európa, és élhetnek tovább
népeid.
\end{itshape}
\bigskip

Továbbutazunk \textbf{Tulé}be, az időszámításunk kezdetekor ültetett
óriás ciprusfához. A hatalmas fa körfogata 36 méter, súlyát 450 tonnára
becsülik, és 42 méter magas. A képeslapokra úgy fényképezik,
hogy előtte felsorakozik egy mariachi-zenekar, mutatva, hogy a ciprus
még náluk is szélesebb.

A yaguli feltárások 2000-ben, amikor ott jártunk, még a kezdeten
álltak. Hőség tombolt, és az óriás kaktuszok, jukkák nem adtak hűsítő
árnyékot. Így gyorsan továbbhaladtunk a közeli, 5000 éves felszíni
sziklamélyedésben látható ősemberrajzhoz, ahol településnyomokat is
találtak. Teljesen kiépítetlen az ösvény, amely odavisz, a magasra nőtt
gazban és szúrós kaktuszok között csak félútig megyek, már onnan is
jól látható a pálcikaember alakja.

Mindenütt, ahol szőnyegkészítéssel foglalkoznak, és bármilyen
sűrűséggel elvetődnek arra turisták, elviszik őket egy-egy házi üzembe.
Nem unalmas a nézelődés, mert mindenütt más a technika, nem
beszélve a színekről és a mintákról. A különböző színeket növényi
nedvekből és szárított bogarakból nyerik. A családi ház teraszán,
a cserepekben hófehér élősködőkkel teli kaktuszok állnak. Ha sodrófával
összepréseljük a kaktuszról leválasztott bogarakat, megkapjuk a
sötétvöröset. Amennyiben a vízben oldott vöröshöz citromlevet csöpögtetünk,
kinyertük a narancssárgát. Fekete növényekből származik
az azúrkék szín. A kézi szövésű szőnyegek készítésénél a szövőszéket
Európában lábbal hajtják, Mexikóban kézzel.

A kis, tizenkét személyes furgon, mellyel egész nap közlekedünk,
\textbf{Mitla} piacára visz bennünket. A földön ülve, terítőkön árulják
a zapoték, mixték leszármazottak áruikat. Mindenből csak keveset, a
házikertek terméseit, a háztáji élelmiszereket és feldolgozott termékeit.
Megakad a szemem egy szép szalmakalapon, különleges a kalap fonása
és dekoratív a szalmából készült virágcsokor a karima fölött. Meg is
veszem a hasznos portékát.

A zapoték nyelvű Mitla város nevének jelentését sokféle fordításban
hallottam. Néhány közülük: a pihenés helye, hercegek/királyok lakhelye,
a halál helye. A romterület felé tartunk, melyet a város utcájától
kaktuszkerítés választ el. Túratársam humorosan meg is jegyzi: Mitla, ahol
nem kolbászból, hanem kaktuszból van a kerítés. A magasabb és alacsonyabb
kaktuszok sorát kötéllel is egymáshoz erősítették, de biztosan nem abból
a célból, nehogy valaki a két szorosan összenőtt tüskés kaktusz között át
merjen mászni. Az épületek formavilága és díszítése kifejezetten hellén
minta. Vallásos szimbólumrendszer jelenik meg a falakon, a rajzolatok
jelentését kozmikus esőnek, kozmikus sugárzásoknak magyarázzák.

Késő délutánra jár, a vacsora előtt azonban még vár ránk egy ínyencség
a pálinkafőzdében. A családi vállalkozásban működő kiskocsma
udvarán a pálinkát kaktuszból készítik, meszkál a neve. Kis tálban
szárított kukacok várnak azokra, akik megadják a módját a kóstolásnak.
A vállalkozó szelleműek elmondása szerint a kukac íze leginkább a
mogyoróéra emlékeztet. Látványra nem rosszabb, mint azon kínai
szeszeké, melyekben egész kígyótestek úsznak, megadva ezzel az ital bukéját.

Snapsz nélkül is mélyen alszom a kimerültségtől és a változatos
élményektől.

Az új nap reggelén a \textbf{Cañon del Sumidero} bejáratánál odvas
facsonkon ülő keselyűk fogadnak, amikor motorcsónakon berobogunk
a zavaros vizű Grijalva folyón. A kanyonba való behajózás fészkelődéssel
indul. A csónak egyensúlyának megtartása érdekében menet
közben még kilók szerinti helycserék történnek. Most nincs sok vita
az ülésrend miatt, tekintettel a parton napozó krokodilokra, melyek
a vízben is sűrűn előfordulnak. A csónaktúrát örvények és kis vízesések
nehezítik a tizenöt kilométer hosszú folyón. Az első fehér ember
vezette kenu 1960-ban tudott rajta keresztülhajózni. Előrehaladva a
víz olívazölddé színesedik, a sűrű növénytakaróval borított, meredek
sziklafalak fölöttünk 900 méteres magasságba nyúlnak. Vízinövények
úsznak a vízfelszínen, rajtuk kócsagok, daruk, pelikánok nyújtogatják
nyakukat. Sajnos a ,,civilizációnak'' elég volt pár év, hogy a nyíló vízi
virágok mellett, a part mentén megjelenjenek a műanyag palackok,
nejlonzacskók soha el nem bomló füzérei.

Inkább felfelé nézünk, a nagy vízesésre. Egészen fentről bújik ki a
sziklarepedésből a vízsugár, bukfencezik egyet-egyet az útjába kerülő
növényeken, mohás kiugró szikladarabokon, és egyre lejjebb szállva
vízpárává terül szét. Az első spanyol invázió után a chiapa törzs nagy
része a legenda szerint inkább a kanyonba vetette magát, mint hogy
vállalja a rabszolgasorsot. Aki észreveszi, annak még ma is megörökíthető
egy indián szelleme a sziklafalon. A lezuhogó víz oldalról nézve
egy indián arc profilját formázza. A hazahozott fényképen is tisztán
kivehető sasorra, szeme, szája és tollkoronája.

A folyó mellett ebédelünk halat s mi jó falat, ami a tortilla. Egy
kisfiú balladát énekel nekünk ebéd közben. Kár, hogy semmit nem
értünk a helyi indián nyelvből, de az előadás lényege, népe szomorú
sorsának elbeszélése így is világos számunkra.

Utazunk tovább a következő célállomás, a több ezer méter magas
vulkáni hegyekkel körülvett \textbf{San Cristóbal de las Casas} felé.
Színes, legfeljebb egyemeletes házai között nagyon régen megállt az idő.
Egy lakóháznak tűnő hotelben kapunk szállást. Fellépcsőzünk a bőröndökkel,
mely gyakorlat lassan már meg se kottyan. Edzésbe lehet jönni
a helyi járatú buszok le- és felszállásai és a szálláskeresések közben.
A ház emelete nyitott átrium jellegű, ez az építési mód egész Mexikóra
jellemző. A folyosóról nyílnak a puritán, tiszta szobák. A lépcsőlejárónál,
a falban egy vízcsap medencével a közös mosdó, mellette
fából épített zuhanyozók állnak. Lelátni a földszinti étkezdébe és a
kalitkájában ücsörgő nagy papagájra. A papagáj rövid idő alatt megtanul
magyarul köszönni, de gyorsan átvált ,,anyanyelvére'', a spanyolra.
A reggeli a megszokott kiadós babpüré és tojás tortillával. Utunk
során napközben legtöbbször kis ,,lacikonyhákban'' étkezünk, ahol
forró vaslapon melegszendvicsek készülnek. A darabokra vágott húst,
mely általában csirke, a zsemlét vagy tortillát is megsütik külön-külön.
A szendvics tartalma a hús, a zöldségből kiskanállal kifordított
avokádókrém, paradicsom, savanyúságok és szószok. Sajnálatomra Mexikó
nem a sütemények országa. Linzerekkel vigasztalódom, és a katedrális
homlokzatával, mely úgy néz ki, mint egy cukros sütemény. A napsárga
falat sötétnarancs és vörös frízekkel bontották több részre, majd
az egészet befedték fehér cukormázból szőtt csipketerítővel.

\fullpageillustration{page-063}{%
	A katedrális San Cristóbal de las Casasban}

A vásártérre kora reggel érkeznek a környező falvakból az árusok
kis mennyiségű, saját termesztésű gyümölccsel, zöldséggel. Ezen a
piacon még nem működik a nagykereskedelem. Harsányan virítanak a
színes virágos szőttesek. Nagy sárga napraforgókkal és fehér kálával
teli püspöklila, zöld, piros, kék terítőket bontanak szét az asszonyok.
Hosszú fekete hajukat két copfban fonják össze, sötét, bő szoknyában
ülnek a köveken vagy a magukkal hozott kis székeken.

Dzsippel feljebb megyünk, a kétezer méter feletti fennsíkra, \textbf{%
Chamula} faluba. A lakosok zöme itt nem beszél spanyolul. A nők maguk
készítette vastag, sötét lapszoknyát csavarnak a derekukra, melyet egy
színes sállal erősítenek meg. A temetőnél szállunk ki, ahol egy fiatal
lány hosszan néz a türkizkék és szürke keresztfák sűrűjébe. Odébb
két nő hátára erősített kancsóval indul a hegyoldalnak, a forráshoz,
vízért. Az asszonyok a kertkapuk előtt szoptatják gyermekeiket. Fáradt
és rejtőzködő arcok. A felnőttek tiltakoznak a fényképezés ellen.
Vendégségben pedig illik tiszteletben tartani a házigazdák kérését. Az
önzés persze nem ismeri ezen szabályokat, így egy-két rámenős turistatársam
kődobálásban részesül,

Chamula temploma a felirat szerint 1514-ben épült. Külső falai
fehérre meszeltek, a bejárata körüli élénkzöld frízen virágos sorminta
virít. A közösség megszüntette kapcsolatát Rómával azért, hogy
szabadon gyakorolhassa ősi hitét. A templombelső sajátos keveréke a
katolikus és az ősi hitnek. A szentek szobrait színes ruhákba öltöztetik,
bőrszínük sötétbarna. A templom padlózatát csomókban hosszú
fűszálak borítják, köztük gyertyák égnek. Az oltár előtt ül ezen
a gyepen a falu spirituális vezetője. A falubeliek sorban állnak hozzá.
Odaérkezve egy anya két-három év körüli kisfiát nyújtja felé, a fiú
eszméletlen. A vezetőnő megvizsgálja a fiút. Szegény asszonyt körbefonja
a bánat, mialatt megrendülten áll és vár, hallgatja a jósnőt. El kell
telnie pár percnek, míg a csöndből ki tud mozdulni. Orvos útitársam
lehajol mellettem, és ő is megvizsgálja a gyereket, aki -- mint kiderül
-- már halott. Az anya felemeli gyermekét, belecsavarja takarójába, hátára
veszi, és lassan, a szívét nyomó hatalmas súlyoktól nehéz léptekkel
elindul a templomból kifelé. Felsejlik előttem az indián háborúk és a
hódítók által okozott veszteséglista, melyet ennek a népnek el kellett
szenvednie. Hány anya léphetett át a harcmezei halottakon sajátjait
keresve, hány nehéz lépést elléptek már ezekben a völgyekben?

Kívánom minden népnek, hogy sasmadara újjászületve magasan
szálljon, és a nap közelében fénnyel telítődve sugározza be a földet a jövő
védelmében. Teotihuacán főútja, mint népének országútja új keresztséget
nyerjen a Teremtésben, halottaik pedig nyugodjanak békében.

\illustration{page-064}{%
	Fűárusok -- a templomok padlóját terítik be ezzel a hosszú
	és vékony szálú fűvel}

Az olmék-azték Teotihuacán, az olmék-zapoték Monte Albán és
a mixték Mitla után megérkezünk az első maja városba, \textbf{Palenqué}be.
Szubtrópusi a táj, meleg, páradús a levegő, az őserdőből kiszabadítva
gyönyörű zöld füves, fás terület, melynek jelentős részét ma is erdő
bújtatja el. Élénkzöld színéről kapta a Smaragdváros elnevezést. Belepi
a dzsungel, liánok fonják karjaikat az ősi kövekre, és sűrű aljnövényzet
borul rá. Az épületek alig tíz százalékát szabadították
ki a növénytakaróból, de a láthatóvá lett egykori városrészlet a
legépebben megmaradt ősi városok egyikét mutatja. Palenque spanyol
név, a romok mellett várost alapító dominikánus szerzetesek adták a
helységnek. További öt neve van, melyek 1. az a hely, ahol sok fa van,
2. a kos feje, 3. a kígyó városa (ez a hivatalos neve), 4. a kövek és
5. a vizek városa.

Fénykorát II.~Pakál (II.~Pajzs) király idejében élte, a VII. században.
A feltárt emlékek nagy többsége az ő, illetve fia uralkodása alatt
épült. Palenquében találkozunk először az uralkodó cselekedeteit
megörökítő domborművekkel az épületek falain, oszlopain, stukkóin.

\illustration{page-065}{A Kereszt temploma a palenquei maja romvárosban}

Pakál király sírja, a Feliratok temploma ugyanazt a misztériumot
őrzi, mint az egyiptomi piramisok. A templom padlózatából indul a
föld alatti lépcsősor több kanyarulattal, a túlvilágot jelképező kilenc
szinttel a sírkamrához. A király kőszarkofágját helyezték el először,
majd köré és fölé építették a templomot. A szarkofág tetején látható
rajz az egyik leghíresebb és legtöbbet tanulmányozott maja ábrázolás,
égitestek jeleinek sokasága.

Meghökkentő a csillagvizsgáló melletti épület egyik oldalfalán
a kínai sárkány rajza is. Meglehet, hogy már ezekben az időkben is
tartottak fenn kapcsolatokat egymással a föld népei, vagy működött
a kollektív tudattalan. Agyműtéteket, fogműtéteket végeztek, de ez a
tudás mind elveszett, a naptár jelentése is. Mindez istentől származott,
mondja maja vezetőnk.

\illustration{page-066}{%
	A négyzet alapú torony, a palenquei romváros központi épületegyüttesének,
	a Palotának a része}

A koronázásokat különböző égi jelenségekhez kötötték, melyek
meghatározták az uralkodó ,,pályáját''. A koronázási ünnep hasonló
volt a magyar szakrális királyokéhoz, ahol a rendet, egységet többek
között az is képviselte, hogy minden nemes részt vett a beiktatáson.
A fejedelemségek számára a piramis alapú rendi felépítés megbillenését
jelentette volna egy hiányzó láncszem. Hunyadi Mátyás el is halasztotta
első koronázási ünnepségét a hiányos részvétel miatt, illetve
a csillagok állása szerint határozta meg a ceremónia időpontját. Eleink,
illetve az Óperencián túli maja vezetők trónra, vagyis magaslatra
lépése, a koronázás egy beavató szertartás volt.

Palenque szembeötlő négyemeletes, négyzet alapú tornya a csillagvizsgáló.
Tetején a Vénusz jele látható. A maja birodalom az alvilággal kapcsolta
össze magát, ezt jelzik a nyugati (és nem keleti) oldalon elhelyezkedő
Naptemplom domborművei. Magyar elődeink haláluk után csatlakoztak
az alvilághoz, melyet a sírban nyugat felé nyugodó fejükkel jeleztek.

\section{Agua Azul}

Azúrkék moraja, az őserdőben, a sziklákon lerobogó víz egyenletes
zaja megnyugtató. Ahol kis ér csurog le, ott a víz szétterül, és fátyolba
öltözteti a köveket. A szélesebb vízfolyásnál, meredek szikláknál nagy
hanggal zuhog le, fehér tajtékkal. Fiatal mexikák nem törődnek a
\textit{Vízbe ugrálás életveszélyes} felirattal, hangosan kiáltanak
felénk, öt pesóért
ugranak. Látva, hogy nem fizetünk a mutatványért, nagyokat sikítozva
ingyen is beugrálnak a folyóba. Felkapaszkodunk a domboldalon, és
a sík területen, ahol a folyó csendesen halad előre, mi is megfürdünk.
A parton, a fák alatt hagyjuk a ruhánkat, és meztelen lábbal, óvatosan
belegázolunk a folyóba. A vízfolyással egy irányban haladok a sziklák
felé, nem bírom ki, hogy ne tekintsek le a zuhogó vízbe. A sodrás
egyre erősebb, de olyan csapáson sodródom, ahol a víz csak térdig ér.
A nagy kövekhez érve a szélükön megtámasztom a lábam, és fellépek
az ,,emelvényre''. Erről a kilátóról a hátam mögé nézve a vízesésbe
merül a tekintetem, előttem pedig társaim imbolygó járását figyelem.
A páradús levegőjű dzsungelben bikiniruhás sellőként napozom a
nagy sziklán, és arra várok, Tarzan dobjon felém egy liánt segítségül,
és megrövidítse a visszautat a szabadtéri öltözőig. Tarzannak azonban
éppen eszében sincs fára mászni, a mélyebb részeken karcsapásaitól
fröccsen az Agua Azul.

Egy kombikocsiban összezsúfolódva döcögünk vissza Palenquébe.
A raktérben elnyúlva rögzítem az összes bukkanót, amely hátsó felemre
íródik a rögös úton. Eltelik pár nap, amíg elhalványul e térkép,
s sajgó emléke lecsendesül.

Következő nap a mexikói-guatemalai határ menti \textbf{Frontera Corozal}ba
indulunk tovább. Megérkezésünkkor kihalt a falu, áll a levegő.
Gyerekek, kutyák lődörögnek a poros úton. A benzinkútnál kipakolunk
a dzsipekből, a fiúk innen néznek szét szállás után. A benzinkút
három nagy hordóból áll, felettük rozsdás tetőzetről lóg a felirat,
melyből megtudhatjuk, hogy egy töltőállomáson vagyunk. A hordókon
nincs egyéb jelzés, így számunkra a választékot homály fedi.

Papagájvirágok nyílnak a poros út mentén. Úgy virágzanak, mint
nálunk a napraforgótáblák. Egy belső lendülettel lépnék is a magas
fű közé, hogy elérjem őket, de megszólal a vészharang, kígyók tekeredhetnek
alattuk. Valójában a mérges kígyók jelenléte nem az aljnövényzet
magasságától függ. Másnap Yaxchilánban, az őserdő avarán
bármelyik pillanatban elénk kúszhatott volna a sárga, szakállas kígyó.
Mérge három másodperc alatt végez az áldozattal. Mégis magabiztosabban
járok a lapos fű és az oldalra kúszó növények között. Kétes
biztonságot ad, hogy látom a cipőm orrát. Verem a port vele az úton,
s valamit már megérzek a Corozal közeli sűrű dzsungel atmoszférájából.
Itt még a természet az úr. Lassú légzésemben és lábam egyenletes
lépteiben ráérzek az egyetemes mozgás ritmusára, s szalmakalapom
alól a napba tekintve soha nem észlelt erővel áramlik át rajtam a fény.
Ismerős alakok jönnek velem szembe a melegtől reszkető levegőben.
Lépteikben a cowboyok csizmatalpának hangját hallom, ahogy besétálnak
a vadászmezőkről a kihaltnak tűnő város főterére, hogy aztán
pár másodperc múlva puskaropogások közepette rendet tegyenek a
lezüllött lakosok között.

Jó hírrel jönnek a férfiak, és egy mögöttük robogó dzsippel. Az
egyik veterán járműre feltesszük az útitáskákat, a másik nyitott platójára
pedig felszállunk, állva elférünk rajta mind a tizenketten. Ismét
erősödik a karom, amíg elérünk a faházakig. Sajátságos sorház a szállodánk.
Egymás mellé épített szobákból áll a hotel. Egy nagy, fadeszkás
épületbe négy-öt bejárati ajtó nyílik. Ám amikor belépünk a számunkra
kijelölt helyiségbe, megdörzsölöm a szemem, hogy tényleg jól
látom-e a válaszfalat, illetve nem látom. A szobák között a falécek nem
érnek fel a plafonig, jó 30-40 centiméterrel alatta véget érnek. Mire
leszáll az est, már fáj az oldalam a sok nevetéstől, a felkínált közösségi
lét e furcsa példáján. Azt is hallom, amit a tér tőlem legmesszebb lévő
sarkában beszélnek. Kissé udvariatlan találós játékot játszunk azzal,
hogy a különböző neszek a szobák mely részén zizzennek és milyen
tevékenységből származnak. Az ajtón nem merünk kimenni, mert egy
kutyafalka gyűlt össze a szállás előtt, és morgó jelenlétük nem biztos,
hogy a bejáratot őrzi, inkább a kijárás ellen való. A hoteltulajdonos
végül elhajtja az ebeket, és elindulhatunk vacsorázni. A helyi kricsmiben
nyugtalanul nézem Leonardo da Vinci \textit{Az utolsó vacsorá}jának
szőnyegbe szőtt másolatát. Végül azért jóllakottan, csoportosan gyalogolunk
vissza a faházig. A határfaluban nem mertünk egyedül kóborolni
az éjszakában. Fürdőszobánk a kert végében egy hordónyi langyos víz.
Egyetlen zseblámpánkkal ketten-hárman összeállva vonulunk hátra,
és próbáljuk kimérni a nekünk járó vízmennyiséget.

Frontera Corozal volt számomra az a hely, ahol Mexikó élő közepén
éreztem magam, ellentétben a múzeumokkal, ahol csupán az ország
határmezsgyéjén jártam.

Másnap reggel kenuval érjük el a Mexikót és Guatemalát elválasztó
Usumacinta folyón a dzsungelt, melyben még részben feltáratlanul rejtőzik
a zöld kövek maja városa, \textbf{Yaxchilán}. Egy romos épületegyüttes
denevérekkel zsúfolt, sötét folyosóján át érünk ki egy tisztásra. Az e
próbatételt nem vállaló látogatók jobb, ha tovább sem mennek. Vastagon
lefújtam magam moszkitóriasztó spray-vel, de a vérszívók így is
hemzsegnek körülöttünk. Mindenki folyamatosan csapkod maga körül.
A lábam alatt óriási vörös hangyák szaladgálnak, fel-feltévednek
a cipőmre is, így topogva lépdelek, nehogy továbbfussanak a lábamra.
Csípésük a méhekéhez hasonló, de nem dagad meg. Amiket azonban
a leginkább ki szeretnék kerülni, azok a kígyók. Yaxchilánba kevés
turista jut el. A csend és a dzsungel érintetlensége mégis minden veszélyt
megér. Filodendronok és egyéb trópusi növények kúsznak fel a fákra.
A megtisztított sztéléket a tisztáson helyezték el a kutatók, ponyvatetővel
védve őket az esőtől. Levélhordó hangyák dolgoznak mellettünk,
miközben mi összetört lépcsőfokokon gyalogolunk fel a dombtetőre.
Az apró állatok hosszú sorban hordják a hátukon a faleveleket, melyek
testük méretének többszörösei. Pár nap alatt képesek lekopaszítani
egy teljes lombkoronát. A hangyabolyban megerjesztik, és gombát termesztenek
rajta. Az ősi épületek belső falrajzolatain néhol még látható
az eredeti színezésből a vörös. Az út mentén összetört szobrok fekszenek,
és találok egy kő naptárat is a fák között. Egy virágzó pálmafáért
egy kicsit beljebb lépek az ösvényről a dzsungelbe, de gyorsan besötétedik,
így azonnal visszatérek a csapásra.

A dombtetőn megvárjuk, amíg a csoport visszaindul a folyóhoz
és elhagyja a romterületet. Ekkor kettesben az útitársammal a maja
hatalmi politika egyik főszereplőjének, II.~Jaguár Pajzs templomának
tetejére mászunk fel, ahol egykoron a nyári napfordulóhoz kötődő
szertartást végezték. Ő tigrisként vöröslik mellettem, s teljes biztonságot
ad jelenléte. Körbevesz bennünket a dzsungel, ahol a csend oly
mértékben sűrűsödik, hogy zuhanás érzése fog el, de ez a mélység
megtartó erejű. Láthatatlanul ébred az erdő, hangjai lélekig hatolnak.
Papagájrikácsolás hallik, majd ismét hosszú csönd áramlik felénk
energiával telve. Lassan induló morgás, az üvöltő majom hangja hasít
a fák között. Ereje és hatása végül oroszlánbőgést idéz. Egy mérges
bogár, lehet, hogy pók, mászik az ősi köveken -- mint utólag megtudom --,
melyet a védő hérosz lesöpör, és egy nagy madár emelkedik a
magasba. Szárnyait kiterjesztve, kutató tekintettel vitorlázik a titkokat
rejtő, életet adó, végtelen dzsungel fölött a folyó túloldalára, hogy
hazataláljon.

\bigskip
\begin{itshape}
Eltűnik az idő, lecsendesedtünk, középpontunkban, egy hihetetlenül kis
pontban ülünk, az egész világmindenségben. Talán pár másodpercre lebegünk
így a tér/idő keresztje felett. Ebben a súlytalan állapotban karjaink a
megfoghatatlan végtelen végéig érnek, mindent átölelve az univerzumban.
Nem tudom, hogy én ölelek vagy engem tartanak, mert nem érzékelek határokat,
egy vagyok, nincsenek különálló részek, a mindenség van bennem.
A forgatókönyvet a kezembe vehetem, és előre, hátra forgathatom életem
kerekét. A folyamat önmagába visszahajlik, és másképp húzom a húrokat.
Ez az igazi szabadság! Önmagad fölé emelkedni, határaidat feloldani és a
változást önmagadban előidézni. Nincsenek falak, csak úgy képzeled, hogy
vannak. Addig állnak előtted, míg tudatod látni véli őket.
\end{itshape}
\bigskip

Majd újraindul a film, susogni kezdenek a fák, és mi lekúszunk a
templomtetőről. Visszaballagunk a folyóhoz, hogy mielőtt lemegy
a nap, elérjük szállásunkat. A krokodilokkal teli víz újra próbatétel
több utas számára. A félelmen azonban úrrá lesz a bizonyítási vágy
magunknak és az otthoniaknak, amikor egy-egy, a parton pihenő hüllőt
megközelítünk a fotó kedvéért. A faluban sötétedés után rázendít
a trópusi eső. Mintha dézsából öntenék. Egyenletes, erős koppanásaira
alszom el. Reggel ébredezve a résnyire nyitott ajtóban látom, hogy
jobbról üres kézzel elhaladó helybeli lépdel, balról visszajőve két-három
méteres pálmaágakkal érkezik. Egyre éberebben figyelem, ahogy egymás
után fordul oda-vissza. Felkelek, és kijjebb lököm az ajtót. A falusiak
a tegnapi égszakadásban megsérült háztetőket javítják. A kutyák
ismét itt oldalognak a hotel falánál, rám morog az egyik sovány szuka.
E ,,jó reggelt''-re teljesen kinyílik a szemem. Összeszedelőzködünk és
lemegyünk a folyópartra. Dugig lesz a kenu a csomagjainkkal. Beoldalazunk
még mi is a hajóra, majd felbőg a motor, és egy időre elhagyjuk
Mexikó partjait. Erős örvények között kormányoz a hajótulajdonos,
cikcakkban lavírozunk. Nem mindenki bírja jókedvvel, amikor pördül
a kenu, és én sem bánnám, ha mielőbb kikötnénk Bethelben. Egy
kanyar után balra sodródunk, és rácsúszunk a sáros fövényre. Ez a
kikötő? -- hitetlenkedem, de nagyon úgy tűnik, átértünk \textbf{Guatemalá}ba.
Nem tudok dönteni, hogy melyik pocsolyába lépjek ki, de amikor azt
látom, hogy a bőröndömet is kitenni készül a jóember, gyorsan kiugrom.
Megfogom a csomagjaimat, és elindulok velük a csúszós úton;
összeszorított foggal kántálom magamban: mindenre szükségem van,
amit a bőröndbe pakoltam, mindenre szükségem van, amit \dots

Két-háromszáz méter után szárazföldet érünk, lassan leeresztem
a karom. Ujjanként lefejtem a kezemet a fogantyúról, és püspöklila
tenyerem láttán felállítom a diagnózist, miszerint elhalt a kézfejem.
A vámhivatal épülete deszkafalaival méltó párja minapi szállásunknak.
A kb. 12 négyzetméteres házikóban egyszerű íróasztal mögött
ül a vámos. Körben mindenféle falragasz, de nem úgy tűnik, hogy
turistáknak szólna. Egy fénykép kivételével, amely egy holland lányt
ábrázol, aki pár hónappal ezelőtt eltűnt az országban. Nem ritka,
hogy egy-két turistának nyoma vész. A hivatal előtt zörög a buszunk,
amely Floresbe visz bennünket. Miután bekerül a pecsét az útlevelünkbe,
a csomagokat feldobálják a busz tetejére. Menetrendet sehol
nem látok, de miután mindannyian elhagyjuk a ,,migración''-t, a teli
busz elindul. Egész nap utazunk földutakon, falvakon, kies és kietlen
mezőkön át, néhol műúton is repesztünk. Így utólag minden elismerésem
a guatemalai buszsofőröknek, akik nem kikerülik a buckákat,
hanem átrepülnek felettük. Egy kicsit rázott a busz, de a vesém azért
nem szakadt le, és estére, az ígért időpontra meg is érkezünk
\textbf{Flores}be.

Flores másnapi kirándulásunk kiindulópontja \textbf{Tikal}ba, Guatemala
leghíresebb maja romvárosába. Vezetőnk egy maja indián, aki a dzsungel
növényeit is megismerteti velünk. Egymás után mutatja a gyógynövényeket
és a veszélyes, mérgező leveleket. Olyan tömény növényismereti
órát tart nekünk, hogy sűrűn jegyzetelve akár gyógynövénykalauzt
is írhatnánk a kirándulás végére. Most mégis inkább érzékszerveinkbe
engedjük be a tudást, megtapogatjuk, megszagoljuk, ujjaink közt morzsoljuk
a leveleket, szárakat, gyökereket. Az égbe nyúló trópusi fákon
akkora pálmák kapaszkodnak meg, hogy önmagukban is szép díszei
lennének egy konferenciateremnek. Élősködő zöldek tömegei ,,terhelnek''
egy-egy matuzsálemet. A vándorlófa léggyökereinek segítségével
a fény felé törekszik, így lépeget lassan, s ezzel élete során több
méteres helyváltoztatásra is képes. Átgyalogolunk egy terebélyes példánya
alatt, ahol a csapás jobb és bal oldalán is ugyanaz a fa áll. Léggyökereit
leengedte a földre, és nagy törzsével azokra támaszkodva meghajlik a
kívánt irányba. Pókmajmok lengenek felettünk. Ők a leggyorsabbak
a majmok között. Fürgén szökellnek ágról ágra, mozgásuk merész,
akrobatikus ugrásokban bővelkedik. Karjuk hosszabb, mint a testük,
s emellett segíti őket hosszú lábuk és kapaszkodó farkuk is az
átlendülésben. Pókmaki ropogtat a lombok közt, keresem, hogy hol lehet.
Érdeklődve nézzük egymást, én búgok neki egy-két idétlen becéző
mondatot. Ő egyik hosszú karjával és farkával átfogja a fatörzset, fejét
és mellét lefelé lógatva szemrevételez. Egy idő után komótosan visszaül
az ágra, egyik lábát felemeli és lepisil. Még időben sikerül arrébb
ugranom a sugár elől, és köszönés nélkül otthagyom.

Bolyongunk az erdőben, néha elérünk egy-egy romot, mint például
utunk legmeredekebb piramisát. Nincs nagy késztetésem a tűző napsütésben
a mászáshoz, de hát az nem lehet, hogy a ,,leg''-et hagyjam itt
hódítás nélkül. Elindulok kopott, vagy félméteres lépcsőfokain. Az első
pár lépést még felegyenesedve teszem meg, de felfelé nézve erős szédülés
kap el, így mászva folytatom az utat, mint minden előttem járó.
Monoton léptekkel haladok fölfelé, kiráz a hideg a gondolatra, hogy
visszatekintsek. Tudatállapotom már nem azonos azzal, ami a kezdő
lépéskor volt. Nyöszörögni kezdek, kioldalazom a piramis közepén
épült lépcsősor szélére, és felhúzom magam a ,,pihenőteraszra''. Nekem
ennyi is elég volt, így félútig. Innen még nincs igazán jó kilátás, de már
feleslegesnek érzem, hogy tovább erőlködjem. Már pár órája túrázunk a
párás dzsungelben, aminek hatását itt a piramison megpihenve kezdem
csak érezni. Nagyon kimerültem. Milyen igaz a mondás, hogy hosszú
úton úgy pihenj, hogy lassabban járj vagy állj meg egy kicsit, de ne ülj le,
mert nehezebb lesz továbbmenni. A tanácsot megfogadva nem is ülök,
hanem kiterjesztett lábbal és karral fekszem a magaslaton. Szalmakalapom
az arcomra fektetem, és eggyé válok a mexikóvárosi caballeróval.

\illustration{page-073}{A legmeredekebb piramis a tikali maja romvárosban}

\bigskip
\begin{itshape}
Ég és föld között pihenek, lombkoronák fölött, a menny felé félúton.
Nem tartozom igazan egyik szférához sem. Útközben vagyok. Törött
szárnnyal érkeztem e köztes létbe, hogy bal felemre gyógyírt keressek. S ha
megtalálom a varázsszert, felfedjem titkát a sok sérült szárnnyal verdeső
lezuhantnak. Súlytalannak képzelem magam, felemelkedem és elhagyom
a piramisköveket. Kinyitom a szemem. A dzsungel zöld tengere fölött
úszom, s ha rágondolok, már magamba is húzom az erdőt. Világokat teremtek
és átalakítok. Kattan bennem egy zár. Keresem a kincset: vajon
hova rejtették a maják? Érzékeny detektoraimmal koncentrálok erre a
kérdésre, és testem fénysebességgel lövi ki magát az esőerdő sűrűjébe.
Leereszkedem egy ponton, ahol lábamat mágnes húzza lefelé. Térdre esem
és ásni kezdek. Kaparom a földet először körömmel, majd mint a kutyák,
amikor csontot ásnak elő, megszállottan dobom a hátam mögé a sárdarabokat.
Éles fájdalom hasít belém, a tenyerembe beleáll egy fémdarab, s végigcsorog
rajta a vér. Díszesen megmunkált aranylemezt tartok az arcom elé.
Már vérben forog a szemem, körmeim letörnek, térdem-karom felsebződik
e munkán. Gyűlik az arany körülöttem. De nem kapok levegőt, ó, Istenem,
segíts! Összecsuklom a gödör előtt. Álmomban nagyanyám jön felém, s én
rábízom a kincs őrizetét. Reggelre azonban eltűnik a sok drága holmi, s én
felüvöltök: -- Mit tettél, Nagyanyám? Hol a kincs, hova tetted? -- Ő fehér,
átlátszó lepelben, festetlenül, ezerszer font kontyából hófehér hosszú haját
leengedve a távolba mutat. -- Odaadtam, abba a fehér házba. -- Térden
csúszva, zokogva megyek felé. -- Miért tetted, miért tetted? -- De már nincs
ott senki, aki elé leborulhatnék. Erőtlenül mászok tovább, kifáradva a
zavarodottságtól. Megyek, megyek, de nem sírok már, és gyorsabban mászok.
Az aranyat a fehér házban jó helyen tudom, s vágyam iránta elengedem.
Mellemre esett fejemet felemelem. Fél térdre emelkedem, majd lépésről
lépésre kiegyenesedem. Az állati létből homo sapiensszé -- Ú] EMBERRÉ --
fejlődöm. A csupasz, rögös föld befüvesedik lábam alatt, megritkul az erdő,
s egy tisztás tűnik elő. Odaérve a kapu zárjában már megfordított kulccsal
kinyitom az ajtót, és átlépek rajta. Az ártatlanság kertjében vagyok. Egy
tigris követ a rétre, mellém lép, és lefekszik a fűre. A szemközti dombon,
virágzó fák között paradicsommadarak repülnek. Ahol az imént megérintettem
a földet, ott kristálytiszta patak fakad. Érett gyümölcsök hullanak
kezembe, s a színes virágok pillangókká válnak. A boldogság kék madara
fütyül az ágon. Tisztán dobog a szíve.

A határtalan erdő és mező a legnagyobb gazdagságot nyújtja, mindent,
amire szükséged van. Ha egyszer mégis többet akarsz, mint amennyi szükséges,
az erdő azt is megadja neked, de a kék madár hamisabban énekel
majd fölötted. Aztán még többet kívánsz, idegesíteni kezd a kék madár
veszett rikácsolása, és lelövöd. Mész tovább, és irtod az erdőt. Már nincs
vészharangod, mert eltetted láb alól. És egyszer csak elfogy az erdő. Te ott
állsz boldogtalanul szíved kisemmizett holdbéli táján, és csodálkozva
nézel rá elszürkült művedre. Emlékszel egy énekesmadárra, amelynek füttye
messzire szállt, és te őt követve jártad az erdőt, élvezted zamatos gyümölcseit,
s boldogan éltél e harmóniában. Értetlenül állsz e pusztításban, és
nem tudod, hová törekedtél.

Az erdőn átszűrődő vékony napsugár cirógatására felébredek a kiásott
gödör szélén. Az aranykupac ott csillog mellettem. Visszatolom a kiásott
aranyat a mély verembe, és rádobálom a földet. Ledöngetem a lábammal,
és gallyakat húzok rá. A mágnes húzóereje elenged, és újra a dzsungel fölött
szállok. Visszaérkezem a piramisra, s hanyatt dőlök a teraszán. Rugalmas
leereszkedést vizualizálok. Arcomról leveszem a szalmakalapot, és farral
lefelé visszatalálok a földre.
\end{itshape}
\bigskip

Folytatjuk utunkat a tikali rengetegben. Meredek domb magaslik
előttünk. A fák földből kiemelkedő gyökérhálózata fölé épített kanyargós
falépcsők vezetnek fel a hegyre. Dobogva lépdelünk rajtuk.
A dombtető épületének lépcsőjén fújjuk ki magunkat. Figyeljük a
közeli fákon lengő pókmajmokat és a félénken felénk közeledő ormányos
medvéket. Hosszú orrukkal az ösvények mellett túrták korábban
a földet, de mindig két-három méteres távolságot tartottak tőlünk.
Hosszú, csíkos farkuk olykor antennaként áll fölfelé, testükkel derékszögben,
máskor leengedve, a föld fölött pár centivel kanyarog utánuk.
Ormányos medve érdeklődve nézi, ahogy kekszet ropogtatok. A töltetlen
háztartási keksz talán nem árt meg neki, gondolom, és dobok
egyet felé, két-három lépcsőfokkal lejjebb. Körülbelül másfél méternyire
esik le tőlem, így a medve biztonsági zónáján belülre pottyan
a győri termék. Megnyújtott nyakkal, lassan, szemből közelít, majd
megáll és oldalra fordul. Jól megnéz még egyszer, gyors léptekkel eléri
a kekszet, felkapja és ismét hátrább lép. Egyik vékony mellső lábával
teszi a szájába sűrű egymásutánban a falatokat. Sztereóban roppan a
keksz a fogunk alatt. A közös uzsonna után kifelé indulunk a nemzeti
parkból, buszra szállunk, és egész éjszaka utazunk a fővárosba,
\textbf{Guatemalaváros}ba, ahonnan másnap este valósággal menekülünk.
Délelőtt egy-két látnivaló megtekintését követően elvegyülünk a főtéri
piac tömegében. Vásárolok egy tukáncsaládot. A szivárvány színeiben
pompázik a kerámia tukánmama két kicsinyével.

A csoportnak készpénzhez kell jutnia, ehhez biztonságos bankot
kell keresnünk. Megbeszéljük, hogy két óra múlva itt találkozunk,
a parlament előtt. Négyen átutazunk a város egy másik negyedébe. Az
első bankban sikertelenül járunk, a másodikban is körülményes, hogy
a bankkártyáról pénzt vegyünk fel. Bekérnek egy igazolást a magyar
banktól, és ezzel megkezdődik a többórányi bankközi egyeztetés.
Sajnos nincs mit tennünk, várnunk kell, pénz nélkül nem folytathatjuk
utunkat. Mire visszaindulunk a csoporthoz, már késő délutánra jár.
Eközben a hátramaradottak elrablásunk lehetséges verzióit taglalják.
Délutánra már olyan mértéket üt meg kétségbeesésük, hogy többen
sírva fakadnak. A szürkületben kétes alakok gyűlnek össze a téren.
Ideérkezésünk előtt pedig nem titkolták előttünk, hogy rendszeresen
rabolnak ki turistákat fegyveres bandák. Visszatértünkkor örömkönnyek
fakadnak, de egyetértünk, hogy mostanra mindenkinek elege
lett a fővárosi utcai hangulatból. Az időközben felvett csomagjainkkal
tömött sorban gyalogolunk a buszállomásra. A terminál körül is
gyanúsan sandítanak ránk. A váróteremben kíséretet kapunk: két őr
vagy rendőr -- mindenesetre hivatalos személy -- kísér minket a buszig.
Vigyázzák, hogy minden táska bekerüljön a csomagtérbe. Megkönnyebbülten
hagyjuk el Honduras irányában a várost, és pár óra múlva
megérkezünk Chiquimulába. Fővárosi megpróbáltatásunk sztorivá
szelídül a biztonságos hotel medencéjében, ahol kis társaságunk
elégedetten áztatja fáradt izmait és pihenteti megviselt idegeit.

\illustration{page-077}{A piac Chiquimulában}

Reggel a piac melletti állomásról indul a buszunk a határhoz. Papayával
teli egy dzsip raktere, naranccsal egy másiké. Mindenféle
gyümölcs kapható, így a számunkra egzotikus kaktuszgyümölcs vagy
kókuszdió is. Meglékeltetek egy kókuszt, és szívószál segítségével
élvezem az enyhén édeskés italt. Borzongatóak a pirosra, zöldre, kékre
festett kiscsibék. A tojásba festéket fecskendeznek, amitől a csibék
tollpihéi átszíneződnek, és már e furcsa színben kelnek ki. Indul a
buszunk, robogunk Honduras felé, s hogy egy perc nyugtunk se legyen,
idegenvezetőnk veszi át a kormányt a hepehupás földúton. A határsorompó
egy szál madzag, de ötven méteren belül vámhivatal működik,
Vízum nélkül üldögélünk a ház előtt, és reménykedünk az átjutásban.
Úti célunk, \textbf{Copán}, a maja romváros 12 kilométerre fekszik
innen. Végül egyezség születik a vámon. Kísérettel átmehetünk, de
délután ötre vissza kell térnünk. Kapunk két dzsipet, ezeken jutunk el
Copánba. Útközben elered az eső, és én a vezető mellett ülve, menet
közben veszem be a hátizsákokat az oldalsó ablakon át a platón utazó
társaktól. Copánba azért a lépcsősorért jöttünk, amelynek alsó fokaitól
fölfelé haladva 2500 írásjel látható évszámokkal, és az események
rövid leírásával elolvashatók a copáni maja történelem fordulópontjai.
Az uralkodók életének jelentősebb eseményeit ábrázoló sok sztélé
mellett akad egy hatalmas kőteknőc, amely áldozati kő volt. A teknőc
hátán vésett, spirál alakú vágatban csurgott le a vér. Az egyik falról két
lábra egyenesedett tigris mosolyog le ránk. Walt Disney állítólag erről
a szoborról formázta rajzfilmbeli tigrisét. Visszaúton még Hondurasban
estebédelünk. Az étterem asztalain piros kockás terítő, a falakat
a magyar csárdákhoz hasonlóan tarisznya, kalap, nyereg, földművelő
eszközök díszítik.

\illustration{page-078}{%
	A teknőc, az egykori áldozati kő a copáni maja emlékek között}

A következő nap Chiquimulából indulunk Guatemala régi fővárosába,
Antiguába. Útközben betérünk Esquipulasba egy kis pihenőre,
a koloniális időket őrző hófehér, csipkés templomába. E templom már
előfutára a festői Antigua spanyol építészeti remekműveinek.
A buszmegállókban gyerekek, nők a fejükön tartott kosarakból kínálnak
hosszában felszeletelt és szárított banánt, a banánchipset, nejlonzacskókba
töltött jeges üdítőt.

\textbf{Antiguá}ban szállodánk egy spanyol hacienda mása. Az udvar közepén
robusztus szökőkút, melyet virágzó növények fonnak körbe.
A tornácos, oszlopfős emeleti részen kapunk szobát, melynek berendezése
szintén koloniális stílusú. Antigua a spanyol hódítás első napjaitól
az 1821-ig tartó időszakban nyerte el mai szerkezetét, az épületek
döntő többsége ebből az időből származik. Kétszáz éve mondhatni
nem nyúltak az épületekhez, amikor földrengések rombolták a várost.
Ekkor alapították az új fővárost. Azóta is számos rengés rázta meg
Antiguát, a házak nagy repedésein láthatóak a nyomaik. A települést
három vulkán veszi körül, egyikük érkezésünk előtt egy héttel kezdett
el újra füstölögni. A tűzhányók felhőkbe burkolóznak, csak a kúpjuk
nyúlik ki a takaróból. Este a főtéren marimbazenészek játszanak, s mi
ott lejtünk a spanyol oszlopok, kerengők között, majd a téren
körbe-karikába, végül a tangó ütemére hátrahajlunk a szökőkút fölé.

Több ezer méteres szintkülönbséget teszünk meg, mire elérünk
\textbf{Panajachel}be. Az \textbf{Atitlán-tavat} Lord Byron a világ
legszebb tavának
tartotta. Opálkék vize fölött, a túlparton zafírkékben dereng két vulkán,
s egy vitorlás szórja ránk gyémántcsillanását a tóról. Pihenőnap
a mai, a parton belemerülünk a lágyan fodrozódó hullámok egyenletes
csobbanásába. Bóklászunk a kisvárosban, betűzzük a feliratokat,
leülünk a boltok portálja elé, nekidöntve hátunkat a tartóoszlopoknak,
és bámuljuk az utca forgatagát. Kutyákkal barátkozunk, tacost
falatozunk, majd egy bár utcai frontján meglocsolom a látottakat egy
kókuszdiónyi gyümölcskoktéllal. Este közösen választunk éttermet,
ahol egyszer csak magyar nóta csendül fel a hangszóróból. Megkérdezzük
a tulajdonost, hogyan jutott hozzá a kazettához. A főnök
már nem emlékszik erre. Nem is tudja, milyen nyelven énekelnek, de
nagyon tetszik neki ez a zene, így gyakran forgatja le a vendégeknek.
Örül, amikor megtudja, hogy abból az országból jöttünk, ahol e muzsika
született.

\illustration{page-080}{Az opálkék vizű Atitlán-tó Guatemalában}

Az Atitlán-tavat felhőkön szállva hagyjuk el, föléjük kerülünk, amikor
azok megrekednek egy völgyben. A mexikói határig vagy hatszor
szállunk át buszról buszra. A buszokon mi is a maja nagycsaládhoz
tartozunk. A jármű helyettesíti a házat, a kertet és a kamrát, ideiglenes
szomszédokkal. A hátizsákokat, bőröndöket, zsákokat a busz tetejére
dobálják fel, majd lekötözik. Érdemes vízhatlan csomagolással
utazni, különben az eső nagy kárt tehet az úti holmiban. A járművek
több évtizedes tapasztalattal rendelkező, színesre festett amerikai
veteránok. A guatemalai buszok utasterének idillikus képéhez tartozik,
hogy mi sem természetesebb, mint hogy a hármas ülésen öten is elfértünk.
Szomszédom lábához húzott zsákjában a baromfiudvar utazik,
s két sorral hátrébb egy kismalac visít. Ügyelek a tökre, melyet a fejem
fölött lóbálnak, s a kanyarokban a kézi csomagok helyváltoztatását
figyelve mérlegelem, mikor lendüljön a karom, hogy a zuhanásból elkapja
őket. Kétségkívül a buszon az ajánlott utaslétszámnál kétszer többen
vagyunk. Mégis az őslakosok szó nélkül lépnek vagy ülnek közelebb
szomszédjukhoz, ha újabb felszálló érkezik. Magától értetődően veszik
tudomásul, hogy az a másik is igyekszik valahova, így neki is fel
kell férnie közénk. Az élni és élni hagyni elfogadó magatartása jellemzi
e vándorokat. Mielőtt elérnénk Mexikót, még egy éjszakát töltünk
Guatemalában. A faluban a csomagcipelés elleni lázadás eredményeképpen
vezetőnk ,,eltéríti'' a buszt a hotelig. Hát ilyenek a guatemalai
buszok, utasok és sofőrök.

Másnap tíz órát ülünk egy kereszteződésben a buszcsatlakozásra
várva. A tortillaüzem előtt öten ülünk körben a járdán, és nézzük,
ahogy a lámpa századszorra vált át zöldre vagy pirosra. Nézzük a gyalogos-
és járműforgalmat, és ki-ki vérmérséklete szerint kommentálja
a látottakat, az út eddigi történéseit. A tortillákat mintha az idők kezdetén
is itt sütötték volna, és készítésük a mulandóság felett állna. Van
valami megnyugtató abban, ahogy az olajba dobják a tésztát, ott az
felhólyagosodik, miközben megfordítják, majd kiveszik a forró edényből.
A sütés mozdulatainak állandósága, mely már évszázadok óta
hozza ugyanazt az eredményt, időtlenül lebeg körülöttünk. Elérünk
a határra, ahol a buszállomástól a guatemalai vámházig még vidámak
vagyunk, riksás viszi a bőröndöket. Az útlevélkezelés után azonban
újabb túlélőgyakorlat következik: a senki földjén, a háromszáz méter
hosszú hídon történő átjutás teljes menetfelszereléssel és gyalog.

Ahol minden megpróbáltatást elfelejtek, az a Csendes-óceán-parti
\textbf{Puerto Escondido} szörfparadicsoma. A pálmatetős bungalók mellett,
a kókusz-, legyezőpálma- és banánfák árnyékában úgy töltök el öt napot,
hogy szinte csak étkezni kelek fel naponta egyszer, esténként a
hotel medencéje melletti nyugágyról. Csupán három megrázó pillanat
üt rést e monoton meditáción. Első nap először és utoljára barátkozom
az óceánnal a \textit{Veszélyes partszakasz, tilos a fürdés} figyelmeztető
tábla ellenére. Térdig sem ereszkedem a vízbe, amikor úgy elkap egy áramlat,
hogy minden erőmre szükségem van ahhoz, hogy talpon maradjak.
Vad erővel oldalra, majd befelé húzza a lábszáramat, és én kiáltani sem
tudok az ijedtségtől. Egy idő múlva ki tudom emelni az egyik lábam,
és minden erőmet összeszedve, lassan haladva érek ki a partra. Miután
leckét kapok, hogy milyen porszem vagyok a víz erejével szemben, útitársam
pedig monterájára lép, nem fürdőzünk többet. A harmadik ijedelem
villanyoltás után, a moszkitóháló alatt ér. Az asztalon hagyott
kifliért bemászik egy iguána a tető alatt, le a falon, rá az ágyamra, és
a fejem tetején végiggyalogolva lecsap az élelemre.  Megmozdulni nem
merek, most viszont tudok kiabálni. Mire felkapcsoljuk a villanyt, a jó
félméteres állat a zuhany tetején dézsmálja a zsákmányt.

A szörfösök reggelente, majd naplemente előtt jelennek meg a plázson.
Napközben a hőség miatt kihalt az óceánpart. Az Egyenlítő fölött
tizenhat fokkal egész hirtelen, negyedóra alatt bukik az égi fény
a vízbe. A naplemente minden este más színárnyalatú. Az első nap
aranyecsetet használ, a másodikon ezüsttükröt, a következőn a sötétedő
kékség alatt fénylő grafitszínűre festi az óceán, majd égő cinóbervörös
palettával búcsúzik tőlünk.

Háromhetes élményeim a nyugágyban töltött idő alatt ivódnak végleg
belém. Fokozatosan telítődnek sejtjeim az eddig csak a bőrömre
és retinámra tapadt pillanatokkal. Ehhez a felszívódáshoz szükséges
a medence nyugalma, amikor a vizén vibráló pálmaágak és hibiszkuszok
minden tolakodó gondolatot kiürítenek a fejemből, és a semmi
ürességébe folyik bele az élmény. Ott az is semmivé válik, mert egy lesz
velem. Ezentúl magamban hordom a napfényt, a poros utakat, a caballero
lengő nadrágszárát, a dzsungel páráját, az óceán hullámainak
hangos közeledtét, a kifelé áradó kíváncsi tekinteteket, a befelé néző
öregeket. Leülepedik bennem minden, ami már semmilyen vegykonyhán
nem vonható ki többé. Sejtjeim továbbadják egy magasabb rendnek,
mely körbevesz, s melyhez testem csak közvetít.

Mexikó és Guatemala, lám, így magammal viszlek az örökkévalóságba!
