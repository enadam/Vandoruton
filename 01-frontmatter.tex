\coverillustration{page-000}

\clearpage
\begin{center}
KURTA MARGIT

\bigskip
{\Large\textbf{VÁNDORÚTON}}

\medskip
KELET ÉS NYUGAT, ÉG ÉS FÖLD KÖZÖTT,	\\
BELSŐ UTAKON A KÖZÉPPONT FELÉ

(TÖREDÉK)

\vfill
SZÓFIA KIADÓ \\
Budapest, 2013
\end{center}

\clearpage
\otherfamily

\begin{vplace}
\begin{flushright}
\textbf{Ajánlás}					\\
Apunak, aki nem hisz az asztrológiában,			\\
de hisz a matematikában és távcsővel			\\
fürkészi a csillagokat.					\\
Anyunak és öcsémnek, Lacinak.				\\
Bencének útravalóul.

\bigskip
Kalo Jenőnek, Paksi Zoltánnak, dr. Daubner Bélának
és dr. F. Gallo Bélának köszönettel.
\end{flushright}
\end{vplace}

\clearpage

\begin{flushleft}
Lefizettem mindent az utolsó fillérig.	\\
Hóhérból áldozattá lettem.		\\
Áldassék érte az Úr neve.		\\
Szabad vagyok és örök.
\end{flushleft}

\vfill\noindent
A legtöbbet jelent nekem Kurta Margit írása. Szavaim visszhangját
hallom benne, mint akik ugyanazon helyről indultunk és ugyanoda
térünk vissza. Idő nem számít hozzá, csak mélységes hit. Rengeteget
éltünk át együtt.

Kezdet és vég nélkül ismerjük egymást. Ezoterikus testvérek vagyunk.
Áldott légy. És áldottak mindazok, akiket megérintett idea-áramod,
mely mindörökké elkísér.

Számtalanszor elmentél és tértél vissza újra, hogy segíts azokon,
akik rászorulnak az igaz valóságra és már rápillantottak a csillagokon
túli valóságra, a Napra.

Kívánom, hogy az Angyal továbbra is Veled legyen és kísérjen,
amerre tartasz.

\bigskip\bigskip
A vetületeken túli szeretettel: \hfill \textit{Szepes Máriád}

\bigskip\bigskip\noindent
Budapest, 2006. május 15.

\clearpage

\begin{flushright}
\textbf{Mottó}						\\
Írni nem lehet pénzért, és semmiféle tapsviharért.	\\
Csak elszántsággal, teljes szívvel, magányosan.		\\
Amikor a világ leginkább bolondnak képzel.
\end{flushright}

\chapter{ELŐSZÓ}

\begin{itshape}
Vándorok vagyunk. Még ha soha nem is hagytad el a falut, ahol élsz. Lehet,
hogy szélesebb horizontot fogtál be, mint a kalandor, kinek saruja több
földrészen is partot ért.

Vándorok vagyunk, kik ezerféle fényben, homályban és alakban látjuk
ugyanazt a tájat. A táj egy és ugyanaz az adott pillanatban, olyan kerek
egész, melyben minden vándor szívdobbanása benne van, ó de szép.

Mennyit látsz e tájból, mekkora szeletét? Ha vágyod látni az egész világot,
indulj magad felé. Mert egyetlen igazi utazásra adtak neked jegyet,
az pedig önmagad bebarangolása. Könnyű vagy nehéz utakon, ez viszonylagos.
A térkép, melyet kezedbe kaptál, számos útvonalat felkínál, hogy
célodat elérd. Rajtad áll, hogy könnyebb lesz az út, vagy még kóborolsz egy
kicsit. Térkép nélkül is önmagad felé haladsz, de több a szakadék.

Nehéz az út a középpont felé? Látásmódodon kell változtatnod, s tapasztalásod
bölcsességgé válhat. Mert a tájat csak úgy festheted át, ha te
másképp nézel rá. Átformálni csak magadat tudod, de útközben ezáltal
változnak lábad alatt a kavicsok is, ez az egyetlen út, hogy vezess, s ne
csak kövesd csillagaidat. Mert ha mást akarsz változásra bírni, sorsa égre
írt jelei hátadon nehezebbé válnak, s roggyant térdekkel tudod csak saját
utadat is járni.

Ha e tájon valaki megmutatja neked valódi önmagát, ha csak egy múló
pillanatra is -- és tudni fogod, melyik ez a mozdulat --, még ha úgy érzed
is, egészen nem tetsző, amit láttál, vagy éppen még taszító is, örülj, mert
abban a pillanatban megláttad, ahogy részed benne van az egészben. Az
álarc mögött bármi csúfat látsz, az az igazi jellem, így mégis valódibb és
becsületesebb, mint a felkent máz. Ahhoz pedig, hogy mások meztelen arcának
örülni tudj, ismerned kell a magadét is kendőzetlenül.

Hol az igazság? Itt magadban, a saját igazságodhoz való igazodásban
úgy, hogy közben másoknak nem ártasz. Az önmagadhoz való közeledésben,
a középpontodban, vagyis a szívedben, mely kilátóból a táj részletei
összeolvadnak, a teljesség tárul eléd, és valósággá válik.

Vándor, ekkor elcsendesedsz, mert bejártad a világot és megérkeztél.
\end{itshape}

\section{Szereplők}

\begin{list}{}{\def\makelabel#1{#1 --}}
\item[útitársak]	a maguk rajzolta tájban
\item[tájak, városok]	törékeny díszletek, tükörképeink
\item[kellékek]		riherongyok, tárgyak -- mulandó kis semmiségek
\end{list}

\normalfont
