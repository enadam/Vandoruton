\clearpage
\chapter{VÁNDORÚTON}

Gyermekkoromban nagyapám esténként a vándorlegényről szóló
meséivel altatott el bennünket öcsémmel. Maga szőtte a szegénylegény
történeteit vándorlásairól és küzdelmeiről. Mindig azzal indult
a mese, hogy tarisznyáját megtöltötte hamuban sült pogácsával, mely
csodálatos módon kitartott az út végéig. Voltak vissza-visszatérő
epizódok, melyekből ha kihagyott egy részletet, mi azonnal figyelmeztettük
őt: Az nem úgy volt, Papa! Nagyapám vándorlegényről szóló
mesefolyama volt kicsi gyerekkorom meghatározó kalandélménye. Beleéltem
magam a legény hihetetlen utazásaiba, átéltem tapasztalásait,
és mindig örömmel, valamint megnyugvással fogadtam jutalmát, melyet
a küzdelmek után a királytól kapott. Akkor még értettem minden
lépését és annak következményeit is, természetesnek tartottam, hogy
így működik a világ.

Felnőve azonban mintha elfelejtettem volna e meseélmény értelmét.
S egyszer csak azt vettem észre, hogy beléptem a meséskönyvbe,
mint Alice Csodaországba. Újra kellett tanulnom és tapasztalnom a
történeteket, immár nem mesehallgatóként, hanem vándorleányként.
Kitárult a világ, vitt a sorsom messzire. Ma már tudom, azért jutottam
ilyen távoli vidékekre, hogy egyszer majd valóban hazatérhessek.
A messzeség adja meg a lehetőséget, hogy önmagamhoz visszataláljak,
és a külvilágban tett kalandozások közben egy belső utazáson is
elinduljak.

Én is csak egy vagyok a sok-sok vándor között, akinek a tarisznyájába
belepakoltak valamit útravalóul, vagy éppen magam raktam bele,
csak már nem emlékszem rá. A tarisznya tartalmával jól kell hogy
gazdálkodjam, vigyáznom, hogy a pogácsák frissek maradjanak, és ne
a penész üljön meg rajtuk, ami miatt esetleg idő előtt tönkremegy a
muníció.

Legyen ez egy meghívás neked egy világ körüli útra, de még azon is
túlra, fel a csillagos égboltra, miközben időnként belenyúlok a
tarisznyámba, és megnézem a pogácsákat. Lássuk, mi sül ki belőlük.

\section{A megérkezésről egy ismeretlen helyre és a jelenlévőségről}

Szöulban épphogy megérkeztem a szállodába és letettem poggyászom,
a piacra indultam. Nem éreztem a levegő feszességét, még a thai
pára és lazultság burka vett körül, még nem voltam jelen. Így fel sem
tűnhetett, hogy tömött sorokban menetelnek mellettem az úttesten,
én vidáman lépkedtem előre. A sisakok, védőpajzsok csak átsuhantak
a szemem előtt, hangtalanul lépdelt mellettem az ezred. Egyszerre értünk
a térre. Ekkor hirtelen kirobbant ebből a feszültségből az első
üvöltés, és abban a pillanatban meglátom a tér láthatatlan válaszvonalának
táblákkal nekifeszülő egyetemista tömeget. Gumibotok emelkednek fel,
és én rohanok le az előttem kínálkozó aluljáróba. Futok
előre, látom az árusok kapkodó igyekezetét, ahogy a portékájukat szedegetik
össze, és fehér maszkot öltenek az orruk és a szájuk elé. Futásom
egy pillanat alatt meneküléssé változik, egy olyan összeszedett
állapottá, melyet az antilopok élhetnek át a szavannán. Minden idegszálam
a merre kérdésre koncentrál, százszoros sebességgel fogadom
az információt a szememen át, és csak azt látom magam előtt, amerre
mennem kell, a fölös nézeteket leszakítja agyam gépezete. Föl, balra a
lépcsőn. Jobbról öklendezve gurulnak le, a vér is folyik, szúr a levegő.
Öklendezem én is a hányingertől, de már távolodom a hangzavartól,
elérek egy irodaházat és befutok az előcsarnokába.

\bigskip
\begin{itshape}
Bárhol jársz, hagyd magad mögött azt, ahonnan jössz, és érkezz meg
az újra nyitott szívvel és szemmel, légy jelen ott, ahol vagy, mert különben
lehet, hogy felrobbantanak, s még csak észre sem veszed, mi készülődött.
Mindez érvényes arra az utcasarokra is, ahol napjában többször
áthaladsz, és a robbantás felcserélhető bármi mással, egy szempillantással,
amely megérinthetett az üvegen át, amelytől egész nap jókedved lett, mert
jelen voltál.
\end{itshape}

\section{Két szoborról és a válaszutakról}

Szöulban a piacon, a tüntetés utáni napon. Igazán csak nézelődni
jöttem, úgy piacozni, különösebb vásárlási szándék nélkül. Az élet
meghatározó élményei pedig mindig ilyenkor érnek, amikor a legkevésbé
számítasz rá, mert akkor erőteljesen és feledhetetlenül szembejött
velem egy összegzett élet, egy fából faragott szobor. Öregek, két
öreg, egy férfi és egy nő, egymásra támaszkodva, hajlott háttal
megérkezetten, befejezetten, teljesen. A szoborban életük tapasztalása,
közösségük megbonthatatlansága, összetartozásuk ezer szála, a
megérkezettek bölcsessége, kiegészülésük egysége. A szobor kiugrik a
térből, és nem hagyja, hogy más dimenziót lássak. -- Mennyi? -- bökök
rá. -- Az sok -- és továbbmegyek. De a szobor ekkor már elhatározta,
hogy örökre az enyém marad. Visszamegyek. -- Mégis megveszem azt
a szobrot -- mondom az árusnak. -- Nem eladó -- kapom a választ, és
csak állok ott bambán, mint azok, akik éppen most fedezik fel, hogy
meglopták őket.

Azóta is sokat gondolok arra a faragványra. Itt áll a szobában, néha
áthelyezem egyik helyről a másikra -- gondolatban. Azt hiszem, mégis
a legjobb helyet találta meg magának azzal, hogy nem repült haza
velem a bőröndömben. Sokkal jobb helyet talált magának a szívemben

\bigskip
S még egy szoborról, még előbbről. Kis szobrok csontból a pekingi
utcán, az Ég templománál, sok-sok egymás mellett. Az egyik megállít,
rám mosolyog, így vonzza oda a tekintetem. Egyértelmű, hogy hozzám
kéredzkedik, én már nem tehetek mást, mint hogy magamhoz veszem.
-- Mutathatok mást is -- szól az árus, de intek, hogy köszönöm,
nem. A taoista pap a kobrájával, a barackjával és azzal a mindenséget
tükröző nyugodt mosolyával már rám talált. Még nem tudom, hogy a
kis golyó a kezében mi is, csak jóval később látom meg, hogy a golyó az
barack, és fedezem fel, hogy a bot az egy kobra.

\bigskip
\begin{itshape}
Sokszor nem is vesszük észre, amikor megtörténik, hogy egy hétköznapi
eseményen át mit üzen az élet. Két szobor a piacon. Felkínál valamit, ami
mellett elmegyünk, és letesszük a voksunk egy másiknál. Mert sohasem
a nagy események harangzúgásában vagy az összeomlás pillanatában hagyunk
el keresztutakat, az élet jelentéktelennek tűnő sok-sok apró lépésében
már benne van, hogy az útelágazásnál merre kanyarodunk. Légy résen,
még az utolsó lépésnél is meggondolhatod magad, de amiért már kinyújtod
kezedet, azt megkapod, és pontosan azt, amiért nyúltál, s mindennek több
oldala van.
\end{itshape}
