\clearpage
\chapter{TORONTÓ}

Háromszor jártam Torontóban. Az első utazás volt egyben az első
tengerentúli repülésem is. Az indulás izgalmát csak fokozta az elvárások
magas mércéje és a szokatlanul rideg időjárás. Utam előtt egy
héttel a mínusz húsz-huszonöt fokos hidegben megfagytak és szétrepedtek
a kémények, sok tető sem bírta a hirtelen lezúduló havat és
jeget. Poggyászomba a legmelegebb ruháimat csomagoltam.

Mint egy eszkimó érkezem meg a torontói reptérre, ahol gyönyörű,
plusz tízfokos tavaszi napsütés fogad. Mire a szállodába érek, a fülem
mögül, a lábujjaim közül, mindenhonnan csorog rólam a víz. Könnyedebb
holmira nem tudom cserélni, így nyitott télikabátban, hajadonfőtt
indulok neki a városnak. Járom a szaküzleteket, tárgyalok az
ügynökünkkel. A pár napban nemigen jut idő városnézésre, csak késő
este érek vissza a hotelbe. Pedig a Niagara a közelben nagyon vonzó
célpont. Ezen a lehetőségen mélázok a szálló forgóajtaján belépve,
és mire kilépek a hallba, megvan a megoldás. Éjszakai kirándulásra
megyek! A portán megkapom az exkluzív ajánlatot. A fehér limuzin
a cilinderes sofőrrel egy óra múlva értem jöhet. Örömöm rögtön lelohad
a diszkréten elém tolt árajánlat láttán, de megkérdezem, más nem
érdeklődött-e a program iránt. Egy angol házaspár szintén szeretne e
huszadik századi hintóval vidékre hajtani. A portás összehozza a
találkozót, majd az angolokkal közösen kibéreljük a kocsit. Éjszakára
ismét megjön a tél, lehűl a levegő, és jeges hó csapkodja az autó
szélvédőjét. Alig merek a vízesésnél a sötétben, a süvítő szélben kiszállni.
Az orrommal pillanatok alatt elveszítem a kapcsolatot. Arcom is egyre
merevebbé válik, de szememmel erősen kutatom a színes reflektorokkal
megvilágított zuhatagot, hogy lássam már végre azt a víztömeget,
mely óriási robajjal zuhog le valahol itt előttem. A Niagara jó része
megfagyott az utóbbi hetek kemény időjárásának köszönhetően. A hatalmas
jégtáblák mint zuhanyfüggönyök takarják el a mögöttük nagy
hanggal spriccelő locsolófejeket. Nem elég, hogy esik a hó, a katlanból
is száll fel ránk a jeges pára. Egyszerre gyönyörűek és rémisztőek a
változó színekben játszó gerendányi és karhossznyi jégcsapok orgonasípjai,
a morajlás és a ködből előbukkanó szürke frakkos, cilinderes
sofőr alakja. A sétányon a fehér sötétségben majdnem összeütközöm
az angolokkal, és már nem lepődnék meg egy sikolyon, valamint Sherlock
Holmes bemutatkozásán. Ő vezeti a nyomozást, és megkér, hogy
térjek vissza a szállodába, és ha nincs ellene kifogásom, a reggelinél
feltenne pár kérdést. A nyomozás már másnap lezárul, a tettesek a
csippeva indián őslakók, akik a dübörgő víz engesztelésére egy fiatal
lányt dobtak a katlanba még sok száz évvel ezelőtt. Szelleme bolyongott
aznap este a páraködben.

Miután üzleti ügyeim Londonba szólítanak, nem törődve a hóviharral,
elindulok a repülőtérre. Cammognak az autók, néha pörgünk
egyet-egyet magunk körül. A terminálon megtudom, hogy több járatot
is töröltek, de az én londoni gépem elindul, bár az indiai járat utasaival
kiegészülve. Több mint egyórás késéssel végre beszállhatunk. Még
nem tudom, mi vár rám. Hosszú télikabátomat bélésével kifelé gondosan
összehajtogatom, és beteszem a kézipoggyásztartóba. Ekkor megjelenik
a mumus. Kerekes kocsiját rádobja a kabátomra, jól be is gyűri
vele, de nem szólok semmit. Pedig akkor kellett volna menekülőre fogni.
Az intő jelet sajnos elengedtem magam mellett. Eljön a vacsoraidő,
alig várom a forró teát. Amikor megkapom és a tálcámra helyezem, az
előttem ülő mumus pihenni készül. Szó nélkül megnyomja az ülése
háttámláját hátradöntő gombot. A forró tea az ölembe és a székem
ülésére ömlik. Rezignáltan hátranéz, hogy mit sikítozom itt, és szó
nélkül visszafordul. Cserenadrág persze nincs a táskámban, türelmesen
várnom kell, míg cukros-ragacsosan megszárad rajtam. Tele van a
gép, egy üres hely sincs, sehová nem tudok átülni. Az ülésemre kispárna
kerül, de alatta szintén ragadós nedvesség marad. Éjszaka utazunk.
Reggelre érünk az óceán túlpartjára, és csak késő délután indul tovább
a gépem Budapestre. Eredeti tervem, hogy majd menet közben alszom,
dugába dől. A bőröndöt Budapestig feladtam, ruhacserére Londonban
nyílik először lehetőség. Ébren álmodozom egy frissítő fürdőről
és egy száraz nadrágról. Kábán hallom a leszállást jelző kapitányi
mondatokat, le sem hunytam a szemem. És a mumus újra akcióba lép.
Kerekes kocsiját egy erős mozdulattal kirántja a csomagtartóból, és
ezzel a vad indulattal keresztben végig is hasítja kabátom bélését. Már
sírni sincs erőm, mint egy hajléktalan utazom be a városba, kabátom
lifegő bélését felfogva, cuppogós, megkeményedett, recsegő nadrággal.
Felszabadító érzéssel fordulok vissza a Heathrow-ra az új nadrágban,
korgó gyomrom most másodlagos tényező. Az újságosstandnál veszek
egy tábla csokoládét, s mint aki sose evett, tömöm pillanatok alatt
magamba. Harminckét órája nem aludtam. A becsekkelésig még van annyi
idő, hogy az olvadt csokit lemossam a kezemről. Könyökkel tudom
megfogni a kabátom, repül utánam a bélés. Hirtelen ismerős hangot
hallok, -- Margit, Margit! -- kiáltásokat, majd röviddel utána meglátom
a cégtulajdonost, amint karját lengetve közeledik felém. Nyújtja
előre kitárt karját, én dugom hátra a csokoládés kezem. Meghökkenve
áll meg előttem, de gyorsan túlteszi magát bezárkózásomon. Vörösen,
elcsigázva teszem rendbe magam a mosdóban. Együtt szállunk fel az
édes otthon felé a cégvezetővel és a cégtulajdonossal. Valaki beszél
belőlem az út eredményeiről, azt sem tudom, mit mond, én félájultan,
lelassult reflexekkel, kipeckelt szemmel bámulok a semmibe. Késő este
fordítom a kulcsot otthonom bejáratán. Pár perc még az ágyig, talán
addig még kibírom, de ha nem kerülök oda rövid időn belül, biztos,
hogy meghalok.

Másodszorra egy nemzetközi kiállításra érkezem. Leszállás közben,
a városra történő ráhangolódásként nézem az Ontario-tó napsütötte,
csillogó vízfelszínét, a rajta ringó vitorlásokat, jachtokat,
kirándulóhajókat. A nagy zöld parkokat, ahonnan a mozi golflabda
alakú épületét, innen, föntről nézve, akár egy ütővel ki is lehetne
röpíteni az űrbe. Sci-fi vetítésekor micsoda témaközeli előadást
lehetne tartani odafönn. Az előadás végén pedig a mozi visszapottyanna a
földre. Ha telibe talál az ütés, akkor az \textit{Utazás a föld középpontja
felé} című Verne Gyula-regény élő bemutatója lenne programon. Elmosolyodom
ötletemen, de azért lehet, hogy memoáromban szerepelni fog
a golfverseny története \textit{Münchhausen báró után szabadon} címmel.

A kiállítás előtt és után van egy kis szabad időm. Irány a Niagara
nappal! A szeptemberi őszben égővörös juharfák szegélyezik az
Ontario-tó nagy hajlatát. Skywaynek, Égi útnak nevezik ezt az útszakaszt,
mert magasan ível át a tó fölött, mintha nyílvesszőt lőttek
volna ki fölé. Nagy parkok, majorságok mellett suhanunk el, az ország
gyümölcsöskertjébe érkezünk. A Kanada jelképéből, a juharlevélből
készített mézédes juharszirup palacsintára öntve a legfinomabb.
A juharfák szomszédságában jegenyefák levelei táncolnak az
ágakon. Zöld, sárga, vörös és barna színek kavalkádja lobogja körbe
az autóutat. A gyümölcsfarmok kis elárusító verandáinál, árudáinál
megéri megállni az ízletes, friss gyümölcsért. A stanecli körte olyan
édes és ellenállhatatlan, hogy egyszerre megeszem. Az őszibarackot
félreteszem a következő pihenőig. Niagara-on-the-Lake Kanada élő
skanzenje, azt mondják, a legszebb régi kisvárosa. Az 1800-as évek
első felének pasztellszínű házai sorakoznak egymás mellett a Király és
a Királynő elnevezésű utcákban. A régi kórház, múzeum-patika, viktoriánus
hotel, lekvárüzlet, óratorony, törvényházból avanzsált színház
régimódi épületei között egy nagyra nőtt ékszerdobozban érzem
magam. Óriások családja felejthette itt, ezen a gyönyörű természeti
tájon, kirándulás után a felborult étkezőkosarát. Az épen maradt
porcelánkészlet egy kisvárossá változott át, melynek viktoriánus mintáján
hangulatos éttermek várják az ideérkező kirándulókat. A lekváros csészében
a Graves család már több nemzedék óta főzi finom édességeit,
híres gyümölcsdzsemjeit.

Visszatérünk a parkútra, és nemsokára elérjük a vízesést. A 800-900
méter széles katlan körül a korláttal övezett sétány most is nedves,
folyamatosan érkezik felénk a szitáló vízpára. Azon a pontján, ahol a
dübörgő víz egy méterrel előttem hömpölyög le az 50 méteres mélybe,
hosszan nézek bele az átlátszó, üvegzöld vízbe. Együtt áradok a nagy
erővel érkező, majd a peremen átforduló és zuhanó víztömeggel. Az
előretóduló megfeszített erő a homlokpárkányon nagy levegőt vesz,
majd elernyed, és teljesen ellazultan megadja magát a föld vonzerejének.
A relaxáció olyan energiával tölt fel, hogy bármilyen hegynek
neki tudnék most indulni. De még közelebb szeretnék kerülni a messze
dübörgő hanghoz. A feláldozott lány emlékére a Hableány nevet
viselő hajón esőköpenybe burkolózva imbolygok a karéj mélyén. Minél
közelebb érek a félköríves fehér tajtékhoz, annál kevesebbet látok.
Betakar és magába fogad a Niagara. Vizet lélegzem, és a fentről érkező
robaj átrohan rajtam, kimosva a felgyülemlett salakanyagot. Minden
megkeményedés fellazul, kövek pörögnek le a folyóba. A kikötőbe
visszaérve remegő térdekkel, a patkó fölött átívelő szivárványba
kapaszkodva szállok ki a hajóból. A mélység után következhet a magasság.
A Skylon-torony üvegfalú gyorsliftjén érkezem a körbeforgó éttermébe.
Ebéd közben már nyugodt szívveréssel nézek le a vízipipázó kanyarulatba.

A Niagara-vízesés mellett elterülő, gondos kezek által ápolt parkban
a sűrű, élénkzöld fűben barna, cirmos vagy fekete chipmunkok\footnote{%
amerikai mókus}
futnak sebesen. Pillanatok alatt eltűnnek a lombkoronákban. Hosszú,
dús szőrzetű farkukat félkörívben húzzák maguk alá, ránézésre
megegyezik testük méretével. Sokszor rendeznek több százezer
virágszálat is számláló kiállítást a nagy területen, de e fesztiválok
közötti időben is pihentető a kert harmóniája. Felfrissülve térek vissza
a városba.

Torontó nagy kiterjedésű, folyamatosan növekvő, nyugodt mozgású
város. Szellős, tágas terek, parkok jellemzik virágágyásokkal,
szobrokkal. A szétnyitott hengerre hasonlító városháza előtti füvön
elnyúlhatunk, és az ég felé fordulhatunk, mint Henry Moore absztrakt
bronzszobra, az \textit{Íjász}. Közadakozásból vásárolták meg a torontóiak.
Ugye nem meglepő, hogy Torontó Budapesthez hasonlóan a Nyilas
városok körébe tartozik? A füvön hanyatt dőlve élvezzük a város derűs
duruzsolását. A városháza szétnyitott hengeréből mint egy tegezből
dobálhatták ki a nyílvesszőket, a Downtown (Belváros) nyúlánk
toronyházait. Azonban ezek a felhőkarcolók sem nyomják agyon a központot.
Az ég felé törő hivatalok között a Royal Bank Plaza szárnyaló
üvegtornyának tollazata átlátszatlan füstüveg, így veri vissza a bézs,
arany, fehér és acélkék árnyalatait. Az aranymadár tollazatának
vázát csillogó fémdarabok alkotják. Gazdag ruházatú, de állig
begombolkozott szárnyas. Alatta bevásárlóközpontok labirintusát találjuk.
Nemcsak a belváros föld alatti világát jellemzik shopping centerek, de
a város több pontján, például a metróállomások közelében is föld alatti
üzletsorok között keressük a kijáratot. A földfelszínre sokszor nem
törnek fel a bevásárlónegyedek, az utcaszintre lépve iroda- és lakóházakat
vagy szállodát találunk. Persze léteznek bevásárlóutcák is,
ilyen az 1977-ben megnyílt Eaton bevásárlóközpont környéke. Üvegteteje
alatt hatvan kanadai vadlúd szoborcsoportja húz délre jellegzetes
V alakzatban, ezzel szimbolikusan összekötve az Eaton központot
az Ontario-tóval. A Tejutat Égi útnak is hívják, ahogy a tó fölött ívelő
híd neve is az, Skyway. A Tejút másik elnevezése egyes mitológiákban
Vadludak ösvénye. A Nyilas pedig tejúti csillagkép!

Emelkedjünk is magasabbra, és nem is egy kicsit. A CN Tower a
világ legmagasabb szabadon álló szerkezete. 553 méteres magasságával
majdnem az Eiffel-torony kétszeresére nyúlik fel, az Empire State
Buildingnél is 110 méterrel magasabb. Ötvennyolc másodperc alatt
repülünk fel az üvegfalú lifttel a belső kilátóig. Két emelettel
magasabban vacsorázunk. A torony csészealj alakú négy kilátószintjének
egyike 351 méter magasságban az éttermi emelet, ahonnan a kivilágított
város körpanorámája tárul elénk, mialatt az étterem 72 perc alatt
körbeforog a tengelye körül. Mielőtt visszaindulnánk saját bolygónkra
vagy a Tejútra, vegyünk részt egy intergalaktikus űrszondaversenyen a
George Lucas filmvállalata által tervezett szimulátoron, vagy egy
stratégiai lézerjátékban, ahol az arénában speciális fény- és
hangeffektusok közepette küzdhetünk.

A Nyilas szellemi horizontot keres. Ennek megfelelően a Science
Center élő laboratóriumában nyugodtan kipróbálhatjuk a berendezési
tárgyakat. Működés közben megtapasztalhatjuk, hogy át tudunk-e
lőni lézersugarat egy téglafalon. Különösen tetszett az orvostudománnyal,
az emberi testtel kapcsolatos kézzelfogható modellek terme.
A pálmát mégis a hurrikán vitte el, amelyet gombnyomásra be tudtam
indítani, és kicsinyített hű mása előttem süvített fel. A Tudományok
Háza egy csodavilág, ahol a mindennapi életben természetesnek vett
életfunkciók és természeti jelenségek működésének mélyére látunk,
a legegyszerűbb és legbonyolultabb technikai eszközök mechanikáját
ismerhetjük meg. Ezzel, csodálatos világunk felfedezésével zárult
második látogatásom Kanadában.

Harmadszorra a szakmai világszövetség éves kongresszusára érkezem
meghívott előadóként. A reptér melletti szállodában töltöm a pár
napot, a városba csak hivatalos ügyben utazom be. Hazafelé azonban
New York-i megállással repülök.
